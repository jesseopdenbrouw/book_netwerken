%%%%%%%%%%%%%%%%%%%%%%%%%%%%%%%%%%%%%%%%%%%%%%%%%%%%%%%%%%%%%%%%%%%%%%%%%%%%%%%%%%%%
%%%
%%%   GELIJKSPANNINGSTHEORIE - OPGAVEN
%%%
%%%%%%%%%%%%%%%%%%%%%%%%%%%%%%%%%%%%%%%%%%%%%%%%%%%%%%%%%%%%%%%%%%%%%%%%%%%%%%%%%%%%


\begin{enumerate}[labelindent=0pt,labelwidth=\widthof{8.88.\ },label=\textbf{\thechapter.\arabic*.},leftmargin=!,ref=\thechapter.\arabic*]

\setcounter{figure}{0}
\setcounter{table}{0}
\setcounter{equation}{0}
\setcounter{lstlisting}{0}
\makeatletter 
\renewcommand{\thefigure}{P\thechapter.\@arabic\c@figure}
\renewcommand{\thetable}{P\thechapter.\@arabic\c@table}
\renewcommand{\theequation}{P\thechapter.\@arabic\c@equation}
\renewcommand{\thelstlisting}{P\thechapter.\@arabic\c@lstlisting}
\makeatother

\item
\label{que:gelstroomwet1}
Gegeven is het netwerk in figuur~\ref{fig:gelstroomwet1} met stroomtakken. Bereken de stroom $I$.

\begin{figure}[!ht]
\centering
\begin{tikzpicture}[line width=1pt]
\draw (-2,1) to [short, i=\SI{5}{\ampere}, -*] (0,0);
\draw (-1.5,-1.5) to [short, i=\SI{2}{\ampere}, -*] (0,0);
\draw (2,0.5) to [short, i=\SI{3}{\ampere}, -*] (0,0);
\draw (1,-1.5) to [short, i=$I$, -*] (0,0);
\end{tikzpicture}
\caption{Stroomvoerende verbindingen.}
\label{fig:gelstroomwet1}
\end{figure}


\item
\label{que:gelstroomwet2}
Gegeven is het netwerk in figuur~\ref{fig:gelstroomwet2} met stroomtakken. Bereken de stroom $I$.

\begin{figure}[!ht]
\centering
\begin{tikzpicture}[line width=1pt]
\draw (-2,1) to [short, i=\SI{3}{\ampere}, -*] (0,0);
\draw (-1.5,-1.5) to [short, i<=\SI{5}{\ampere}, -*] (0,0);
\draw (2,0.5) to [short, i<=\SI{2}{\ampere}, -*] (0,0);
\draw (1,-1.5) to [short, i<=$I$, -*] (0,0);
\end{tikzpicture}
\caption{Stroomvoerende verbindingen.}
\label{fig:gelstroomwet2}
\end{figure}

\item
\label{que:gelspanningswet1}
Gegeven is de netwerken in figuur~\ref{fig:gelspanningswet1} met spanningsbronnen. Bereken van beide netwerken de spanning $U_X$.

\begin{figure}[!ht]
\centering
\begin{tikzpicture}[bookcircuit]
\draw (0,0) to[V, v=\SI{10}{\volt}] ++(0,2)
            to[V, v<=\SI{-5}{\volt}] ++(2,0)
            to[R, v^=$U_x$] ++(0,-2)
            to[V, v<=\SI{7}{\volt},-.] ++(-2,0) 
;
\draw (6,0) to[V, v<=\SI{3.5}{\volt}] ++(0,2)
            to[V, v<=\SI{2.7}{\volt}] ++(2,0)
            to[R, v^=$U_y$] ++(0,-2)
            to[V, v<=\SI{-1.9}{\volt},-.] ++(-2,0) 
;
\end{tikzpicture}
\caption{Spanningsbronnen in een kring.}
\label{fig:gelspanningswet1}
\end{figure}%

\item
\label{que:gelserie1}
Drie weerstanden van \SI{1.2}{\kilo\ohm}, \SI{2.7}{\kilo\ohm} en \SI{3.9}{\kilo\ohm} zijn in serie geschakeld. Bepaal de totale weerstand.

\item
\label{que:gelparallel1}
Drie weerstanden van \SI{2.7}{\kilo\ohm}, \SI{5.6}{\kilo\ohm} en \SI{8.2}{\kilo\ohm} zijn parallel geschakeld. Bepaal de totale weerstand.

\item
\label{que:gelserieparallel1}
Gegeven is het netwerk in figuur~\ref{fig:gelserieparallel1}. Bepaal de totale weerstand tussen de punten A en B.

\begin{figure}[!ht]
\centering
\begin{tikzpicture}[bookcircuit]
\draw (0,0) node[left] {A} 
			to[R=\SI{3.3}{\kilo\ohm}, o-*] ++(2,0) node (1) {}
			to[short] ++(0,0.5)
			to[R=\SI{4.7}{\kilo\ohm}] ++(2,0)
			to[short,-*] ++(0,-0.5) node (2) {}
			to[short,-o] ++(1,0) node[right] {B}
		(1) to[short] ++(0,-0.5)
			to[R=\SI{5.6}{\kilo\ohm}] ++(2,0)
			to[short] (2)
;
\end{tikzpicture}
\caption{Netwerk van weerstanden.}
\label{fig:gelserieparallel1}
\end{figure}

\item
\label{que:gellatticenetwork}

Gegeven is het weerstandsnetwerk in figuur~\ref{fig:gellatticenetwork}. Bepaal de vervangingsweerstand tussen de punten A en B.

\begin{figure}[!ht]
\centering
\begin{tikzpicture}[bookcircuit]
\draw (0,0) node [left] {A}
	to[short,o-*]                  ++(1,0) node (1) {}
    to[R, R=\SI{70}{\kilo\ohm},-*] ++ (0,-2)
(1) to[R, R=\SI{20}{\kilo\ohm},-*] ++(2,0) node (2) {}
    to[R, R=\SI{100}{\kilo\ohm},-*] ++(0,-2)
(2) to[R, R=\SI{80}{\kilo\ohm},-*] ++(2,0) node (3) {}
    to[R, R=\SI{30}{\kilo\ohm},-*] ++(0,-2)
(3) to[R, R=\SI{40}{\kilo\ohm}] ++(2,0)
    to[R, R=\SI{20}{\kilo\ohm}]    ++(0,-2)
	to[short,-o]                   (0,-2) node (4) [left] {B}
;
\end{tikzpicture}
\caption{Laddernetwerk van weerstanden.}
\label{fig:gellatticenetwork}
\end{figure}

\iffalse
Dit weerstandsnetwerk wordt ook wel \textsl{laddernetwerk} genoemd en bestaat uit een combinatie van parallel- en serieweerstanden. De niet-vereenvoudigde formule is:
\begin{equation}
 R_{AB} = 60\parallel(35+(50\parallel(25+50\parallel(30+20)))) \qquad\text{(alle waarden in \si{\kilo\ohm})}
\end{equation}
waarbij $\parallel$ de zogenaamde paralleloperator is. Het uitrekenen is eenvoudig door steeds serie- en parallelberekeningen uit te voeren. De achterste twee weerstanden (\SI{20}{\kilo\ohm}+\SI{30}{\kilo\ohm}) levert \SI{50}{\kilo\ohm} op en dat staat weer parallel aan de achterste \SI{50}{\kilo\ohm}. Dus dat levert \SI{25}{\kilo\ohm} op. Dat staat dan weer in serie met de \SI{25}{\kilo\ohm} (midden-boven) en levert dus weer \SI{50}{\kilo\ohm} etc. Uitwerken levert uiteindelijk op:
\begin{equation}
R_{AB} = \SI{30}{\kilo\ohm}
\end{equation}
\fi

\item
\label{que:gelspanningsdeling1}
Gegeven is het netwerk in figuur~\ref{fig:gelspanningsdeling1}. Bepaal de spanningen $U_1$ en $U_2$.

\begin{figure}[!ht]
\centering
\begin{tikzpicture}[bookcircuit]
\draw (0,0) to[V, v=\SI{12}{\volt}] ++(0,2)
			to[short] ++(0.5,0)
            to[R=\SI{6}{\ohm}, v=$U_1$] ++(2,0)
			to[short] ++(0.5,0)
			to[R=\SI{18}{\ohm}, v=$U_2$] ++(0,-2)
			to[short,-.] (0,0)
;
\end{tikzpicture}
\caption{Netwerk met spanningsbron en weerstanden.}
\label{fig:gelspanningsdeling1}
\end{figure}



\item
\label{gel:spanningsdeling1}
Gegeven is het netwerk in figuur~\ref{fig:gelspanningsdeling2}. De interne weerstand van de spanningsmeter is \SI{10}{\mega\ohm}. Bereken spanning die de spanningsmeter meet.

\begin{figure}[!ht]
\centering
\begin{tikzpicture}[bookcircuit]
\draw (0,0) to[V, v=\SI{24}{\volt}] ++(0,2)
			to[R=\SI{1}{\mega\ohm},-*] ++(2,0) node (1) {}
			to[R=\SI{1}{\mega\ohm},-*] ++(0,-2) node (2) {}
			to[short,-.] (0,0)
		(1) to[short] ++(2,0)
			to node[draw,circle,fill=white] {V} ++(0,-2)
			to [short,-.] (0,0)
;
\end{tikzpicture}
\caption{Netwerk van weerstanden en spanningsmeter.}
\label{fig:gelspanningsdeling2}
\end{figure}

% Vmeter = (1//10) / (1+(1//10))*24 = 11,428571428571428571428571428571 V

\item
\label{que:gelstroomdeling3}
Gegeven is het netwerk in figuur~\ref{fig:gelstroomdeling3}. Bereken de stromen $I_1$ en $I_2$.

\begin{figure}[!ht]
\centering
\begin{tikzpicture}[bookcircuit]
\draw (0,0) to[I, l=\SI{15}{\milli\ampere}] ++(0,2)
			to[short,-*] ++(2,0) node (A) {}
			to[R=\SI{680}{\ohm},-*,i>^=$I_1$] ++(0,-2)
		(A) to[short] ++(2,0) 
			to[R=\SI{820}{\ohm},i>^=$I_2$] ++(0,-2)
			to[short,-.] (0,0)
;
\end{tikzpicture}
\caption{Netwerk met stroombron en weerstanden.}
\label{fig:gelstroomdeling3}
\end{figure}

% I_1 = 8,2 mA, I_2 = 6,8 mA

\item
\label{que:gelstroomdeling4}
Gegeven is het netwerk in figuur~\ref{fig:gelstroomdeling4}. Bereken de stromen $I_1$ t/m $I_4$.

\begin{figure}[!ht]
\centering
\begin{tikzpicture}[bookcircuit]
\draw (0,0) to[I, l=\SI{80}{\milli\ampere}] ++(0,2)
			to[short,-*] ++(2,0) node (A) {}
			to[R=\SI{120}{\ohm},-*,i>^=$I_1$] ++(0,-2)
		(A) to[short] ++(1.5,0) node (B) {} 
			to[R=\SI{68}{\ohm},-*,i>^=$I_2$] ++(0,-2)
		(B) to[short] ++(1.5,0) node (C) {}
			to[R=\SI{82}{\ohm},-*,i>^=$I_3$] ++(0,-2)
		(C) to[short] ++(1.5,0) 
			to[R=\SI{100}{\ohm},i>^=$I_4$] node (D) {} ++(0,-2)
			to[short,-.] (0,0)
;
\end{tikzpicture}
\caption{Netwerk met stroombron en weerstanden.}
\label{fig:gelstroomdeling4}
\end{figure}

%% I_1 = 14,738 mA, I_2 = 26,008 mA, I_3 = 21,5679 mA, I_4 = 17,68568 mA
%% I_1 = 14,74      I_2 = 26,01      I_3 = 21,57       I_4 = 17,69 mA       = 80,01 mA
%% I_1 = 14,7       I_2 = 26,0       I_3 = 21,6        I_4 = 17,7           = 80,0

\item
\label{que:gelspanningsdeling2}
Gegeven is het netwerk in figuur~\ref{fig:gelspanningsdeling3}. Bereken de stroom $I$ die de bron levert.
Bereken de spanning $U$ die over de weerstand van \SI{5}{\ohm} staat.
\begin{figure}[!ht]
\centering
\begin{tikzpicture}[bookcircuit]
\draw (0,0) to[V=\SI{18}{\volt}] ++(0,2)
			to[short,-*,i=$I$] ++(2,0) node (A) {}
			to[R=\SI{12}{\ohm},-*] ++(0,-2)
		(A) to[short,-*] ++(1,0) node (B) {}
            to[short] ++(0,0.5)
			to[R=\SI{18}{\ohm}] ++(4,0)
			to[short,-*] ++(0,-.5) node (C) {} 
        (B) to[short] ++(0,-0.5) 
			to[R=\SI{4}{\ohm}] ++(2,0)
			to[R=\SI{5}{\ohm},v=$U$] ++(2,0)
			to[short] ++(0,0.5)
			to[short] ++(0.5,0)
			to[R=\SI{6}{\ohm}] ++(0,-2)
			to[short,-.] (0,0) 
;
\end{tikzpicture}
\caption{Netwerk met stroombron en weerstanden.}
\label{fig:gelspanningsdeling3}
\end{figure}

\item
\label{que:gelmaxspanning1}
Bereken de maximale spanning die over een weerstand van \SI{1}{\kilo\ohm}/\SI{ 0.25}{\watt} mag worden geplaatst.

%% $P=$\dfrac{U^2}{R} $\leftarrow$ U = \sqrt{P\cdotR}  U = \sqrt{0.25$\cdot$1000} = \SI{15.8}{\volt}$

\item
\label{que:gelmaxspanning2}
Drie weerstanden van \SI{680}{\ohm}, \SI{560}{\ohm} en \SI{820}{\ohm}, alledrie met een maximaal vermogen van \SI{0.25}{\watt}, worden in serie geplaatst. Bereken de maximale spanning die over de drie weerstanden geplaatst mag worden.

%% Imax = sqrt{0.25/820} = 17.5 mA --> Vmax = 36 V

\item
\label{que:gelinterneweerstand1}
Een batterij heeft in onbelaste toestand een spanning van \SI{9.5}{\volt}. Als de batterij belast wordt met een weerstand van \SI{680}{\ohm} zakt de spanning naar \SI{7.3}{\volt}. Bereken de interne weerstand van de batterij.

% Ibelast = 7,3/680 = 10,73529411764705882352941176471 mA
% Ri = (9,5-7,3)/10.735e-3 = 204,93... Ohm \approx 205 ohm

\item
\label{que:gelkarimp1}
Gegeven is het netwerk in figuur~\ref{fig:gelkarimp1}. De waarden voor de weerstanden is als volgt: $R_1 = \SI{100}{\ohm}$, $R_2 = \SI{150}{\ohm}$ en $R_L = \SI{200}{\ohm}$. Bepaal de vervangingsweerstand tussen de punten A en B.

\begin{figure}[!ht]
\centering
\begin{tikzpicture}[bookcircuit]
\draw (0,0) node [left] {A}
    to[R, R=$R_1$,-*] ++(2,0) node (2) {} node[above] {X}
    to[R, R=$R_2$,-*] ++(0,-2)
(2) to[R, R=$R_1$]    ++(2,0) node (3) {}  node[above] {Y}
    to[R, R=$R_L$]    ++(0,-2)
	to[short,-o]        (0,-2) node (4) [left] {B}
;
\end{tikzpicture}
\caption{Netwerk van weerstanden.}
\label{fig:gelkarimp1}
\end{figure}

\item
\label{que:gelkarimp2}
Gegeven is het netwerk in figuur~\ref{fig:gelkarimp1}. De waarden voor de weerstanden is als volgt: $R_1 = \SI{25}{\ohm}$, $R_2 = \SI{100}{\ohm}$ en $R_L = \SI{75}{\ohm}$. Bepaal de vervangingsweerstand tussen de punten A en B.


\item
\label{que:gelkarimp3}
Gegeven is het weerstandennetwerk in figuur~\ref{fig:gelkarimp1}. De waarden van de weerstanden zijn als in opgave~\ref{que:gelkarimp2}. Tussen de punten A en B worden een ideale spanningsbron aangesloten met de waarde $U_{AB}$ = \SI{1}{\volt}. Bepaal de spanningen op punt X en Y (dus $U_{XB}$ en $U_{YB}$).

\itemstar 
\label{que:gelkarimp4}
Gegeven is het weerstandennetwerk in figuur~\ref{fig:gelkarimp1}. We stellen nu dat $R_{AB}$ gelijk is aan $R_L$, dus als we het netwerk inkijken vanuit de punten A en B dan meten we dezelfde waarde als $R_L$ (ja dat kan, zie de uitkomsten van opgave~\ref{que:gelkarimp1} en~\ref{que:gelkarimp2}). Bewijs nu dat geldt:
%
\begin{equation}
\label{equ:gelkarimp4}
R_L^2 = \sqrt{R_1^2 + 2R_1R_2}
\end{equation}
%
Hint: stel eerst de vergelijking op voor de vervangingsweerstand $R_{AB}$, dus zoiets als $R_{AB} = \ldots$ (een functie van $R_1$, $R_2$ en $R_L$). Vul daarna voor $R_{AB}$ gewoon $R_L$ in, dus dan krijgen we iets van $R_L = \ldots$ (een functie van $R_1$, $R_2$ en $R_L$). Daarna is het gewoon wat eenvoudige wiskunde. Stug doorrekenen en zorgen dat de basisregels van de rekenkunde netjes gehanteerd worden.

\itemstar 
\label{que:gelkarimp5}
Gegeven is het weerstandennetwerk in figuur~\ref{fig:gelkarimp1}. We stellen nu dat $R_{AB}$ gelijk is aan $R_L$, dus als we het netwerk inkijken vanuit de punten A en B dan meten we dezelfde waarde als $R_L$ (ja dat kan, zie de uitkomsten van opgave~\ref{que:gelkarimp1} en~\ref{que:gelkarimp2}). Bewijs nu dat geldt:
%
\begin{equation}
\label{equ:gelkarimp5}
\dfrac{U_{YB}}{U_{AB}} = \dfrac{R_L-R_1}{R_L+R_1}
\end{equation}

Hint: deze is ondoenlijk, maar wel leuk. Hierbij komt echt inzicht kijken.

\itemstar
\label{que:gelkarimp6}
Een bekende kabelexploitant heeft onlangs de signaalsterkte van het tv-signaal verhoogd. Daardoor kan de tv het signaal niet meer goed verwerken. De oplossing ligt in het gebruik van het netwerk in figuur~\ref{fig:gelkarimp1}.
Hierbij wordt de weerstand $R_L$ vervangen door een coaxiale kabel. De weerstand van de coaxiale kabel is \SI{75}{\ohm}. Verder verzwakken we het signaal 2 keer (de versterking is dus~\num{0.5}). We kunnen nu~\eqref{equ:gelkarimp4} en~\eqref{equ:gelkarimp5} gebruiken om de waarden van $R_1$ en $R_2$ uitrekenen. Bepaal deze waarden.

\item
\label{equ:gelstroomdeling1}
Gegeven is het netwerk in figuur~\ref{fig:gelstroomdeling1}. Het netwerk bestaat uit een parallelschakeling van een weerstand van \SI{3.3}{\kilo\ohm} en een weerstand van \SI{5.6}{\kilo\ohm}. In serie met deze (deel-)schakeling staat een onbekende weerstand $R_X$. De bron levert een spanning van \SI{48}{\volt} en een onbekende stroom $I_X$. De deelstroom door de weerstand van \SI{5.6}{\kilo\ohm} is \SI{5}{\milli\ampere}. Bepaal de waarden voor $R_X$ en $I_X$.

\begin{figure}[!ht]
\centering
\begin{tikzpicture}[bookcircuit]
%%%\draw (0,0) to[V, v<=\SI{12}{\volt}]   ++(0,4) node (1) {} node [left] {}
%%%			to[short, -*, i=\SI{10}{\milli\ampere}] ++(2,0) node (1) {}
%%%            to[R,R=\SI{5}{\kilo\ohm}, i=\SI{2}{\milli\ampere}] ++(0,-2)
%%%			to[short, -*] ++(1,0) node (2) {}
%%%			to[R,R=$R_Y$] ++(0,-2)
%%%			to[short, -.] (0,0)
%%%        (1) to[short] ++(2,0)
%%%            to[R, R=$R_X$] ++(0,-2)
%%%            to[short] (2) 
\draw (0,0) to[V, v=\SI{48}{\volt}]   ++(0,4) node (1) {} node [left] {}
            to[short, i=$I_x$] ++(3,0)
			to[R=$R_X$, -*] ++(0,-2) node (1) {}
			to[short] ++(-1,0)
			to[R=\SI{3.3}{\kilo\ohm},-*] ++(0,-2)
        (1) to[short] ++(1,0)
            to[R=\SI{5.6}{\kilo\ohm}, i=\SI{5}{\milli\ampere}] ++(0,-2)
			to[short,-.] (0,0)
;
\end{tikzpicture}
\caption{Netwerk met spanningsbron en weerstanden.}
\label{fig:gelstroomdeling1}
\end{figure}

% U_56 = 5.6*5 = 28 V
% I_39 = 28/3.9 = 8.4848484848... = 8.5 mA
% I_X  = 8.5+5.0 = 13.5 mA
% R_X  = (48-28) / 13.484848484 = 1483 ohm

\item
\label{que:gelthevenin1}
Gegeven zijn de netwerken in figuur~\ref{fig:gelnetwerkvoortheveninanalysis1}. Bepaal voor elk netwerk het Thévenin-vervangingsnetwerk.


\begin{figure}[!ht]
\centering
\begin{tikzpicture}[bookcircuit]
\draw (0,1) to[V, v=\SI{12}{\volt}] ++(0,2) %node (1) {} node [left] {}
            to[R=\SI{10}{\kilo\ohm}, -*] ++(2,0) node (1) {}
			to[short,-o] ++(1,0) node[right] {A}
        (1) to[R=\SI{5}{\kilo\ohm},-*] ++(0,-2)
			to[open,-o] ++(1,0) node[right] {B}
			to[short,-.] ++(-3,0)
;
\draw (6,0) to[V,v=\SI{12}{\volt}] ++(0,2) node(1) {}
			to[V,v=\SI{6}{\volt}] ++(0,2)
			to[R=\SI{12}{\kilo\ohm},-*] ++(2,0) node (2) {}
			to[short,-o] ++(1,0) node[right] {A}
		(2) to[R=\SI{18}{\kilo\ohm},-*] ++(0,-4)
			to[open,-o] ++(1,0) node[right] {B}
			to[short,-.] ++(-3,0)
		(1) to[short] ++(1,0)
		    node[sground]{}
;
\end{tikzpicture}
\caption{Netwerken met spanningsbron en weerstanden.}
\label{fig:gelnetwerkvoortheveninanalysis1}
\end{figure}

% U_AB_open =U_th= 5/(10+5)*12 = 4 V
% I_k = 12/10kohm = 1,2 mA
% Rth = 4/1,2 mA = 3,33.. kohm

\item 
\label{que:gelthevenin6}
Gegeven zijn de netwerken in figuur~\ref{fig:gelthevenin6}. Bepaal de kortsluitstroom $I_k$ en de interne weerstand (th\'evenin- en norton-weerstand) gezien tussen de klemmen A en B.

\begin{figure}[!ht]
\centering
\begin{tikzpicture}[bookcircuit]
\draw (0,0) to [V=\SI{10}{\volt}] ++(0,2)
			to [R=\SI{12}{\kilo\ohm}] ++(2,0) node (1) {}
			to [R=\SI{18}{\kilo\ohm},*-*] ++(0,-2)
		(1) to [R=\SI{33}{\kilo\ohm},-o] ++(2,0) node[right] {A}
			to [open] ++(0,-2) node[right] {B}
			to [short,o-.] (0,0)
;
\draw (6.5,0) to[I,l=\SI{0.8}{\milli\ampere}] ++(0,2)
			to [R=\SI{5.6}{\kilo\ohm}] ++(2,0) node (1) {}
			to [R=\SI{8.2}{\kilo\ohm},*-*] ++(0,-2)
		(1) to [R=\SI{10.0}{\kilo\ohm},-o] ++(2,0) node[right] {A}
			to [open] ++(0,-2) node[right] {B}
			to [short,o-.] (6.5,0)
;
\end{tikzpicture}
\caption{Twee netwerken.}
\label{fig:gelthevenin6}
\end{figure}


\item
\label{que:gelthevenin5}
Gegeven is het netwerk in figuur~\ref{fig:gelthevenin3}. Bepaal de stroom $I_X$ door de spanningsbronnen om te werken naar stroombronnen en dan stroomdeling toe te passen.

\begin{figure}[!ht]
\centering
\begin{tikzpicture}[bookcircuit]
\draw (0,0) to[V, v=\SI{18}{\volt}]   ++(0,2) %node (1) {} node [left] {}
            to[R=\SI{10}{\ohm},-*] ++(2,0) % node (A) {} node[above] {$U_x$}
			to[R=\SI{15}{\ohm}] ++(2,0)
			to[V, v<=\SI{6}{\volt}]   ++(0,-2) %node (1) {} node [left] {}
            to[short, -.] (0,0)
        (A) to[R=\SI{12}{\ohm},i>^=$I_x$] ++(0,-2)
;
\end{tikzpicture}
\caption{Netwerk met spanningsbronnen en weerstanden.}
\label{fig:gelthevenin3}
\end{figure}

% I_X = 0,73 A (- 11/15 A)

\item
\label{que:gelthevenin2}
Gegeven is het netwerk in figuur~\ref{fig:gelnetwerkvoortheveninanalysis2}. Bepaal het Thévenin-vervangingsnetwerk tussen de punten A en B. 

\begin{figure}[!ht]
\centering
\begin{tikzpicture}[bookcircuit]
\draw (0,0) to[V, v=\SI{12}{\volt},-*]   ++(0,2) node (1) {} node [left] {}
            to[R, R=\SI{1.2}{\kilo\ohm}, -*] ++(2,0) node (2) {} node [above] {A}
			to[R, R=\SI{3.9}{\kilo\ohm}, -*] ++(2,0) node (3) {} node [right] {}
			to[R, R=\SI{2.7}{\kilo\ohm}]     ++(0,-2)
			to[short]          (0,0)
	    (1) to[short]          ++(0,1)
            to[R, R=\SI{3.3}{\kilo\ohm}]     ++(4,0)	
			to[short]          ++(0,-1)
		(2) to[R, R=\SI{1.8}{\kilo\ohm}, -*] ++(0,-2) node [below] {B}
%			to node[sground]{} ++(0,0)
%        (3) to[I, l=$I$, invert]       ++(2,0)
;
\end{tikzpicture}
\caption{Netwerk met spanningsbron en weerstanden.}
\label{fig:gelnetwerkvoortheveninanalysis2}
\end{figure}

%%
%% Uth = 10.7973 V, Ik = 11.00 mA, Rth = 981.57 \ohm. nodal8.asc


\item
\label{que:gelthevenin2b}
Gegeven is het netwerk in figuur~\ref{fig:gelnetwerkvoortheveninanalysis2}. De weerstand tussen A en B is nu \SI{1.8}{\kilo\ohm}. Deze weerstand wordt vervangen door een andere weerstand zodanig dat maximale vermogensoverdracht plaatsvindt in deze weerstand. Bepaal de waarde van de vervangende weerstand.

\item
\label{que:gelthevenin4}
Gegeven is het netwerk in figuur~\ref{fig:gelthevenin4}. Bepaal het thévenin-vervangingsschema tussen de punten A en B.
\begin{figure}[!ht]
\centering
\begin{tikzpicture}[bookcircuit]
\draw (0,0) to[R=\SI{30}{\ohm},-*] ++(0,2) node (1) {}
            to[V, v<=\SI{10}{\volt},-*] ++(2,0) node (2) {}
            to[V, v<=\SI{10}{\volt},-*] ++(2,0) node (3) {} node[right] {A}
			to[R=\SI{30}{\ohm}] ++(0,-2) node[right] {B}
			to[short] (0,0)
		(1) to[short] ++(0,1.5)
			to[R=\SI{15}{\ohm}] ++(4,0)
			to[short] ++(0,-1.5)
		(2) to[R=\SI{30}{\ohm},-*] ++(0,-2)
;
\end{tikzpicture}
\caption{Netwerk met spanningsbronnen en weerstanden.}
\label{fig:gelthevenin4}
\end{figure}


\item
\label{que:gelsuperpos1}
Gegeven is het netwerk in figuur~\ref{fig:gelsuperpos1}. Bepaal de spanning $U_x$ door middel van superpositie.

\begin{figure}[!ht]
\centering
\begin{tikzpicture}[bookcircuit]
\draw (0,0) to[V=\SI{6}{\volt}] ++(0,2)
			to[R=\SI{4}{\ohm}] ++(3,0) node[above] {$U_x$}
			to[short] ++(2,0)
			to[R=\SI{8}{\ohm}] ++(0,-2)
			to[short] ++(-2,0) node (1) {}
			to[short,*-.] (0,0)
		(1) to[I,l=$\dfrac{1}{4}$\si{\ampere},*-*] ++(0,2)
;
\end{tikzpicture}
\caption{Netwerk met spanningsbron, stroombron en weerstanden.}
\label{fig:gelsuperpos1}
\end{figure}

%% Ubron: Ux = 4 V
%% Ibron: Ux = 2/3 V
%% Uxtotaal = 4 2/3 V


\item
\label{que:gelthevenin3}
Gegeven is het netwerk in figuur~\ref{fig:gelnetwerkvoortheveninanalysis3}. Bereken het maximale vermogen dat in $R_X$ gedissipeerd kan worden.

\begin{figure}[!ht]
\centering
\begin{tikzpicture}[bookcircuit]
\draw (0,0) to[V=\SI{12}{\volt}]   ++(0,4) %node (1) {} node [left] {}
            to[short,-*] ++(2,0) node (1) {}
            to[R=\SI{10}{\ohm},-*] ++(0,-2) node (A) {} node[left] {A}
			to[R=\SI{20}{\ohm},-*] ++(0,-2) node (C) {}
        (1) to[short] ++(2,0)
            to[R=\SI{40}{\ohm},-*] ++(0,-2) node (B) {} node[right] {B}
            to[R=\SI{20}{\ohm}] ++(0,-2)
            to[short, -.] (0,0)
        (A) to[R=$R_X$] (B)
		(C) to node[sground] {} ++(0,0)
;
\end{tikzpicture}
\caption{Netwerk met spanningsbron en weerstanden.}
\label{fig:gelnetwerkvoortheveninanalysis3}
\end{figure}

\item
\label{que:gelnodal1}
Gegeven is het netwerk in figuur~\ref{fig:gelnetwerkvoortheveninanalysis3}. Neem $R_X=\SI{30}{\ohm}$. Bereken de spanningen op de punten A en B met behulp van de knooppuntspanningsmethode.

% VA = 7.46667, VB = 5.06667, nodal9.asc

\item
\label{que:gelnodal2}
Gegeven is het netwerk in figuur~\ref{fig:gelnodal2}. Een ontwerper wil de spanning $U_x$ op \SI{3}{\volt} hebben.
Bepaal de waarde van $R_x$.

\begin{figure}[!ht]
\centering
\begin{tikzpicture}[bookcircuit]
\draw (0,0) to[V, v=\SI{8}{\volt}]   ++(0,2) %node (1) {} node [left] {}
            to[R=\SI{2}{\ohm},-*] ++(2,0) node (A) {} node[above] {$U_x$}
			to[R=\SI{4}{\ohm}] ++(2,0)
			to[V, v<=\SI{6}{\volt}]   ++(0,-2) %node (1) {} node [left] {}
            to[short, -.] (0,0)
        (A) to[R=$R_x$] ++(0,-2)
;
\end{tikzpicture}
\caption{Netwerk met spanningsbronnen en weerstanden.}
\label{fig:gelnodal2}
\end{figure}

%% (Ux-U1)G1 + (Ux-0)G3 + (Ux-U2)G2 = 0
%%
%% G2 = (U1-Ux)G1 + (U2-Ux)G2 / Ux
%%
%% Rx = 12/13 Ohm

\item
\label{que:nodal3}
Gegeven is het netwerk in figuur~\ref{fig:gelnodal3}. Bepaal de spanning $U_2$ en $U_x$ als functie van de overige parameters. Gebruik hiervoor de knooppuntspanningsmethode.

\begin{figure}[!ht]
\centering
\begin{tikzpicture}[bookcircuit]
\draw (0,0) to[V=$U_1$] ++(0,4)
            to[short] ++(2,0)
            to[R=$R_{in}$, v=$U_{in}$,-*] ++(0,-2) node (1) {} node[left] {$U_x$}
            to[R=$R_1$,-*] ++(0,-2)
			node[sground] {}
            to[short,-.] ++(-2,0)
            to[open] ++(6.5,4) %node (3) {} node[right] {$U_{2}$}
            to[R=$R_{out}$,] ++(-2,0)
            to[cV, v_<=$aU_{in}$] ++(0,-2) node (2) {}
			to[short] ++(0,-2) node (2) {}
			to[short] ++(-2.5,0)
		(1) to[short] ++(2.5,0)
			to[R=$R_2$] ++(2,0)
			to[short,-*] ++(0,2)
			to[short, -o] ++(1,0) node[right] {$U_2$}
		(2) to[short, *-o] ++(3,0)
;
\end{tikzpicture}
\captionof{figure}{Opamp-netwerk.}
\label{fig:gelnodal3}
\end{figure}

\item
\label{que:gelwheatstone1}
Gegeven is het netwerk in figuur~\ref{fig:gelwheatstone1}. Bepaal de waarde van $R$ als geldt dat de weerstand van \SI{1.5}{\kilo\ohm} stroomloos is.

\begin{figure}[!ht]
\centering
\begin{tikzpicture}[bookcircuit]
\draw (0,0) to[V=\SI{12}{\volt},-*]   ++(0,2) node (1) {} node [left] {}
            to[R=\SI{1.2}{\kilo\ohm}, -*] ++(2,0) node (2) {} node [above] {A}
			to[R=\SI{1.5}{\kilo\ohm}, -*] ++(2,0) node (3) {} node [right] {B}
			to[R=\SI{2.2}{\kilo\ohm}]     ++(0,-2)
			to[short]          (0,0)
	    (1) to[short]          ++(0,1)
            to[R=\SI{1.8}{\kilo\ohm}]     ++(4,0)	
			to[short]          ++(0,-1)
		(2) to[R=$R$, -*] ++(0,-2) %node [below] {B}
%			to node[sground]{} ++(0,0)
%        (3) to[I, l=$I$, invert]       ++(2,0)
;
\end{tikzpicture}
\caption{Netwerk met spanningsbron en weerstanden.}
\label{fig:gelwheatstone1}
\end{figure}

\itemstar
In figuur~\ref{fig:gelcube1} is een kubus van weerstanden (niet getekend) te zien. De weerstand in iedere ribbe is $R$. De punten A en B zijn lichaamsdiagonaal tegenover elkaar geplaatst. Bepaal de weerstandswaarde tussen de punten A en B.

\begin{figure}[!ht]
\centering
\begin{tikzpicture}[bookcircuit,scale=2]
\draw (0,0,0) -- ++(0,0,-1) -- ++(0,1,0) -- ++(0,0,1) -- cycle;
\draw (1,0,0) -- ++(0,0,-1) -- ++(0,1,0) -- ++(0,0,1) -- cycle;
\draw (0,0,0) -- ++(1,0,0);
\draw (0,0,-1) -- ++(1,0,0);
\draw (0,1,0) -- ++(1,0,0);
\draw (0,1,-1) -- ++(1,0,0);
\draw[fill=black] (0,1,0) circle [radius=1pt] node[left] {A};
\draw[fill=black] (1,0,-1) circle [radius=1pt] node[right] {B};
\end{tikzpicture}
\caption{Kubus met  weerstanden.}
\label{fig:gelcube1}
\end{figure}

Hint: Plaats een denkbeeldige spanningsbron van \SI{1}{\volt} tussen de punten A en B en stel de vergelijkingen op voor de onbekende spanningen in het netwerk. Op deze manier kan de stroom die de denkbeeldige bron levert worden berekend en dus ook de weerstand die de bron ondervindt.

\end{enumerate}
