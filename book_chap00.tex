%%%%%%%%%%%%%%%%%%%%%%%%%%%%%%%%%%%%%%%%%%%%%%%%%%%%%%%%%%%%%%%%%%%%%%%%%%%%%%%
%%%
%%%   VOORWOORD
%%%
%%%%%%%%%%%%%%%%%%%%%%%%%%%%%%%%%%%%%%%%%%%%%%%%%%%%%%%%%%%%%%%%%%%%%%%%%%%%%


\chapter{Voorwoord}
\label{cha:voorwoord}


\subsubsection*{Leeswijzer}

\subsubsection*{Studiewijzer}
%%%In onderstaande tabel wordt een overzicht gegeven van de stof die een student
%%%bij een eerste introductie onderwezen zou moeten krijgen. Uiteraard is iedereen
%%%vrij om zelf de onderwerpen te kiezen.
%%%
%%%\begin{tabbing}
%%%\hspace{1em}\=\hspace{3cm}\=\kill
%%%\ifusebookpartI
%%%   \> \textbf{Inleiding} \> \\
%%%   \>  Hoofdstuk 1 \> 1.1--1.6, 1.7 (alleen 1.7.1), 1.8--1.12  \\ 
%%%   \>  Hoofdstuk 2 \> 2.1--2.11, 2.13, 2.15 \\ 
%%%   \>  Hoofdstuk 3 \> 3.1--3.6, 3.9, 3.10 \\ 
%%%   \>  Hoofdstuk 4 \> 4.1--4.3 (alleen 4.3.1 en 4.3.2), 4.4--4.6, 4.8--4.10 \\ 
%%%   \>  Hoofdstuk 5 \> 5.1--5.3, 5.6--5.19, 5.21, 5.23 \\ 
%%%   \>  Hoofdstuk 6 \> 6.1--6.3, 6.5, 6.6, 6.8--6.12, 6.15--6.18 \ifusebookpartII \\ \fi
%%%\fi
%%%\ifusebookpartII
%%%   \>              \> \\
%%%   \> \textbf{Geavanceerd 1} \> \\
%%%   \>  Hoofdstuk 7 \> helemaal \\
%%%   \>  Hoofdstuk 8 \> 8.1, 8.3, 8.6 (aanbevolen 8.2, 8.4 en 8.5) \\
%%%   \>  Hoofdstuk 9 \> helemaal \\
%%%   \>  Hoofdstuk 10 \> helemaal \ifusebookpartIII \\ \fi
%%%\fi
%%%\ifusebookpartIII
%%%	\ifusebookpartI\ifusebookpartII\else \\ \fi\fi
%%%   \>              \> \\
%%%   \> \textbf{Geavanceerd 2} \> \\
%%%   \>  Hoofdstuk 11 \> helemaal \\
%%%   \>  Hoofdstuk 12 \> helemaal \\
%%%   \>  Hoofdstuk 13 \> helemaal
%%%\fi
%%%\end{tabbing}


\subsubsection*{Verantwoording inhoud}
%%%\ifusebookasbook
%%%Dit boek voldoet aan acht van de negen punten die opgesomd zijn bij het aandachtsgebied
%%%Digitale techniek in de basis Body of Knowlegde and Skills (BoKS) Elektrotechniek
%%%zoals is vastgelegd door de HBO Engineering. Alleen het onderdeel AD/DA ontbreekt. De
%%%BoKS is te vinden via~\cite{hboengineering2016boks}.
%%%\fi

\subsubsection*{Website}
Op de website \url{http://ds.opdenbrouw.nl} zijn slides, practicumopdrachten
en aanvullende informatie te vinden. De laatste versie van dit boek wordt
hierop gepubliceerd. Er zijn ook voorbeeldprojecten voor de Quartus-software
van Altera te vinden. De projecten kunnen vaak zonder aanpassingen op het
DE0-bordje van Terasic uitgeprobeerd worden.

\subsubsection*{Dankbetuigingen}

%\bigskip%
\hfill \author, \the\year.
