%%%De spanning over een spoel kan berekend worden met:
%%%\begin{equation}
%%%U = L\dfrac{\text{d}i}{\text{d}t}
%%%\end{equation}
%%%Hierin is $L$ de zelfinductie van de spoel. Bij een sinusvormige spanning met hoekfrequentie
%%%$\omega$ kunnen we dus schrijven:
%%%\begin{equation}
%%%U = L\dfrac{\text{d}\, i \sin \omega t}{\text{d}t} = \omega L i\cos \omega t
%%%\end{equation}
%%%De amplitude van de spanning is dus afhankelijk van de hoekfrequentie $\omega$.
%%%
%%%De spanning over een condensator kan berekend worden met:
%%%\begin{equation}
%%%U = \dfrac{1}{C}\int i\,\text{d}t 
%%%\end{equation}
%%%
%%%Hierin is $C$ de capaciteit van de condensator. Bij een sinusvormige spanning met hoekfrequentie
%%%$\omega$ kunnen we schrijven:
%%%\begin{equation}
%%%U = \dfrac{1}{C}\int i\sin \omega t\,\text{d}t = -\dfrac{i}{\omega C} \cos \omega t
%%%\end{equation}
