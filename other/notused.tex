%%%We berekenen nu eerst de stroom die de bron moet leveren. Daarvoor moeten we de weerstand bepalen
%%%die de bron ziet. De weerstand wordt gevormd door de parallelschakeling van $R_p$ met $R_u$, in serie
%%%met $R_s$. We kunnen dus voor $I$ schrijven dat:
%%%%
%%%\begin{equation}
%%%I = \dfrac{U}{R_s + \dfrac{R_p\cdot R_u}{R_p+R_u}}
%%%\end{equation}
%%%%
%%%Vervolgens berekenen we de stroom $I_u$ door gebruik te maken van stroomdeling:
%%%%
%%%\begin{equation}
%%%I_u = I\cdot\dfrac{R_p}{R_p+R_u}
%%%\end{equation}
%%%%
%%%Combineren we de twee vorige vergelijkingen dan vinden we:
%%%%
%%%\begin{equation}
%%%I_u = \dfrac{U}{R_s + \dfrac{R_p\cdot R_u}{R_p+R_u}}\cdot\dfrac{R_p}{R_p+R_u}
%%%\end{equation}
%%%%
%%%We mogen in bovenstaande vergelijking $R_s$ ook schrijven als:
%%%%
%%%\begin{equation}
%%%R_s = \dfrac{R_s\cdot(R_p+R_u) + R_p\cdot R_u}{R_p+R_u} = \dfrac{R_s\cdot R_p + R_s\cdot R_u + R_p\cdot R_u}{R_p+R_u}
%%%\end{equation}
%%%%
%%%
%%%%
%%%\begin{equation}
%%%I_u = \dfrac{U\cdot\dfrac{R_p}{R_p+R_u}}{\dfrac{R_s\cdot R_p + R_s\cdot R_u + R_p\cdot R_u}{R_p+R_u}}
%%%\end{equation}
%%%%
%%%Zowel in de teller als de noemer wordt gedeeld door $R_p+R_u$. We kunnen dit tegen elkaar wegstrepen:
%%%%
%%%\begin{equation}
%%%I_u = \dfrac{U\cdot R_p}{R_s\cdot R_p + R_s\cdot R_u + R_p\cdot R_u}
%%%\end{equation}
%%%%
%%%De delen nu teller en noemer door $R_s+R_p$:
%%%%
%%%\begin{equation}
%%%I_u = \dfrac{U\cdot\dfrac{R_p}{R_s+R_p}}{\dfrac{R_s\cdot R_p + R_s\cdot R_u + R_p\cdot R_u}{R_s+R_p}}
%%%    = \dfrac{U\cdot\dfrac{R_p}{R_s+R_p}}{\dfrac{R_s\cdot R_p + R_u\cdot(R_s+R_p)}{R_s+R_p}}
%%%    = \dfrac{U\cdot\dfrac{R_p}{R_s+R_p}}{\dfrac{R_s\cdot R_p}{R_s+R_p}+R_u} 
%%%\end{equation}
%%%%
%%%Over deze laatste vergelijking kunnen we twee opmerkingen maken. De teller vertegenwoordigt een spanning en
%%%wel de open klemspanning van de schakeling. We noemen deze spanning de th\'eveninspanning $U_T$.
%%%De noemer beschrijft een weerstandswaarde gevormd door de parallelschakeling van $R_s$ en $R_p$ in serie
%%%met $R_u$. De parallelschakeling van $R_s$ en $R_p$ wordt de th\'eveninweerstand  $R_T$ genoemd. We krijgen dus:
%%%%
%%%\begin{equation}
%%%U_T = U\cdot\dfrac{R_p}{R_s+R_p} \qquad\text{en}\qquad R_T = \dfrac{R_s\cdot R_p}{R_s+R_p}
%%%\end{equation}
%%%%
%%%De stroom $I_u$ wordt dus beschreven met de vergelijking:
%%%%
%%%\begin{equation}
%%%I_u = \dfrac{U_T}{R_T+R_u}
%%%\end{equation}
%%%%


%%%We hebben hier te maken met een niet-ideale spanningsbron uit paragraaf~\ref{sec:gelnietidealespanningsbron}.

%%%\begin{tikzpicture}[bookcircuit]
%%%\draw (0,0) to [cV<=$2I_1$] +(0,2);
%%%\end{tikzpicture}