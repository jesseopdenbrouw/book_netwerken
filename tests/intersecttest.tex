% used PGFPlots v1.16
\documentclass[border=5pt]{standalone}
\usepackage{pgfplots}
    \usetikzlibrary{intersections}
    \pgfplotsset{
        % use this `compat' level or higher, so TikZ coordinates don't have to
        % be prefixed by `axis cs:'
        compat=1.11,
    }
\begin{document}
\begin{tikzpicture}
    \begin{axis}[
        yticklabel style={
            /pgf/number format/fixed,
        },
    ]
        \addplot [name path global=GaussCurve]
            gnuplot [domain=48.00:56.00,samples=100]
                {exp(-0.5*((x-52.64)/1.82)**2)/(sqrt(2*pi)*1.82)};
        \path [name path global=HelperLine]
            (48,0.13288) -- (56,0.13288)
                coordinate [at start]   (start)
        ;
        \draw [dashed,name intersections={of=GaussCurve and HelperLine}]
            (start) -- (intersection-2) -- (intersection-2 |- 0,0)
                % -------------------------------------------------------------
                % using `\pgfplotspointgetcoordinates' stores the (axis)
                % coordinates of e.g. the coordinate (intersection-2) in
                % `data point', which then can be called by `\pgfkeysvalueof'
%                node [at start,below left] {
%                    \pgfplotspointgetcoordinates{(intersection-2)}
%                    $(
%                        \pgfmathprintnumber[fixed]{\pgfkeysvalueof{/data point/x}},
%                        \pgfmathprintnumber[fixed]{\pgfkeysvalueof{/data point/y}}
%                    )$
%                }
                % -------------------------------------------------------------
        ;
        \fill [red] (intersection-2) circle (1mm);
    \end{axis}
\end{tikzpicture}
\end{document}