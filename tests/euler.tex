\documentclass[12pt,fleqn]{article}

\usepackage[left=1in,right=1in,top=1in,bottom=1.5in]{geometry}
\usepackage{parskip}
\usepackage{mathtools}
\usepackage{amssymb}
\setlength{\mathindent}{1em}
\newcommand\strutrule{\rule{0mm}{1.9ex}}
\usepackage{lmodern}
\usepackage{microtype}
\usepackage{float}
\usepackage{xcolor}
\usepackage{tikz}
\usetikzlibrary{decorations.pathreplacing}

%% Uning tcolorbox for Infobox environment settings
\usepackage{tcolorbox}
\colorlet{infoboxline}{red!55!black} % Global example color
\colorlet{infoboxbg}{brown!10}
\colorlet{infoboxbgtitle}{brown!20}
\newtcolorbox{infobox}[1]{title=#1,colback=infoboxbg,colframe=infoboxline,fonttitle=\large\scshape,colbacktitle=infoboxbgtitle,coltitle=black,toptitle=5pt,bottomtitle=5pt,sharp corners,boxrule=2pt,titlerule=0pt,top=7pt}

%%
%% https://tex.stackexchange.com/questions/476045/euler-and-minus-sign
%%
%% Imaginary unit, e, and Euler
\newcommand\imaginaryunit{j}                  % the imaginary unit, i for mathematician and theoretical physicist, j for the rest of the world.
\newcommand\imunit{\mathrm{\imaginaryunit}}   % ... in upright math
\newcommand\ce{\mathrm{e}}                    % the constant e, upright of course
\newcommand{\fiximunit}{\if\imaginaryunit j\,\fi}
\makeatletter
\newcommand{\epowim}[1]{\ce^{\epowim@#1}}
\newcommand{\epowim@}{\@ifnextchar-{\epowim@@}{\epowim@@{\fiximunit}}}
\newcommand{\epowim@@}[1]{#1\imunit}
\makeatother
%\def\Re{\mathrm{Re}}  % Redefine \Re
%\def\Im{\mathrm{Im}}  % Redefine \Im




\def\uangle{40}    % angle of unit vector between 30 and 55


\begin{document}

\begin{infobox}{De formule van Euler}
De formule van Euler legt een relatie tussen de goniometrische functies en de complex exponentiële functie. Voor elk reëel getal $\alpha$ geldt:
%
\begin{equation}
\epowim{\alpha} = \cos\alpha + \imunit\sin\alpha
\end{equation}
%
Hierin is $\ce$ de grondtal van de natuurlijk logaritme, $\imunit$ de imaginaire eenheid en zijn $\cos$ en $\sin$ de functies cosinus en sinus met het argument in radialen.\newline

De formule geldt ook voor complexe getallen:
%
\begin{equation}
\ce^{\beta+\imunit\alpha} = \ce^{\beta} \cdot \epowim{\alpha} = \ce^{\beta}(\cos\alpha + \imunit\sin\alpha)
\end{equation}
\end{infobox}

\begin{infobox}{Uitbeelding van de formule van Euler}
\begin{center}
\begin{tikzpicture}[scale=2.5,font=\footnotesize] % scale is radius in cm
\draw[thick] (0,0) circle (1cm);
\draw[-latex,thick] (0,-1.1) -- (0,1.1) node[above] {$\Im$};
\draw[-latex,thick] (-1.1,0) -- (1.1,0) node[right] {$\Re$};
\draw[-latex,thick] (0,0) --(\uangle:1) node[midway,above] {1};
\draw[dashed] (\uangle:1) -| (0,0);
\draw[dashed] (\uangle:1) |- (0,0);
\draw[decorate,decoration={brace,amplitude=4pt,mirror,raise=2pt}] (0,0) -- node[below,yshift=-2mm]{$\cos\alpha$} ({cos(\uangle)},0);
\draw[decorate,decoration={brace,amplitude=4pt,raise=2pt}] (0,0) -- node[left,xshift=-2mm]{$\sin\alpha$} (0,{sin(\uangle)});
\draw[-latex] (0.3,0) arc (0:\uangle:0.3);
\node at (\uangle/2:0.4) {$\alpha$};
\node at (1,0) [xshift=50pt,right,font=\normalsize] {$\epowim{\alpha} = \cos\alpha + \imunit\sin\alpha$};
\end{tikzpicture}
\end{center}
\end{infobox}

\begin{infobox}{De identiteit van Euler}
Als we bij de formule van Euler voor de hoek $\alpha$ het getal $\pi$ invullen, krijgen we een bijzondere vergelijking:
%
\begin{equation}
\epowim{\pi} + 1 = 0
\end{equation}
%
Deze vergelijking legt de relatie tussen de vijf constanten $\ce$, $\pi$, $0$, $1$ en $\imunit$ en de operaties optellen, vermenigvuldigen, machtsverheffen en gelijkstellen. Deze identiteit wordt ook wel ``de mooiste vergelijking van het universum'' genoemd.
\end{infobox}

%$ \epowim{(\omega t + \varphi)} =\cos (\omega t + \varphi) + \imunit\sin (\omega t + \varphi)  $

\begin{infobox}{Bewijs van de formule van Euler}
We kunnen de formule van Euler bewijzen door gebruik te maken van de reeksontwikkeling van $\ce^x$:
%
\begin{equation}
\ce^x = \dfrac{x^0}{0!} + \dfrac{x^1}{1!} + \dfrac{x^2}{2!} + \dfrac{x^3}{3!} + \dfrac{x^4}{4!} + \dfrac{x^5}{5!} + \dfrac{x^6}{6!} + \dfrac{x^7}{7!} + \cdots
\end{equation}
%
Vullen we voor $x = \imunit\alpha$ in, dan krijgen we:
%
\begin{equation}
\jot=10pt
\begin{split}
\epowim{\alpha} &= \dfrac{(\imunit\alpha)^0}{0!} + \dfrac{(\imunit\alpha)^1}{1!} + \dfrac{(\imunit\alpha)^2}{2!} + \dfrac{(\imunit\alpha)^3}{3!} + \dfrac{(\imunit\alpha)^4}{4!} + \dfrac{(\imunit\alpha)^5}{5!} + \dfrac{(\imunit\alpha)^6}{6!} +\dfrac{(\imunit\alpha)^7}{7!} + \cdots \\
 &= 1 + \imunit\alpha + \imunit^2 \dfrac{\alpha^2}{2!} + \imunit^3 \dfrac{\alpha^3}{3!} + \imunit^4 \dfrac{\alpha^4}{4!} + \imunit^5 \dfrac{\alpha^5}{5!} + \imunit^6 \dfrac{\alpha^6}{6!} + \imunit^7 \dfrac{\alpha^7}{7!} + \cdots \\
 &= 1 + \imunit\alpha -\dfrac{\alpha^2}{2!} -\imunit \dfrac{\alpha^3}{3!} + \dfrac{\alpha^4}{4!} + \imunit \dfrac{\alpha^5}{5!}  -\dfrac{\alpha^6}{6!} - \imunit \dfrac{\alpha^7}{7!} \cdots \\
 &= \underbrace{\left(1 - \dfrac{\alpha^2}{2!} + \dfrac{\alpha^4}{4!} -\dfrac{\alpha^6}{6!} + \cdots\right)}_{\cos\alpha}\, +\,\, \imunit\underbrace{\left( \alpha -\dfrac{\alpha^3}{3!} + \dfrac{\alpha^5}{5!} - \dfrac{\alpha^7}{7!} + \cdots \right)}_{\sin\alpha} \\
 &= \cos\alpha + \imunit\sin\alpha
\end{split}
\end{equation}
%
We gaan stilzwijgend ervan uit dat alle rekenkundige operaties ook gelden voor complexe getallen. Euler heeft zijn stelling op deze manier aangetoond.
\end{infobox}

\begin{infobox}{{Wrong, so wrong \ldots}}
In veel boeken wordt geschreven dat $\imunit$ gelijk is aan de wortel van $-1$:
%
\begin{equation}
\imunit = \sqrt{-1}
\end{equation}
%
Dit is absoluut fout. De wortel uit een getal $x$ is gedefinieerd als een niet-negatief getal $y$ waarvoor geldt dat het kwadraat van $y$ gelijk is aan $x$. 
%In de veronderstelling dat $\imunit=\sqrt{-1}$ is, kunnen we nu aantonen dat $\imunit^2$ gelijk is aan $1$:
%%
%\begin{equation}
%\imunit^2 = \sqrt{-1}\cdot\sqrt{-1} = \sqrt{-1\cdot-1} = \sqrt{1} = 1
%\end{equation}
%%
%We gaan hier ten onrechte ervan uit dat geldt dat $\sqrt{a}\sqrt{b}=\sqrt{ab}$ ook geldt voor negatieve getallen.
De enige juiste definitie is:
%
\begin{equation}
\imunit^2 = -1
\end{equation}

\end{infobox}

\begin{infobox}{De stellling van De Moivre}
Voor elke geheel getal $n$ geldt:
%
\begin{equation}
\left(\cos \alpha + \imunit\sin \alpha\right)^n = \cos n\alpha + \imunit\sin n\alpha
\end{equation}
%
We kunnen dat eenvoudig aantonen:
%
\begin{equation}
\left(\cos \alpha + \imunit\sin \alpha\right)^n = \epowim{n\alpha} = \cos n\alpha + \imunit\sin n\alpha
\end{equation}

Zoals gezegd geldt dit echter voor een geheel getal $n$.
\end{infobox}

\begin{infobox}{{Hoeken, hoeken en meer hoeken}}
Volgens de stelling van De Moivre moet gelden dat:
%
\begin{equation}
\cos 2\alpha + \imunit\sin 2\alpha = (\cos \alpha + \imunit\sin\alpha)^2
\end{equation}
%
De gelijkheid is alleen waar als de reële en imaginaire delen links en rechts van het isgelijkteken aan elkaar gelijk zijn. We vermenigvuldigen het rechterlid uit:
%
\begin{equation}
\begin{split}
\cos 2\alpha + \imunit\sin 2\alpha &= (\cos \alpha + \imunit\sin\alpha)^2 \\
 &= \cos^2 \alpha + 2\imunit\cos\alpha\sin\alpha + \imunit^2\sin^2\alpha \\
 &= \cos^2 \alpha - \sin^2\alpha + 2\imunit\cos\alpha\sin\alpha \\
 &= \cos^2 \alpha - (1-\cos^2 \alpha) + 2\imunit\cos\alpha\sin\alpha \\
 &= 2\cos^2 \alpha - 1 + 2\imunit\cos\alpha\sin\alpha
\end{split}
\end{equation}
%
Nu splitsen we de reële en imaginaire delen:
%
\begin{equation}
\cos 2\alpha = 2\cos^2 \alpha - 1 \qquad\text{en}\qquad \sin 2\alpha = 2\cos\alpha\sin\alpha
\end{equation}

Dit zijn de zogenoemde \textsl{dubbele hoekformules} van sinus en cosinus. We kunnen de
machten uitbreiden en vinden dan:
%
\begin{equation}
\begin{array}{lll}
\cos 3\alpha = 3\cos^3\alpha - 3\cos\alpha &\text{en} &\sin 3\alpha = 3\sin\alpha - 4\sin^3 \alpha\\
\cos 4\alpha = 8\cos^4 \alpha - 8\cos^2 \alpha + 1 &\text{en} &\sin 4\alpha = 4 \sin\alpha\cos\alpha - 8\sin^3\alpha\cos\alpha
\end{array}
\end{equation}
%
Voor de \textsl{cosinus} volgen de expansies de zogenoemde \textsl{Tsjebysjev-polynomen}\footnote{Ook wel geschreven als: Chebyshev, Chebyshov, Tschebyschow, Tchebichef of Tchebycheff}:
%
\begin{equation}
\begin{split}
T_0(x) &= 1 \\
T_1(x) &= x \\
T_2(x) &= 2x^2 - 1 \\
T_3(x) &= 4x^3 - 3x \\
T_4(x) &= 8x^4 - 8x^2 + 1 \\
T_5(x) &= 16x^5 - 20x^3 + 5x \\
T_6(x) &= 32x^6 - 48x^4 + 18x^2 - 1 \\
%T_7(x) &= 64x^7 - 112x^5 + 56x^3 - 7x \\
%T_8(x) &= 128x^8 - 256x^6 + 160x^4 - 32x^2 + 1 \\
%T_9(x) &= 256x^9 - 576x^7 + 432x^5 - 120x^3 + 9x 
\end{split}
\end{equation}

waarbij in het linkerlid $T_n(x)$ staat voor $\cos n\alpha$ en het rechterlid $x$ staat voor $\cos\alpha$. Tsjebysjev-polynomen spelen een belangrijke rol bij het ontwerpen van elektronische filters.
\end{infobox}

\begin{infobox}{Relatie met hyperbolische functies}
De hyperbolische sinus- en cosinusfuncties zijn gedefinieerd als:
%
\begin{equation}
\cosh x = \dfrac{\ce^x + \ce^{-x}}{2} \qquad\text{en}\qquad \sinh x = \dfrac{\ce^x - \ce^{-x}}{2}
\end{equation}
%
Vervangen we $x$ door $\imunit\alpha$ dan volgt:
%
\begin{equation}
\cosh \imunit\alpha = \dfrac{\epowim{\alpha} + \epowim{-\alpha}}{2} = \dfrac{\cos\alpha + \imunit\sin\alpha + \cos\alpha - \imunit\sin\alpha}{2} = \dfrac{2\cos\alpha}{2}= \cos\alpha
\end{equation}
%
en
%
\begin{equation}
\sinh \imunit\alpha = \dfrac{\epowim{\alpha} - \epowim{-\alpha}}{2} = \dfrac{\cos\alpha + \imunit\sin\alpha - (\cos\alpha - \imunit\sin\alpha)}{2} = \dfrac{2\imunit\sin\alpha}{2} = \imunit\sin\alpha
\end{equation}
%
Verder geldt:
%
\begin{equation}
\begin{split}
\sinh\alpha &= -\imunit\sin(\imunit\alpha) \\
\cosh\alpha &= \cos(\imunit\alpha)
\end{split}
\end{equation}
%
Het is dus mogelijk om de sinus en cosinus van een imaginair getal te berekenen.
\end{infobox}

\begin{infobox}{Complex toegevoegde getal}
Als voor een complex getal $z$ geldt dat:
%
\begin{equation}
z = a + \imunit b
\end{equation}
%
dan is het \textsl{complex toegevoegde} of \textsl{complex geconjungeerde} getal $z^*$:
%
\begin{equation}
z^* = a - \imunit b
\end{equation}
%
Dan geldt vervolgens:
%
\begin{equation}
\begin{split}
z + z^* &= a + \imunit b + a -\imunit b = 2a \\
z - z^* &= a + \imunit b - (a -\imunit b) = 2\imunit b \\
z \cdot z^* &= (a + \imunit b)(a - \imunit b) = a^2 + b^2
\end{split}
\end{equation}
\end{infobox}

\begin{infobox}{Rekenen met complexe getallen}
Optellen en aftrekken van complexe getallen gaat als volgt:
%
\begin{equation}
\begin{split}
(a + \imunit b) + (c + \imunit d) &= (a + c) + \imunit(b + d) \\
(a + \imunit b) - (c + \imunit d) &= (a - c) + \imunit(b - d) \\
\end{split}
\end{equation}
%
Vermenigvuldigen van twee complexe getallen gaat als volgt:
%
\begin{equation}
\begin{split}
(a + \imunit b) \cdot (c + \imunit d) &= ac + \imunit ad + \imunit bc + \imunit^2 bd \\
 &= (ac - bd) + \imunit(ad + bc) \\
\end{split}
\end{equation}

%
Delen van complexe getallen gaat als volgt:
%
\begin{equation}
\begin{split}
\dfrac{a+\imunit b}{c+\imunit d} = \dfrac{a+\imunit b}{c+\imunit d} \cdot \dfrac{c-\imunit d}{c-\imunit d} = 
\dfrac{(ac + bd) + \imunit(bc-ad)}{c^2-d^2} \qquad (\text{met }c^2-d^2\neq 0)
\end{split}
\end{equation}
\end{infobox}

\end{document}

