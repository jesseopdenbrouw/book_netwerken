\documentclass{article}

\usepackage{mathtools}
\usepackage{tikz}
\usepackage{tikz-3dplot}
\usepackage{graphicx}

\begin{document}

\begin{figure}[!h]
% set the viewing angle
\def\a{45}
\tdplotsetmaincoords{65}{\a}
\centering
\begin{tikzpicture}
	[scale=7,
	tdplot_main_coords,
	axis/.style={->,black,very thin},
	curve/.style={black,thin},rotate=95]
	% radius and axes
	\def\radius{.1}
	\def\axissize{.3}
    % length
	\def\th{1}

	% intervals on the z-axis
%	\draw[axis,-,dashed] (0,0,0) -- (0,0,\th);
	\draw[axis,thick] (0,0,\th-0.5*\axissize) -- (0,0,\th+0.5*\axissize) node[anchor=south]{$I$};

	\tdplotsinandcos{\sintheta}{\costheta}{0}
	% upper end
	\tdplotdrawarc[curve,thick]{(0,0,\th)}{\radius*\costheta}{\a-360}{\a}{}{}
	% single sections
	\foreach \height in {0.5}{
    	% draw the front half with a solid line
		\tdplotdrawarc[curve,solid]{(0,0,\height)}{\radius*\costheta}{\a-180}{\a}{}{}
        % and the back half with a dotted line
        \tdplotdrawarc[curve,dotted]{(0,0,\height)}{\radius*\costheta}{\a-360}{\a-180}{}{}
		\tdplotdrawarc[curve,thick,fill=brown!50!red,opacity=0.5]{(0,0,\height)}{\radius*\costheta}{\a-360}{\a}{}{}
    }
	% lower end
	\tdplotdrawarc[curve,thick,dashed]{(0,0,0)}{\radius*\costheta}{\a}{\a+180}{}{}
	\tdplotdrawarc[curve,thick]{(0,0,0)}{\radius*\costheta}{\a}{\a-180}{}{}
	% right line and labels
	\tdplotsinandcos{\sintheta}{\costheta}{\a}
	\draw[thick] (\radius*\costheta,\radius*\sintheta,0) {} -- (\radius*\costheta,\radius*\sintheta,\th);
	\draw[latex-latex] (1.4*\radius*\costheta,\radius*\sintheta,0) -- node[above] {$L$} (\radius*\costheta,1.4*\radius*\sintheta,\th);
	% left line
	\tdplotsinandcos{\sintheta}{\costheta}{\a+180}
	\draw[thick] (\radius*\costheta,\radius*\sintheta,0) -- (\radius*\costheta,\radius*\sintheta,\th);
	
	% label of the diameter
	\draw[latex-latex] (\radius,0,\th/2) --  (-\radius,0,\th/2) node[anchor=south west,yshift=2mm,xshift=0.5mm] {$d$};
	\foreach \ang/\dis in {0/0,1/.06,-1/-0.04,-0.5/-0.1} {
		\shade[ball color=brown!50!red,opacity=0.90] (-\a+\ang:-1.5-\dis) circle (0.008cm) node[yshift=-1.5mm,font=\tiny] {\textsl{q}\raisebox{.7ex}{-}};
	}	
\end{tikzpicture}
\end{figure}

De weerstand van een metalen draad met lengte $L$ en diameter $d$ is:

\begin{eqnarray}
R = \frac{\rho L}{A}
\end{eqnarray}

waarbij $A$ de oppervlakte van de dwarsdoorsnede is met:

\begin{equation}
A = \dfrac{1}{4}\pi d^2
\end{equation}

\end{document}
% Thanks to Jörg Petzold for creating this example
