\documentclass{article}
\usepackage{parskip}
\usepackage{mathtools}
\usepackage[a4paper,left=1in,right=1in,top=1in,bottom=1.4in]{geometry}
\usepackage[charter]{mathdesign}
\usepackage[english]{babel}
\usepackage{tikz}
\usepackage{pgfplots}
\usetikzlibrary{spy}
\usetikzlibrary{backgrounds}
\usetikzlibrary{decorations}
\date{}
\begin{document}
\pagestyle{empty}
%
% declare extra background layer, (main = default)
\pgfdeclarelayer{background layer}
\pgfsetlayers{background layer,main}
\definecolor{darkgray}{rgb}{0.25,0.25,0.25}
\definecolor{lightgray}{rgb}{0.75,0.75,0.75}
%

\def\ll{1.0}
\def\rr{2.0}

\begin{figure}[!ht]
\centering
\begin{tikzpicture}
    [line cap=round,line join=round,scale=2,
     %using the 'spy' to magnify a part of the picture
     spy using outlines={circle,lens={scale=3}, size=4cm, connect spies},
     %using the decoration 'brace' (=a curly brace as path replacement)
     decoration={brace,amplitude=2pt},
	declare function={ f(\x) = (0.3*\x*\x);}]
	\coordinate (leftc) at (\ll,{f(\ll)});
	\coordinate (rightc) at (\rr,{f(\rr)});
	\coordinate (meet) at (\rr,{f(\ll)});

	\draw[domain=0:3,thick] plot (\x,{f(\x)}) node[below right] {$f(x)$};
	\draw[red] (leftc) circle (1pt) node[above left, black] {$A$};
	\draw[red] (rightc) circle (1pt) node[above left, black] {$B$};
	\draw[blue,shorten >=-1cm,shorten <=-1cm] (leftc) -- (rightc);

 \draw [decorate,decoration={brace,mirror,amplitude=4pt},color=green!30!black] (leftc)--(meet)
   node [midway,anchor=north,inner sep=5pt, outer sep=1pt]{$\Delta x$};

 \draw [decorate,decoration={brace,amplitude=4pt},color=violet] (rightc)--(meet)
   node [midway,anchor=west,inner sep=5pt, outer sep=1pt]{$\Delta y$};

  \spy [brown] on (2,0.6) in node [left] at (6.5,1);

  \begin{pgfonlayer}{background layer}
   \fill[yellow!20] (-0.5,-0.5) rectangle (4,3);
   %\node at (1,3)[blue]{\Large$ \mathrm{Difference \, Quotient}$};
  \end{pgfonlayer}
\end{tikzpicture}
\caption{Part of the function $f(x)=x^2$.}
\end{figure}

The slope of the line through the points $A$ and $B$ is:

\begin{equation}
\textrm{slope}_{AB}=\frac{\Delta y}{\Delta x} = \frac{f(x+\Delta x) - f(x)}{\Delta x}
\end{equation}

We want to determine the slope in point $A$. Therefore, we move point $B$ to point $A$. This means that $\Delta x$ will slowly become 0. This way, we get a \emph{limit}.

\begin{equation}
\textrm{slope}_A = \lim\limits_{\Delta x \rightarrow 0} \frac{f(x+\Delta x) - f(x)}{\Delta x}
\end{equation}

Note that this limit will give us a \emph{function}.



\subsubsection*{Example}
Find the slope in point $A(1,1)$ for the function $f(x) = x^2$.

First we calculate the limit funtion:

\begin{equation}
\jot=7pt
\begin{split}
\mathrm{slope}_A &= \lim\limits_{\Delta x \rightarrow 0}\frac{(x+\Delta x)^2 - x^2}{\Delta x} \\
 &= \lim\limits_{\Delta x \rightarrow 0}\frac{x^2 + 2x\Delta x + (\Delta x)^2 - x^2}{\Delta x} \\
 &= \lim\limits_{\Delta x \rightarrow 0}\frac{2x\Delta x + (\Delta x)^2}{\Delta x} \\
 &= \lim\limits_{\Delta x \rightarrow 0}2x + \Delta x \\
 &= 2x 
\end{split}
\end{equation}

The slope in $A(1,1)$ is 2.
\end{document}
