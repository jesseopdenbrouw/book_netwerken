
\documentclass[a4paper,12pt,fleqn]{book}

\usepackage{parskip}
\usepackage{mathtools}
\usepackage{siunitx}
\sisetup{output-decimal-marker = {,}}
%\usepackage{amssymb}
\setlength{\mathindent}{1em}
\DeclareMathSymbol{,}{\mathord}{letters}{"3B}
\abovedisplayskip=30pt
\belowdisplayskip=30pt
\abovedisplayshortskip=30pt
\belowdisplayskip=30pt
%%%%%%%%%%%%%%%%%%%%%%%%%%%%%%%%%%%%%%%%%%%%%%%%%%%%%%%%%%%%%%%%%%%%%
%%
%% TIKZ AND FRIENDS
%%
%%%%%%%%%%%%%%%%%%%%%%%%%%%%%%%%%%%%%%%%%%%%%%%%%%%%%%%%%%%%%%%%%%%%%
\usepackage{calc}
%\usepackage[betterproportions,siunitx]{circuitikz}
\usepackage{circuitikz-dutch}
%\usepackage[betterproportions]{circuitikz}
\usetikzlibrary{bending}
\usetikzlibrary{arrows.meta,arrows}
\usepackage{pgfplots}
\usetikzlibrary{intersections}
\usepgfplotslibrary{units}
\pgfplotsset{compat=1.16}

%\makeatletter
%\usepgflibrary{profiler}
%\pgfprofilenewforenvironment{tikzpicture}
%\pgfprofilenewforenvironment{equation}
%\pgfprofilenewforcommand{\pgfusepath}{1}
%%\pgfprofilenewforcommand{\pgf@circ@drawvoltagegeneric}{}
%\pgfprofilenewforcommand{\pgfnode}{5}
%\pgfprofilenewforcommand{\pgfmathparse}{1}
%\makeatother


% Directory prefix for Gnuplot plots
\tikzset{prefix=gnuplot/}
% Bipoles have the same line witdh as the wires
\ctikzset{bipoles/thickness=1}
% Style for schematics
\tikzset{bookcircuit/.style={line width=1pt,scale=1.25}}

% Get the x and y coordinates of a point
\makeatletter
\newcommand{\gettikzxy}[3]{%
  \tikz@scan@one@point\pgfutil@firstofone#1\relax
  \global\edef#2{\the\pgf@x}%
  \global\edef#3{\the\pgf@y}%
}
\makeatother

% Transform x coordinate to graph coordinate
\makeatletter
\newcommand\transformxdimension[1]{
    \pgfmathparse{((#1/\pgfplots@x@veclength)+\pgfplots@data@scale@trafo@SHIFT@x)/10^\pgfplots@data@scale@trafo@EXPONENT@x}
}
% Transform y coordinate to graph coordinate
\newcommand\transformydimension[1]{
    \pgfmathparse{((#1/\pgfplots@y@veclength)+\pgfplots@data@scale@trafo@SHIFT@y)/10^\pgfplots@data@scale@trafo@EXPONENT@y}
}
\makeatother

%%%\pgfplotsset{xticklabel={$\mathsf{\pgfmathprintnumber{\tick}}$},
%%%             yticklabel={$\mathsf{\pgfmathprintnumber{\tick}}$},
%%%             every axis/.append style={font=\sffamily}
%%%}

%%
%% Dutch inductors are like American inductors, but resistors are european
%%
\ctikzset{inductor = american}
\ctikzset{resistor = european}
%\pgf@circuit@bipole@voltage@straighttrue

\makeatletter

%%
%% Dutch independent current source
%%
%% Extra space for label
%\gdef\pgf@circ@res@cursep{0ex}
%
%%% 4 pt label separation from the arrow
%\ctikzset{bipoles/isource/labelsep/.initial=4}



%%%%%%%%%%%%%%%%%%%%%%%%%%%%%%%%%%%%%%%%%%%%%%%%%%%%%%%%%%%%%%%%%%%%%

%%
%% based on https://tex.stackexchange.com/questions/476045/euler-and-minus-sign
%%
%% Imaginary unit, e, and Euler
\newcommand\imaginaryunit{j}                   % the imaginary unit, i for mathematician and theoretical physicist,
                                               % j for the rest of the world.
\newcommand\imunit{\mathrm{\imaginaryunit}}    % ... in upright math
\newcommand\ce{\mathrm{e}}                     % the constant e, upright of course

\newcommand{\epowre}[1]{\ce^{#1}}              % e to the power of real
\makeatletter
\newcommand{\fiximunit@@}{\if\imaginaryunit j\,\fi}
\newcommand{\epowim}[1]{\ce^{\epowim@#1}}      % e to the power of imaginary
\newcommand{\epowim@}{\@ifnextchar-{\epowim@@}{\epowim@@{\fiximunit@@}}}
\newcommand{\epowim@@}[1]{#1\imunit}
\newcommand{\epowim@@@}{\@ifnextchar-{\epowim@@@@}{+\epowim@@@@{}}}
\newcommand{\epowim@@@@}[1]{#1\imunit}
\newcommand{\epowcom}[2]{\ce^{#1\epowim@@@#2}} % e to the power of complex
\newcommand{\cis}[1]{\cis@#1}                  % cos + imaginary sin
\newcommand{\cis@}{\@ifnextchar-{\cis@@@}{\cis@@}}
\newcommand{\cis@@}[1]{\cos#1 + \imunit\sin#1}
\newcommand{\cis@@@}[2]{\cos#2 - \imunit\sin#2}
%\renewcommand{\Re}{\mathrm{Re}}  % Redefine \Re
%\renewcommand{\Im}{\mathrm{Im}}  % Redefine \Im

\makeatother

\endinput


\begin{document}

Gegeven is het netwerk in figuur XXX. Bepaal het Thèvenin-vervangingsschema.

\begin{figure}[!ht]
\centering
\begin{tikzpicture}[bookcircuit]
\draw (0,0) to [V, V<=$\SI{12}{\volt}$] ++(0,2)
            to [R, R=$\SI{1.2}{\kilo\ohm}$, -*] ++(2,0) node (1) {} node [above] {$U_x$}
			to [R, R=$\SI{2.7}{\kilo\ohm}$, -*] ++(0,-2)
        (1) to [R, R=$\SI{3.3}{\kilo\ohm}$, -o] ++(2,0) node (a) [right] {A}
            to [open] ++(0,-2) node (b) [right] {B}
            to [short, o-.] (0,0)
;
\end{tikzpicture}
\caption{Netwerk voor berekenen Thèvenin-vervangingsschema.}
\end{figure}

De open klemspanning tussen de punten $A$ en $B$ bedraagt:
%
\begin{equation}
U_{TH} = U_{AB,open} = \dfrac{2700}{1200+2700}12 = \num{0.6923}\cdot12 = \SI{8.3077}{\volt}
\end{equation}
%
Om de kortsluitstroom tussen de punten $A$ en $B$ te berekenen moet punt $A$ met $B$ verbonden worden middels een kortsluiting. Om de stroom voor de weerstand van \SI{3.3}{\kilo\ohm} te berekenen, bepalen we eerst de spanning $U_x$:
%
\begin{equation}
U_{x,kort} = \dfrac{2700\parallel3300}{1200 + (2700\parallel3300)}\cdot12 = \dfrac{1485}{1200+1485}\cdot12 = \SI{6.6369}{\volt}
\end{equation}
%
Vervolgens kunnen de kortsluitstroom uitrekenen:
%
\begin{equation}
I_k = \dfrac{\num{6.6369}}{\num{3300}} = \num{2.0112e-3} = \SI{2.0112}{\mA}
\end{equation}
%
We vinden de Thèvenin-weerstand door de open klemspanning te delen door de kortsluitstroom:
%
\begin{equation}
R_{TH} = \dfrac{U_{AB,open}}{I_k} = \dfrac{\num{8.3077}}{\num{2.0112e-3}} = \SI{4131}{\ohm}
\end{equation}
%
Het vervangingsschema bestaat dus uit een spanningsbron van \SI{8.3077}{\volt} en een serieweerstand van \SI{4131}{\ohm}:

\begin{figure}[!ht]
\centering
\begin{tikzpicture}[bookcircuit]
\draw (0,0) to [V, V<=$\SI{8.3077}{\volt}$] ++(0,2)
            to [R, R=$\SI{4131}{\ohm}$, -o] ++(2,0) node (a) [right] {A}
            to [open] ++(0,-2) node (b) [right] {B}
            to [short, o-.] (0,0)
;
\end{tikzpicture}
\caption{Het Thèvenin-vervangingsschema.}
\end{figure}


\end{document}