\documentclass[a4paper]{article}

\usepackage{tikz}
\usepackage[european]{circuitikz}

%% Make a real good voltage source
\makeatletter
%% Output routine for voltage sources
\def\pgf@circ@drawvoltagegenerator{
	\ifpgf@circuit@bipole@voltage@below
		coordinate (pgfcirc@Vfrom) at ($(\ctikzvalof{bipole/name}.center) ! \ctikzvalof{voltage/bump a} ! (\ctikzvalof{bipole/name}.-120)$)
		coordinate (pgfcirc@Vto) at ($(\ctikzvalof{bipole/name}.center) ! \ctikzvalof{voltage/bump a} ! (\ctikzvalof{bipole/name}.-60)$)
	\else
		coordinate (pgfcirc@Vfrom) at ($ (\ctikzvalof{bipole/name}.center) ! \ctikzvalof{voltage/bump a} ! (\ctikzvalof{bipole/name}.120)$)
		coordinate (pgfcirc@Vto) at ($ (\ctikzvalof{bipole/name}.center) ! \ctikzvalof{voltage/bump a} ! (\ctikzvalof{bipole/name}.60)$)
	\fi
	\ifpgf@circuit@europeanvoltage
		\ifpgf@circuit@bipole@voltage@backward
				(pgfcirc@Vfrom)  node {$-$}  (pgfcirc@Vto) node {$+$}
		\else
				(pgfcirc@Vfrom)  node {$+$}  (pgfcirc@Vto) node {$-$}
		\fi
	\else% american voltage
		\ifpgf@circuit@bipole@voltageoutsideofsymbol
		% if it is a battery, must put + and -
			\ifpgf@circuit@bipole@voltage@backward
				(pgfcirc@Vfrom)  node {$-$}  (pgfcirc@Vto) node {$+$}
			\else
				(pgfcirc@Vfrom)  node {$+$}  (pgfcirc@Vto) node {$-$}
			\fi
		\fi
	\fi
}

%% Create a Dutch version of the current source
\makeatletter
\pgfcircdeclarebipole{}{\ctikzvalof{bipoles/isource/height}}{isource}{\ctikzvalof{bipoles/isource/height}}{\ctikzvalof{bipoles/isource/width}}{
    \pgfpointorigin
    \pgfsetlinewidth{\pgfkeysvalueof{/tikz/circuitikz/bipoles/thickness}\pgfstartlinewidth}
    \pgfpathellipse{\pgfpointorigin}{\pgfpoint{0}{\pgf@circ@res@up}}{\pgfpoint{\pgf@circ@res@left}{0}}
    \pgfpathmoveto{\pgfpoint{\pgf@circ@res@step}{\pgf@circ@res@up}}
    \pgfpathlineto{\pgfpoint{\pgf@circ@res@step}{\pgf@circ@res@down}}
  \pgfusepath{draw}
  \pgfscope
    \pgfsetarrowsend{latex'}
    \pgfpathmoveto{\pgfpoint{0.8\pgf@circ@res@left}{1.3\pgf@circ@res@up}}
    \pgfpathlineto{\pgfpoint{0.8\pgf@circ@res@right}{1.3\pgf@circ@res@up}}
    \pgfusepath{draw}   %draw arrow
  \endpgfscope
}
\makeatother

\begin{document}

\begin{figure}[!ht]
\begin{circuitikz}
%\tikzset{every node/.style={inner sep=1em}}
\draw (0,0) to[I, l_={$I_B$,},name=A, mirror, inner sep=1ex] (0,2);
\draw[red] (A.sw) rectangle (A.ne);
\draw (0,2) to[I, l={$I_B$},name=B, inner sep=10em] (2,2);
\draw[blue] (B.sw) rectangle (B.ne);
\draw (B.sw) -- (B.ne);
\end{circuitikz}
\end{figure}

\begin{figure}[!ht]
\centering
\begin{circuitikz}
\draw (0,0) to[V_>=$U_B$] (0,-2);
\draw (0,0) to[/tikz/circuitikz/bipoles/length=30pt, R=$R_i$, -*] (2,0);
\draw (2,0) to[short, i=$I_B$] (3,0);
\draw (3,0) to[/tikz/circuitikz/bipoles/length=30pt, R=$R_u$] (3,-2);
\draw (3,-2) to[/tikz/circuitikz/bipoles/length=30pt,short, -*] (2,-2);
\draw (2,-2) to[short] (0,-2);
\end{circuitikz}
\caption{Testcircit}
\end{figure}

\begin{circuitikz}
\draw (0,0) to[I, name=A] (0,2);
\node[left=2pt] at (A.n) {$I_A$};

\draw (0,0) to[I, name=B] (2,0);
\node[above=5pt] at (B.n) {$I_B$};
\end{circuitikz}

\begin{figure}[!ht]
\centering
\begin{circuitikz}
\draw (0,0) to[V_>=$1$V] (0,-2);
\draw (0,0) to[/tikz/circuitikz/bipoles/length=30pt, R=$25\ \Omega$, -*] (2,0);
\draw (2,0) to[short, i=$I_D$] (4,0);
\draw (4,0) to[/tikz/circuitikz/bipoles/length=20pt, empty diode, l=$D$] (4,-2);
\draw (4,-2) to[/tikz/circuitikz/bipoles/length=30pt,short, -*] (2,-2);
\draw (2,-2) to[short] (0,-2);
\draw[<->,shorten <=5pt,shorten >=5pt,thin] (2,0) -- (2,-1) node[anchor=east] {$U_D$} -- (2,-2);
\end{circuitikz}
\caption{Testcircuit}
\end{figure}



\end{document}