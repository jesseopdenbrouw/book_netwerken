\documentclass[a4paper]{article}

\usepackage{tikz}
\usepackage[european]{circuitikz}

%%%    \ctikzset{bipoles/resistor/height=0.15}
%%%    \ctikzset{bipoles/resistor/width=0.4}
%%%    %\ctikzset{resistor = european}

%% Make a real good voltage source
\makeatletter
%% Output routine for voltage sources
\def\pgf@circ@drawvoltagegenerator{
	\ifpgf@circuit@bipole@voltage@below
		coordinate (pgfcirc@Vfrom) at ($(\ctikzvalof{bipole/name}.center) ! \ctikzvalof{voltage/bump a} ! (\ctikzvalof{bipole/name}.-120)$)
		coordinate (pgfcirc@Vto) at ($(\ctikzvalof{bipole/name}.center) ! \ctikzvalof{voltage/bump a} ! (\ctikzvalof{bipole/name}.-60)$)
	\else
		coordinate (pgfcirc@Vfrom) at ($ (\ctikzvalof{bipole/name}.center) ! \ctikzvalof{voltage/bump a} ! (\ctikzvalof{bipole/name}.120)$)
		coordinate (pgfcirc@Vto) at ($ (\ctikzvalof{bipole/name}.center) ! \ctikzvalof{voltage/bump a} ! (\ctikzvalof{bipole/name}.60)$)
	\fi
	\ifpgf@circuit@europeanvoltage
		\ifpgf@circuit@bipole@voltage@backward
				(pgfcirc@Vfrom)  node {$-$}  (pgfcirc@Vto) node {$+$}
		\else
				(pgfcirc@Vfrom)  node {$+$}  (pgfcirc@Vto) node {$-$}
		\fi
	\else% american voltage
		\ifpgf@circuit@bipole@voltageoutsideofsymbol
		% if it is a battery, must put + and -
			\ifpgf@circuit@bipole@voltage@backward
				(pgfcirc@Vfrom)  node {$-$}  (pgfcirc@Vto) node {$+$}
			\else
				(pgfcirc@Vfrom)  node {$+$}  (pgfcirc@Vto) node {$-$}
			\fi
		\fi
	\fi
}

%% Output routine 
\def\pgf@circ@drawcurrent{
	\pgfextra{
		\edef\pgf@circ@ffffff{\pgf@circ@direction}
		\def\pgfcircmathresult{\expandafter\pgf@circ@stripdecimals\pgf@circ@ffffff\pgf@nil}

		\ifnum\pgfcircmathresult >4 \ifnum\pgfcircmathresult <86
			\ifpgf@circuit@bipole@current@below
				\def\pgf@circ@dir{north west} \else \def\pgf@circ@dir{south east}
			\fi
		\fi\fi
		\ifnum\pgfcircmathresult >85 \ifnum\pgfcircmathresult <95
			\ifpgf@circuit@bipole@current@below
				\def\pgf@circ@dir{west} \else \def\pgf@circ@dir{east} 
			\fi
		\fi\fi
		\ifnum\pgfcircmathresult >94 \ifnum\pgfcircmathresult <176
			\ifpgf@circuit@bipole@current@below
				 \def\pgf@circ@dir{south west}\else \def\pgf@circ@dir{north east}
			\fi
		\fi\fi
		\ifnum\pgfcircmathresult >175 \ifnum\pgfcircmathresult <185
			\ifpgf@circuit@bipole@current@below
				  \def\pgf@circ@dir{south}\else\def\pgf@circ@dir{north}
			\fi
		\fi\fi
		\ifnum\pgfcircmathresult >184 \ifnum\pgfcircmathresult <266
			\ifpgf@circuit@bipole@current@below
				 \def\pgf@circ@dir{south east}\else\def\pgf@circ@dir{north west}
			\fi
		\fi\fi
		\ifnum\pgfcircmathresult >265 \ifnum\pgfcircmathresult <275
			\ifpgf@circuit@bipole@current@below
				 \def\pgf@circ@dir{east}\else \def\pgf@circ@dir{west}
			\fi
		\fi\fi
		\ifnum\pgfcircmathresult >274 \ifnum\pgfcircmathresult <356
			\ifpgf@circuit@bipole@current@below
				  \def\pgf@circ@dir{north east}\else\def\pgf@circ@dir{south west}
			\fi
		\fi\fi
		\ifnum\pgfcircmathresult <5
			\ifpgf@circuit@bipole@current@below
				 \def\pgf@circ@dir{north}\else\def\pgf@circ@dir{south} 
			\fi
		\fi
		\ifnum\pgfcircmathresult >355
			\ifpgf@circuit@bipole@current@below
				 \def\pgf@circ@dir{north}\else\def\pgf@circ@dir{south} 
			\fi
		\fi
		
		\ifpgf@circuit@bipole@current@below
			\def\pgf@circ@bipole@current@label@where{-90}
		\else
			\def\pgf@circ@bipole@current@label@where{+90}
		\fi
	}
	
	\pgfextra{\def\pgf@temp{short}\edef\pgf@circ@temp{\ctikzvalof{bipole/kind}}}
		\ifx\pgf@circ@temp\pgf@temp%draw current at a short at middle of the line
				(\tikztostart)--(\tikztotarget)
		\else% normal bipole or source
			\ifpgf@circuit@bipole@current@before
				 (\tikztostart)--(anchorstartnode)
			\else
				(anchorendnode)--(\tikztotarget)
			\fi
		\fi
		\ifpgf@circuit@bipole@current@backward
			\pgfextra{
				\pgfmathsubtract{\pgf@circ@ffffff}{180}
				\edef\pgf@circ@ffffff{\expandafter\pgf@circ@stripdecimals\pgfmathresult\pgf@nil}
				}
		\fi
	coordinate[currarrow,pos=\ctikzvalof{current/distance},rotate=\pgf@circ@ffffff](Iarrow)
	(Iarrow.\pgf@circ@bipole@current@label@where) node[anchor=\pgf@circ@dir]{\pgf@circ@finallabels{current/label}}
}

%%%\makeatother

\begin{document}

%%%\begin{figure}[!ht]
%%%\centering
%%%\begin{circuitikz}
%%%\draw (0,0) to[V_>=$U_B$] (0,-2);
%%%\draw (0,0) to[/tikz/circuitikz/bipoles/length=30pt, R=$R_i$, -o] (2,0);
%%%\draw (2,0) to[short, i=$I_B$] (3,0);
%%%\draw (3,0) to[/tikz/circuitikz/bipoles/length=30pt, R=$R_u$] (3,-2);
%%%\draw (3,-2) to[short, -o] (2,-2);
%%%\draw (2,-2) to[short] (0,-2);
%%%\end{circuitikz}
%%%\caption{Testcircit}
%%%\end{figure}
%%%
%%%\begin{figure}[!ht]
%%%\centering
%%%\begin{circuitikz}
%%%\draw (0,0) to[V<=$U_B$] (0,2) to[/tikz/circuitikz/bipoles/length=25pt, R=$R_i$, -o] ++(2,0)
%%%to[short, i=$I$] ++ (1,0) to[/tikz/circuitikz/bipoles/length=20pt, Do, l=$D$] ++(0,-2)
%%%to[short, -o] ++ (-1,0) to[short] (0,0);
%%%\draw [<->] (2,0.2) -- (2,1.8);
%%%\node at (1.7,1) {$U_D$};
%%%\end{circuitikz}
%%%\caption{Testcircuit}
%%%\end{figure}

\begin{figure}[!ht]
\begin{circuitikz}
\draw (0,0) to[I, l=$I_B$] (0,2);
\draw [->] (-0.5,0.65) -- (-0.5,1.35);
\end{circuitikz}
\end{figure}

%%%\begin{circuitikz}
%%%\draw (2,0) node[npn,nobase](npn){T1};
%%%\draw (npn.E) node[below]{E};
%%%\draw (npn.C) node[above]{C};
%%%\draw (npn.B) node[circ]{} node[left]{B};
%%%\draw[dashed,red,-latex] (1,0.5)--(npn.nobase);
%%%\end{circuitikz}
%%%
%%%\bigskip
%%%
%%%\begin{circuitikz}
%%%\draw (2,0) node[elmech](motor){};
%%%\draw (motor.north) |-(0,2) to [R] ++(0,-2) to[dcvsource]++(0,-2) -| (motor.bottom);
%%%\draw[thick,->>] (motor.center) -- ++(1.5,0) node[midway,above] {$\omega$};
%%%\end{circuitikz}
%%%
%%%\bigskip
%%%
%%%\begin{circuitikz}
%%%\draw (0,0) node[op amp, yscale=-1] (opamp) {}
%%%(opamp.+) node[left] {$v_+$}
%%%(opamp.-) node[left] {$v_-$}
%%%(opamp.out) node[right] {$v_o$}
%%%(opamp.up) --++(0,-2.5) node[vcc]{5\,\textnormal{V}}
%%%%(opamp.down) --++(0,0.5) node[vee]{-5\,\textnormal{V}}
%%%;
%%%\end{circuitikz}

\end{document}