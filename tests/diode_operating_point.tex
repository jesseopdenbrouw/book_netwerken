\documentclass[a4paper,fleqn]{article}
\usepackage[showframe]{geometry}
\usepackage[latin1]{inputenc}
\usepackage{nimbusmono}
\usepackage{charter}
\usepackage[bitstream-charter]{mathdesign}
\usepackage[stretch=10]{microtype}
\usepackage[dutch]{babel}
\usepackage[font=footnotesize,format=plain,labelfont=bf,up,textfont=sl,up]{caption}
\usepackage[labelformat=simple,font=footnotesize,format=plain,labelfont=bf,textfont=sl]{subcaption}
\captionsetup[figure]{format=hang,justification=centering,singlelinecheck=off,skip=2ex}
\captionsetup[table]{format=hang,justification=centering,singlelinecheck=off,skip=2ex}
\captionsetup[subfigure]{format=hang,justification=centering,singlelinecheck=off,skip=2ex}
\captionsetup[subtable]{format=hang,justification=centering,singlelinecheck=off,skip=2ex}
\renewcommand\thesubfigure{(\alph{subfigure})}
\renewcommand\thesubtable{(\alph{subtable})}
\usepackage{sansmath}
\usepackage{mathtools}
\setlength{\mathindent}{1em}
\DeclareMathSymbol{,}{\mathord}{letters}{"3B}
\usepackage{parskip}


%%%%%%%%%%%%%%%%%%%%%%%%%%%%%%%%%%%%%%%%%%%%%%%%%%%%%%%%%%%%%%%%%%%%%
%%
%% TIKZ AND FRIENDS
%%
%%%%%%%%%%%%%%%%%%%%%%%%%%%%%%%%%%%%%%%%%%%%%%%%%%%%%%%%%%%%%%%%%%%%%
\usepackage{tikz}
\usepackage[european,betterproportions,americaninductors]{circuitikz}
%\usepackage[]{circuitikz}
\usepackage{pgfplots}
\usetikzlibrary{intersections}
\usepgfplotslibrary{units}

% New dashed-dotted style
\tikzstyle{loosely dashdotted} = [dash pattern=on 3pt off 4pt on \the\pgflinewidth off 4pt]
% Directory prefix for Gnuplot plots
\tikzset{prefix=gnuplot/}
% Bipoles have the same line witdh as the wires
\ctikzset{bipoles/thickness=1}
% Reinitialize the generic box for European resistor
%%%\ctikzset{bipoles/generic/height/.initial=.20}
%%%\ctikzset{bipoles/generic/width/.initial=.60}


% Get the x and y coordinates of a point
\makeatletter
\newcommand{\gettikzxy}[3]{%
  \tikz@scan@one@point\pgfutil@firstofone#1\relax
  \global\edef#2{\the\pgf@x}%
  \global\edef#3{\the\pgf@y}%
}
\makeatother

% Transform x coordinate to graph coordinate
\makeatletter
\newcommand\transformxdimension[1]{
    \pgfmathparse{((#1/\pgfplots@x@veclength)+\pgfplots@data@scale@trafo@SHIFT@x)/10^\pgfplots@data@scale@trafo@EXPONENT@x}
}
% Transform y coordinate to graph coordinate
\newcommand\transformydimension[1]{
    \pgfmathparse{((#1/\pgfplots@y@veclength)+\pgfplots@data@scale@trafo@SHIFT@y)/10^\pgfplots@data@scale@trafo@EXPONENT@y}
}
\makeatother

%%%\pgfplotsset{xticklabel={$\mathsf{\pgfmathprintnumber{\tick}}$},
%%%             yticklabel={$\mathsf{\pgfmathprintnumber{\tick}}$},
%%%             every axis/.append style={font=\sffamily}
%%%}

%% Create a Dutch version of a voltage source
\makeatletter
%% Output routine for voltage sources
\def\pgf@circ@drawvoltagegenerator{
	\ifpgf@circuit@bipole@voltage@below
		coordinate (pgfcirc@Vfrom) at ($(\ctikzvalof{bipole/name}.center) ! \ctikzvalof{voltage/bump a} ! (\ctikzvalof{bipole/name}.-120)$)
		coordinate (pgfcirc@Vto) at ($(\ctikzvalof{bipole/name}.center) ! \ctikzvalof{voltage/bump a} ! (\ctikzvalof{bipole/name}.-60)$)
	\else
		coordinate (pgfcirc@Vfrom) at ($ (\ctikzvalof{bipole/name}.center) ! \ctikzvalof{voltage/bump a} ! (\ctikzvalof{bipole/name}.120)$)
		coordinate (pgfcirc@Vto) at ($ (\ctikzvalof{bipole/name}.center) ! \ctikzvalof{voltage/bump a} ! (\ctikzvalof{bipole/name}.60)$)
	\fi
	\ifpgf@circuit@europeanvoltage
		\ifpgf@circuit@bipole@voltage@backward
				(pgfcirc@Vfrom)  node {$-$}  (pgfcirc@Vto) node {$+$}
		\else
				(pgfcirc@Vfrom)  node {$+$}  (pgfcirc@Vto) node {$-$}
		\fi
	\else% american voltage
		\ifpgf@circuit@bipole@voltageoutsideofsymbol
		% if it is a battery, must put + and -
			\ifpgf@circuit@bipole@voltage@backward
				(pgfcirc@Vfrom)  node {$-$}  (pgfcirc@Vto) node {$+$}
			\else
				(pgfcirc@Vfrom)  node {$+$}  (pgfcirc@Vto) node {$-$}
			\fi
		\fi
	\fi
}

%% Create a Dutch version of the current source
\makeatletter
\pgfcircdeclarebipole{}{\ctikzvalof{bipoles/isource/height}}{isource}{\ctikzvalof{bipoles/isource/height}}{\ctikzvalof{bipoles/isource/width}}{
    \pgfpointorigin
    \pgfsetlinewidth{\pgfkeysvalueof{/tikz/circuitikz/bipoles/thickness}\pgfstartlinewidth}
    \pgfpathellipse{\pgfpointorigin}{\pgfpoint{0}{\pgf@circ@res@up}}{\pgfpoint{\pgf@circ@res@left}{0}}
    \pgfpathmoveto{\pgfpoint{\pgf@circ@res@step}{\pgf@circ@res@up}}
    \pgfpathlineto{\pgfpoint{\pgf@circ@res@step}{\pgf@circ@res@down}}
  \pgfusepath{draw}
  \pgfscope
    \pgfsetarrowsend{latex'}
    \pgfpathmoveto{\pgfpoint{0.8\pgf@circ@res@left}{1.3\pgf@circ@res@up}}
    \pgfpathlineto{\pgfpoint{0.8\pgf@circ@res@right}{1.3\pgf@circ@res@up}}
    \pgfusepath{draw}   %draw arrow
  \endpgfscope
}
\makeatother

\usepackage{sansmath}

%%%%%%%%%%%%%%%%%%%%%%%%%%%%%%%%%%%%%%%%%%%%%%%%%%%%%%%%%%%%%%%%%%%%%


\begin{document}
\pagestyle{empty}

\section{Belastingskarakteristiek diode en weerstand}

De schakeling:
\begin{figure}[!ht]
\centering
\begin{circuitikz}[line width=1pt]
\draw (0,0) to[V_>=$U_B$] (0,-2);
\draw (0,0) to[R=$R$, -*] (2,0);
\draw (2,0) to[short, i=$I_D$] (4,0);
\draw (4,0) to[empty diode, l=$D$] (4,-2);
\draw (4,-2) to[short, -*] (2,-2);
\draw (2,-2) to[short] (0,-2);
\draw[<->,shorten <=5pt,shorten >=5pt,thin] (2,0) -- (2,-1) node[anchor=east] {$U_D$} -- (2,-2);
\end{circuitikz}
\caption{Serieschakeling van een weerstand en een diode.}
\end{figure}


De spanning-stroom-vergelijking van een diode is:
%
\begin{equation}
I_D = I_S\cdot\left(\text{e}^{\left(\frac{U_D}{n\cdot U_T}\right)}-1\right)
\end{equation}
%
Met:
\begin{itemize}\itemsep-4pt
\item[] $I_D$: de diodestroom;
\item[] $I_S$: de donkerstroom van de diode, $10^{-12}$ A;
\item[] $U_D$: de diodespanning in doorlaat;
\item[] $n$ : de emissieco\"efficient, $1$;
\item[] $U_T$: thermische spanning, $25,69257$ mV bij 25 $^{\circ}$C;
\end{itemize}

%%%%\begin{figure}[!ht]
%%%%\centering
%%%%\begin{tikzpicture}[domain=0:0.69,xscale=6,yscale=0.066666667]
%%%%    %\draw[very thin,color=gray] (-0.1,-1.1) grid (1.1,3.9);
%%%%    \draw[->] (-0.05,0) -- (1.05,0) node[right] {\footnotesize $U_D$ [V]};
%%%%    \draw[->] (0,-5) -- (0,45) node[above] {\footnotesize $I_D$ [mA]};
%%%%    \foreach \i/\ii in {0.1/{0,1},0.2/{0,2},0.3/{0,3},0.4/{0,4},0.5/{0,5},0.6/{0,6},0.7/{0,7},0.8/{0,8},0.9/{0,9},1.0/{1,0}} {
%%%%	    \draw     (\i,0) node[below] {\footnotesize\ii};
%%%%    	\draw[-]  (\i,1) -- (\i,-1);
%%%%    }
%%%%    \foreach \i in {10,20,30,40} {
%%%%	    \draw     (0,\i) node[left] {\footnotesize\i};
%%%%    	\draw[-]  (-0.01,\i) -- (0.01,\i);
%%%%    }
%%%%    
%%%%    % Draw the V-I curve of the diode. The dark current in in milliamps.
%%%%    \draw[color=red] plot[id=diodevi,samples=100,prefix=gnuplot/]
%%%%                     function{0.0000000001*(exp(x/0.02569257028945)-1)} ;
%%%%       % node[right] {$f(x) = \frac{1}{20} \mathrm e^x$};
%%%%    \draw[color=blue, -] (0,40) -- (1.0,0);
%%%%    \draw[dotted, -] (0.6587471914,0) -- (0.6587471914,13.65011234);
%%%%    \draw[dotted, -] (0,13.65011234)     -- (0.6587471914,13.65011234);
%%%%\end{tikzpicture}
%%%%\caption{Belastingskarakteristiek van een serieschakeling van een diode en een weerstand.}
%%%%\end{figure}

\begin{figure}[!ht]
\centering
\begin{tikzpicture}
	\sansmath
	\begin{axis}[width=8cm,height=5cm,
		xlabel=$U_D$,
		x unit=V,
		ylabel=$I_D$,
		y unit=mA,
		xtick={0.1,0.2,...,1.05},
		axis x line*=bottom,
		axis y line*=left,
        xmin=0, xmax=1.05,
        ymin=0, ymax=42,
        restrict y to domain=0:40,
        % enlargelimits={abs=10pt},
        x tick label style={
        /pgf/number format/.cd,
            fixed,
            fixed zerofill,
            precision=1,
	        use comma,
    	    1000 sep={},
            /tikz/.cd,
        }
	]
	% Draw diode curve, diode current is in mA
	\addplot [domain=0:1,samples=201,red,thick,name path=dline] {0.0000000001*(exp(x/(1.0*0.02569257028945))-1)};
	% Draw load line
	\addplot [domain=0:1,samples=201,blue,thick,name path=rline] {(1.0-x)/0.025};
	% Find the intersection of the diode curve and the load line
	\path [name intersections={of=rline and dline , by=op}];
	% Draw horizontal an vertical li lines
	\draw [dashed] ({{0,0}} |- op) -- (op);
	\draw [dashed] ({{0,0}} -| op) -- (op);
	
	% Calculate the intersection coordinates and print them
	\path(op) node [above right,xshift=5pt] {\footnotesize($\!\!\!$\pgfgetlastxy{\macrox}{\macroy}
        \transformxdimension{\macrox}%
        \pgfmathprintnumberto[use comma]{\pgfmathresult}{\dummy}\global\edef\udiode{\dummy}%
        \pgfmathprintnumber[use comma]{\pgfmathresult},%
        \transformydimension{\macroy}%
        \pgfmathprintnumberto[use comma]{\pgfmathresult}{\dummy}\global\edef\idiode{\dummy}%
        \pgfmathprintnumber[use comma]{\pgfmathresult})};

%%%    \path (op) node [below left] {
%%%                    \pgfplotspointgetcoordinates{(intersection-1)}
%%%                    $(
%%%                        \pgfmathprintnumber[fixed]{\pgfkeysvalueof{/data point/x}},
%%%                        \pgfmathprintnumber[fixed]{\pgfkeysvalueof{/data point/y}}
%%%                    )$
%%%                };
	
	
	
	\end{axis}

\end{tikzpicture}
\caption{Belastingskarakteristiek van een serieschakeling van een diode en uitwendige weerstand en de bron.}
\end{figure}

%Via Maple is gevonden: $U_D = 0,6587471914$ V en $I_D = 13,65011234$ mA.

Voor het instelpunt geldt dat de spanning over de diode \udiode\ V is. De stroom door de
diode (en dus ook de weerstand en de bron) is \idiode\ mA.




\end{document}