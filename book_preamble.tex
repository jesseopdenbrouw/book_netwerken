%%%%%%%%%%%%%%%%%%%%%%%%%%%%%%%%%%%%%%%%%%%%%%%%%%%%%%%%%%%%%%%%%%
%%%
%%%       PREAMBLE
%%%
%%%%%%%%%%%%%%%%%%%%%%%%%%%%%%%%%%%%%%%%%%%%%%%%%%%%%%%%%%%%%%%%%%

%% 12pt charachters, A4 paper size, twoside printing, equation left aligned
%% equation indent at 1 em
\documentclass[12pt,a4paper,final,twoside,fleqn]{book}

% Set widow penalty
\clubpenalty2000

%% Use T1 output font encoding
\usepackage[T1]{fontenc}

%% Copy credentials
\renewcommand{\author}{\bookauthor}
\renewcommand{\title}{\booktitle}
\renewcommand{\date}{\today}

%% Some defines for scaling figs
%% Scaled 1/1
\def\figscale{0.6} % Really it should be 1-exp{-1} = 0.63212055882855767840447622983854
\def\figscaleA{0.5}
\def\figscaleAA{0.43}
\def\figscaleB{0.3}
\def\figscaleC{1.0}
\def\figscaleE{0.7}
\def\figscaleF{0.6}
\def\figscaleG{0.8}

%% Define which part of the book is included
\newif\ifusebookpartI\bookpartI
\newif\ifusebookpartII\bookpartII
\newif\ifusebookpartIII\bookpartIII
\newif\ifusebookasbook\bookasbook

%% Dutch spelling of chapter, section, etc.
\usepackage[dutch]{babel}
%% Set American style quotes
\usepackage[style=american]{csquotes}

%% Set page layout
\usepackage[a4paper,bindingoffset=0.2in,left=1.0in,right=1.0in,top=1.0in,bottom=1.4in,
                                                               footskip=0.5in]{geometry}
                                                               
%% Use of dashed lines in tables
\usepackage{tabu}
\usepackage{tabularx}
\usepackage{booktabs}
\usepackage{multirow}

%% Use array's
\usepackage{array}

%% Define and use colors
\usepackage[x11names,table]{xcolor}

%% Include rotating (includes graphicx) packages
\usepackage{rotating}

%% Nice calculations with lengths
%% Note: calc has to be loaded before enumitem
\usepackage{calc}

%% Enumerate items
\usepackage[inline]{enumitem}
%% Use an asterisk before item number, see
%% http://tex.stackexchange.com/questions/61263/add-an-asterisk-left-of-an-enumerate
\setlist[enumerate]{before=\setupmodenumerate}
%
\newif\ifmoditem
\newcommand{\setupmodenumerate}{%
  \global\moditemfalse
  \let\origmakelabel\makelabel
  \def\moditem##1{\global\moditemtrue\def\mesymbol{##1}\item}%
  \def\makelabel##1{%
%    \origmakelabel{\ifmoditem\llap{\mesymbol\enspace}\fi##1}%
    \origmakelabel{\ifmoditem\llap{\mesymbol\ }\fi##1}%
    \global\moditemfalse}%
}
\newcommand{\itemstar}{\moditem{\textbf{*}}}

%% Use floats
\usepackage{float}
%% Separation between floats 12pt --> 24 pt
\setlength{\floatsep}{24.0pt plus 2.0pt minus 2.0pt}

%% Using footnotes in tables
%%%\usepackage{tablefootnote}
\usepackage{threeparttable}
%\renewcommand{\TPTminimum}{1em}
\renewcommand{\TPTnoteSettings}{\footnotesize}
%\makeatletter
%\setlist[tablenotes]{label=\tnote{\alph*},ref=\alph*,itemsep=\z@,topsep=\z@skip,partopsep=\z@skip,parsep=\z@,itemindent=\z@,labelindent=\tabcolsep,labelsep=.2em,leftmargin=*,align=left,before={\footnotesize}}
%\makeatother

% http://archive.cs.uu.nl/mirror/CTAN/macros/latex/contrib/biblatex/doc/biblatex.pdf
\PassOptionsToPackage{hyphens}{url}
\usepackage[
    backend=biber,
    backref=true,
    backrefstyle=none,
    sortcites=true,
    sorting=none,
    doi=false, % doi informatie wordt niet weergegeven
    %uniquename=true,
    %uniquelist=true,
    maxcitenames=3,
    %issn=false, werkt niet
    language=american
]{biblatex}
\addbibresource{book_citations.bib}
\DefineBibliographyStrings{dutch}{
    backrefpage = {blz.},
    backrefpages = {blz.},
}
%% Do not show ISSN numbers
\AtEveryBibitem{\clearfield{issn}}
\AtEveryCitekey{\clearfield{issn}}

%% Use the AMS Mathematical characters, no other AMS packages required
\usepackage{mathtools}
\usepackage{siunitx}
\sisetup{output-decimal-marker = {,}}
\sisetup{tophrase={{\ en\ }}}
%\usepackage{amssymb}
\setlength{\mathindent}{1em}
\DeclareMathSymbol{,}{\mathord}{letters}{"3B}
\abovedisplayskip=30pt
\belowdisplayskip=30pt
\abovedisplayshortskip=30pt
\belowdisplayskip=30pt

%% Find out what engine we're running...
\usepackage{ifluatex,ifxetex}

%% Settings for fonts et al.
\ifnum 0\ifxetex 1\fi\ifluatex 1\fi>0
\usepackage[math-style=TeX]{unicode-math}
%\usepackage{unicode-math}
\usepackage{fontspec}
\setmainfont[Ligatures=TeX]{Calibri}
\defaultfontfeatures{Scale=MatchUppercase}
\setsansfont{Calibri}
\setmonofont{Consolas}
\setmathfont[slash-delimiter=frac]{Cambria Math}
\setmathfont[range=up]{Calibri}
\setmathfont[range=it]{Calibri Italic}
\setmathfont[range=bfup]{Calibri Bold}
\setmathfont[range=bfit]{Calibri Bold Italic}
\setoperatorfont\normalfont % For log, sin, cos, etc.
\def\checkmark{\textsl{v}}
\def\muup{μ}
\makeatletter
\DeclareRobustCommand{\LaTeX}{L\kern-.20em%
        {\sbox\z@ T%
         \vbox to\ht\z@{\hbox{\check@mathfonts
                              \fontsize\sf@size\z@
                              \math@fontsfalse\selectfont
                              A}%
                        \vss}%
        }%
        \kern-.06em%
        \TeX}
\makeatother
\else
%% Use the Charter and Nimbus Mono fonts
\usepackage{nimbusmono}
%\usepackage{charter}
\usepackage[bitstream-charter]{mathdesign}
%% Use microtype
\usepackage[stretch=10]{microtype}
\pdfminorversion=5
\pdfobjcompresslevel=1
%% Set input encoding to UTF-8
\usepackage[utf8]{inputenc}
\makeatletter
\DeclareRobustCommand{\LaTeX}{L\kern-.20em%
        {\sbox\z@ T%
         \vbox to\ht\z@{\hbox{\check@mathfonts
                              \fontsize\sf@size\z@
                              \math@fontsfalse\selectfont
                              A}%
                        \vss}%
        }%
        \kern-.05em%
        \TeX}
\makeatother
\fi

%%
%% Find out which engine is running
\newcount\bookmajorversion
\newcount\bookminorversion

\ifluatex
\bookmajorversion=\luatexversion
\bookminorversion=\luatexversion
\divide \bookmajorversion by 100
\multiply \bookmajorversion by 100
\advance \bookminorversion by -\bookmajorversion
\divide \bookmajorversion by 100
\def\booktexbanner{Lua\LaTeX\ \the\bookmajorversion.\the\bookminorversion.\luatexrevision}
\else
\ifxetex
\bookmajorversion=\XeTeXversion
\def\booktexbanner{Xe\LaTeX\ \the\bookmajorversion\XeTeXrevision}
\else
\bookmajorversion=\pdftexversion
\bookminorversion=\pdftexversion
\divide \bookmajorversion by 100
\multiply \bookmajorversion by 100
\advance \bookminorversion by -\bookmajorversion
\divide \bookmajorversion by 100
\def\booktexbanner{pdf\LaTeX\ \the\bookmajorversion.\the\bookminorversion.\pdftexrevision}
\fi
\fi

%%
%% Do we include advanced sections?
\newif\ifuseadvanced\bookuseadvanded

%%
%% Creating advanced sections with * prepended to sec number
\newcommand*{\secmark}{}
\newcommand*{\marktotoc}[1]{\renewcommand{\secmark}{#1}}
\newcommand*{\advanced}{%
  \renewcommand*{\secmark}{*}%
  \addtocontents{toc}{\protect\marktotoc{*}}%
}
\newcommand*{\basic}{%
  \renewcommand*{\secmark}{}%
  \addtocontents{toc}{\protect\marktotoc{}}%
}

% http://archive.cs.uu.nl/mirror/CTAN/macros/latex/contrib/titlesec/titlesec.pdf
\usepackage{titlesec}
\usepackage{titletoc}
\usepackage[titles]{tocloft}
\newcommand{\chapnumfont}{%     % define font for chapter number
  \usefont{T1}{pnc}{b}{n}%      % choose New Chancery, bold, normal shape
  \fontsize{100}{100}%          % font size 100pt, baselineskip 100pt
  \selectfont%                  % activate font
  \vspace*{-0.5\baselineskip}	% half baseline skip up
}
\titleformat{\chapter}[display]
{\bfseries}
{\chapnumfont\color{chapnumcolor}{\thechapter}}
{0ex}
{\ifnum 0\ifxetex 1\fi\ifluatex 1\fi>0\else\fontfamily{phv}\selectfont\fi\Huge\bfseries}
[{\vspace*{1ex}\titlerule[0.8pt]}]
\titleformat{\section}{\ifnum 0\ifxetex 1\fi\ifluatex 1\fi>0\else\fontfamily{phv}\selectfont\fi\large\bfseries}{\secmark\thesection}{1em}{}
\titleformat{\subsection}{\ifnum 0\ifxetex 1\fi\ifluatex 1\fi>0\else\fontfamily{phv}\selectfont\fi\bfseries}{\secmark\thesubsection}{1em}{}
\titlespacing*{\section}{0pt}{\baselineskip}{0.4\aftersubsection}
\titlespacing*{\subsection}{0pt}{.8\baselineskip}{0.0\aftersubsection}
\titlespacing*{\subsubsection}{0pt}{.6\baselineskip}{0pt}
\titlespacing*{\paragraph}{0pt}{1.0ex plus 1ex minus .2ex}{1.5em}
\newlength{\aftersubtitle}
\setlength{\aftersubtitle}{1.2\baselineskip}
\newlength{\aftersubsection}
\setlength{\aftersubsection}{\aftersubtitle}
\addtolength{\aftersubsection}{-\baselineskip}
\titlespacing*{\subsection}{0pt}{.8\baselineskip}{\aftersubsection}
\titlespacing*{\subsubsection}{0pt}{.6\baselineskip}{0pt}
%% Remove page number on first page of ToC
\AtBeginDocument{\addtocontents{toc}{\protect\thispagestyle{fancy}}}
%% Make space for page number in TOC wider
\makeatletter
\renewcommand\@pnumwidth{1cm}
\makeatother
%% Wider chapnum and secnum
%\renewcommand*\cftchapnumwidth{2em}
\renewcommand*\cftsecnumwidth{3em}

% http://archive.cs.uu.nl/mirror/CTAN/macros/latex/contrib/footmisc/footmisc.pdf
% ruimte onder aan de pagina tussen tekst en voetnoot niet na voetnoot
% package moet VOOR fancyhdr om warning te voorkomen
\usepackage[
    bottom,
    hang,
    multiple
]{footmisc}
% inspringen
\setlength{\footnotemargin}{1em}
% ruimte tussen footnotes:
\setlength{\footnotesep}{.6\baselineskip}


%% Using fancy headers and footers
%% http://ftp.snt.utwente.nl/pub/software/tex/macros/latex/contrib/fancyhdr/fancyhdr.pdf
\usepackage{fancyhdr}
\newcommand{\bookfancy}{
    \fancyhead{} % clear all header fields
    \fancyhead[LE,RO]{\footnotesize\finalconceptdraft}
    %\fancyhead[RO,LE]{\thepage} % Right Odd, Left Even
    \fancyfoot{} % clear all footer fields
    \fancyfoot[LE,RO]{\thepage}   % page number in "outer" position of footer line
    \fancyfoot[RE,LO]{\booktitle} % other info in "inner" position of footer line
    \renewcommand{\headrulewidth}{0pt}
    \renewcommand{\footrulewidth}{0.4pt}
}
\pagestyle{fancy}
\bookfancy
\fancypagestyle{plain}{\bookfancy}
%% Clear header and footer for blank pages, possibly print "This page ..."
\makeatletter
\renewcommand*{\cleardoublepage}{\clearpage\if@twoside \ifodd\c@page\else
\hbox{}\vfill\begin{center}\thispageintentionallyleftblank\end{center}\vfill\vfill%
\thispagestyle{empty}%
\newpage%
\if@twocolumn\hbox{}\newpage\fi\fi\fi}
\makeatother

%% Indexing words...
\usepackage[noautomatic]{imakeidx}
%% The index page is typeset...
\indexsetup{othercode=\footnotesize\thispagestyle{fancy}}
\makeindex[columns=2]
%\usepackage{showidx}

%% Making captions nicer...
\usepackage[font=footnotesize,format=plain,labelfont=bf,up,textfont=sl,up]{caption}
\DeclareCaptionLabelSeparator{emdash}{\ \ ---\ \ }
\usepackage[labelformat=simple,font=footnotesize,format=plain,labelfont=bf,textfont=sl]{subcaption}
\captionsetup[figure]{format=hang,justification=centering,singlelinecheck=off,skip=2ex}
\captionsetup[table]{format=hang,justification=centering,singlelinecheck=off,skip=2ex}
\captionsetup[subfigure]{format=hang,justification=centering,singlelinecheck=off,skip=2ex}
\captionsetup[subtable]{format=hang,justification=centering,singlelinecheck=off,skip=2ex}
%% Put parens around the subfig name (a) (b) etc. Needs labelformat simple
\renewcommand\thesubfigure{(\alph{subfigure})}
\renewcommand\thesubtable{(\alph{subtable})}

% Parskip et al.
\usepackage{parskip}[=v1]
\makeatletter
\setlength{\parfillskip}{00\p@ \@plus 1fil}
\makeatother

%% Nice strike out of text (more used in numbers)
\usepackage{cancel}

%% Some colors
\definecolor{dkgreen}{rgb}{0,0.6,0}
%\definecolor{gray}{rgb}{0.5,0.5,0.5}
\definecolor{mauve}{rgb}{0.58,0,0.82}
%\definecolor{lightgray}{rgb}{0.95,0.95,0.95}
\definecolor{GKYgray}{rgb}{0.65,0.65,0.65}
\definecolor{lightpink}{HTML}{F8E0E0}%{FBF3F3}%
\definecolor{thuasgreen}{RGB}{158,167,0}
\definecolor{thuasgreen2}{RGB}{159,169,89}
\ifusebookasbook
\colorlet{bookcolor}{black}
\colorlet{chapnumcolor}{gray!90}
\colorlet{listingnumbercolor}{bookcolor}
\colorlet{listingbgcolor}{gray!10}
\colorlet{infoboxbg}{gray!5}
\colorlet{infoboxbgtitle}{gray!15}
\colorlet{venndiagramcolor}{lightgray}
%\colorlet{regcell}{gray!90}
\else
\colorlet{bookcolor}{red!55!black} % Global book color
\colorlet{chapnumcolor}{bookcolor}
\colorlet{listingnumbercolor}{bookcolor}
\colorlet{listingbgcolor}{bookcolor!10}
\colorlet{linecolor1}{bookcolor}
\colorlet{linecolor2}{red!70!black}
%\colorlet{listingcolor}{red!30!white}
\colorlet{infoboxbg}{brown!10}
\colorlet{infoboxbgtitle}{brown!20}
\colorlet{venndiagramcolor}{blue!15}
\colorlet{regcell}{thuasgreen!50!white}
\fi

%% Use computer code listings
\usepackage{listings}
%% Use textcomp for upright quotes in listings
\usepackage{textcomp}

%% Define listing style for VHDL
\lstset{%
  language=VHDL,
  basicstyle=\footnotesize\ttfamily,
  numbers=left,
  numberstyle=\tiny\color{listingnumbercolor},
  stepnumber=1,                           
  numbersep=8pt,
  backgroundcolor=\color{listingbgcolor},
  showspaces=false,
  showstringspaces=false,
  showtabs=false,
  frame=lines,
  rulecolor=\color{bookcolor},
  tabsize=4,
  captionpos=b,
  breaklines=true,
  breakatwhitespace=false,
  title=\lstname,
  upquote=true,
  aboveskip=\baselineskip
}

%% Using hyperrefs...
\usepackage{hyperref}
\ifusebookasbook
\hypersetup{
	colorlinks=true,
	linkcolor=black,
	citecolor=black,
	urlcolor=black,
    pdftitle={\booktitleI\ \booktitleII, \booksubtitle},
    pdfauthor={\bookauthor},
    %pdfsubject={Beginselen van de digitale techniek},
    pdfsubject={},
    pdfkeywords={\bookkeywords},
	pdfdisplaydoctitle=true,
	pdfpagelayout=TwoPageRight
}
\else
\hypersetup{
	colorlinks=true,
	linkcolor=bookcolor,
	citecolor=bookcolor,
	urlcolor=bookcolor,
    pdftitle={\booktitleI\ \booktitleII, \booksubtitle},
    pdfauthor={\bookauthor},
    %pdfsubject={Beginselen van de digitale techniek},
    pdfsubject={},
    pdfkeywords={\bookkeywords},
	pdfdisplaydoctitle=true,
	pdfpagelayout=TwoPageRight
}
\fi
\urlstyle{sf}%

%% Lay things over a picture
%% http://mirrors.ctan.org/macros/latex/contrib/overpic/opic-abs.pdf
\usepackage[abs]{overpic}

%% Set \FloatBarrier to empty
\def\FloatBarrier{}

%% Create EAN13 ISBN number figure
\usepackage{ean13isbn}

%%%
%%% LaTeX installed packages loading done here
%%%
%%% From here on, only own-written packages and commands
%%%

%% Better overline command for use in logic functions
\newcommand*{\oline}[1]{\overline{#1\mathstrut}}

%% A nice circled x for use in prime implicant tables
\newcommand{\cx}{\raisebox{-1.2pt}{\textcircled{\raisebox{1.2pt}{x}}}}

%% Prints nice "solution" header (dutch: uitwerking)
\newcommand{\uitw}[1]{\addvspace{1em}\textbf{Uitwerking opgave \ref{#1}.}\vspace*{0.2em}\newline}

%% Circled numbers
%\def\circled#1{{\ooalign{\hfil\lower.1ex\hbox{#1}\hfil\crcr\Orb}}}
\def\circled#1{\textcircled{\raisebox{.95pt}{\scriptsize #1}}}

%% Typeset propositions and postulates
\newcommand{\propositie}[1]{\par\hspace*{1em}#1\par}

%% Log notation with base
\newcommand\logbase[1]{\mathop{{}^{#1}\mathrm{log}}}

%% To input and print files.
%% \booklising{lang}{caption}{filename}{ext}{floatplace}
%%%\newcommand{\booklisting}[5]{%
%%%\begin{figure}[#5]%
%%%\lstinputlisting[language=#1,caption={#2.},label=cod:#3]{code/#3.#4}%
%%%\vspace*{-\baselineskip}
%%%\end{figure}%
%%%}
\newcommand{\booklisting}[6][]{%
\begin{figure}[#6]%
\lstinputlisting[language=#2,caption={#3.},label=cod:#4,#1]{code/#4.#5}%
\vspace*{-\baselineskip}
\end{figure}%
}
%\lstinputlisting[language=#1,caption={#2.},label=cod:#3,float]{code/#3.vhd}%
%\lstinputlisting[language=#1,caption={#2.},label=cod:#3,float,floatplacement=h]{code/#3.vhd}%
%\begin{minipage}{\textwidth}%
%\lstinputlisting[language=#1,caption={#2.},label=cod:#3]{code/#3.vhd}%
%\end{minipage}%
%%% werkt niet
%%%\captionsetup{belowskip=100pt}
%%%\setlength{\abovecaptionskip}{-100pt plus 0pt minus 0pt}

\newcommand{\lstmodelsim}[1]{\lstinline[language=modelsim]|#1|}   %[
\newcommand{\lstvhdl}[1]{\lstinline[language=vhdl]|#1|}   %[
\newcommand{\lstc}[1]{\lstinline[language=C]|#1|}   %[
\newcommand{\lstpseu}[1]{\lstinline[language=HHSEprocPseudoLang]|#1|}   %[
\newcommand{\lstasm}[1]{\lstinline[language=HHSEprocAssembler]|#1|}   %[

%% To put single figures
%% \bookfigure{floatplace}{scale}{imagename}{caption}{label}
\newcommand{\bookfigure}[5]{%
\begin{figure}[#1]%
\centering%
\includegraphics[scale=#2]{images/#3}%
\caption{#4.}%
\label{fig:#5}%
\end{figure}%
}

%% Print hidden characters by using the white color, used for aligning numbers.
%% Please note that the dash must cast to mathbin. See:
%% http://tex.stackexchange.com/questions/21598/how-to-color-math-symbols
\newcommand\ph[1]{\if-#1\mathbin{\textcolor{white}{-}}\else\textcolor{white}{#1}\fi}

%% Lesser and more space around the binay dot and plus and xor
\let\oldcdot\cdot
\renewcommand{\cdot}{\kern-.10em\oldcdot\kern-.10em}
\newcommand{\cdotw}{\kern.08em\oldcdot\kern.08em}
\newcommand{\cdotww}{\kern.16em\oldcdot\kern.16em}
\newcommand{\plusw}{\kern.08em+\kern.08em}
\newcommand{\plusww}{\kern.16em+\kern.16em}
\newcommand{\oplusw}{\kern.08em\oplus\kern.08em}
\newcommand{\oplusww}{\kern.16em\oplus\kern.16em}

%% Vertically align cell contents
\newcolumntype{M}[1]{>{\centering\arraybackslash$\vcenter\bgroup\hbox\bgroup}m{#1}<{\egroup\egroup$}}
\newcolumntype{C}{>{\centering\arraybackslash$\vcenter\bgroup\hbox\bgroup}c<{\egroup\egroup$}}

%% Make cells in tables vertically wider
\newcommand\Tstrut{\rule{0pt}{2.0ex}}         % = `top' strut
\newcommand\Bstrut{\rule[-0.9ex]{0pt}{0pt}}   % = `bottom' strut

%% Check if a label exists
\makeatletter
\newcommand{\iflabelexists}[3]{\@ifundefined{r@#1}{#3}{#2}}
\makeatother

%% Using tcolorbox for Infobox environment settings
\usepackage[most]{tcolorbox}
\newtcolorbox{infobox}[1][]{title=#1,colback=infoboxbg,colframe=bookcolor,fonttitle=\large\scshape,colbacktitle=infoboxbgtitle,coltitle=black,toptitle=5pt,bottomtitle=5pt,sharp corners,boxrule=2pt,titlerule=0pt,top=7pt,parbox=false,floatplacement=!b,float=!b}

\newcounter{example}[chapter]
\setcounter{example}{0}
\newtcolorbox[use counter=example,number within=chapter,list inside=example]{example}[1][]{%
    % Example Frame Start
    empty,% Empty previously set parameters
    title={Voorbeeld \theexample\ifx\hfuzz#1\hfuzz \else : #1\fi},% use \thetcbcounter to access the example counter text
%	list entry=Voorbeeld~\theexample.\quad#1,
    % Attaching a box requires an overlay
    attach boxed title to top left,
    % Ensures proper line breaking in longer titles
    minipage boxed title,
    % (boxed title style requires an overlay)
    boxed title style={empty,size=minimal,toprule=0pt,top=4pt,bottom=2pt,left=3mm,overlay={}},
    coltitle=bookcolor,fonttitle=\bfseries,
    before=\par\medskip\noindent,parbox=false,boxsep=0pt,left=3mm,right=0mm,top=5pt,breakable,pad at break=0mm,
       before upper=\csname @totalleftmargin\endcsname0pt, % Use instead of parbox=true. This ensures parskip is inherited by box.
    % Handles box when it exists on one page only
    overlay unbroken={\draw[bookcolor,line width=.5pt] ([xshift=-0pt]title.north west) -- ([xshift=-0pt]frame.south west); },
    % Handles multipage box: first page
    overlay first={\draw[bookcolor,line width=.5pt] ([xshift=-0pt]title.north west) -- ([xshift=-0pt]frame.south west); },
    % Handles multipage box: middle page
    overlay middle={\draw[bookcolor,line width=.5pt] ([xshift=-0pt]frame.north west) -- ([xshift=-0pt]frame.south west); },
    % Handles multipage box: last page
    overlay last={\draw[bookcolor,line width=.5pt] ([xshift=-0pt]frame.north west) -- ([xshift=-0pt]frame.south west); },%
}
\makeatletter % no indent for entries
\renewcommand{\l@tcolorbox}{\@dottedtocline{1}{0pt}{2.3em}}
\makeatother

%% Hypenation of some uncommon words in Dutch
\hyphenation{Kar-naugh-dia-gram Kar-naugh-dia-gram-men min-term min-ter-men flip-flop flip-flops im-pli-cant im-pli-can-ten twee-tal-len vier-tal-len acht-tal-len zes-tien-tal-len timing-dia-gram-men timing-dia-gram toe-stands-dia-gram toe-stands-dia-grammen ge-schre-ven}

%% The really backmatter
\newcommand{\bookbackmatter}{%
\ifusebookasbook
\cleardoublepage%
\hspace*{0pt}
\thispagestyle{empty}
\clearpage
\pagecolor{Sienna2}
\hspace*{0pt}
\vfill
%\hfill\EANisbn[SC5b,ISBN=978-80-7340-097-2] 
\thispagestyle{empty}
\fi
}

%%% No package loading from here
