%%%%%%%%%%%%%%%%%%%%%%%%%%%%%%%%%%%%%%%%%%%%%%%%%%%%%%%%%%%%%%%%%%%%%
%%
%% TIKZ AND FRIENDS
%%
%%%%%%%%%%%%%%%%%%%%%%%%%%%%%%%%%%%%%%%%%%%%%%%%%%%%%%%%%%%%%%%%%%%%%
\usepackage{calc}
%\usepackage{tikz}
\usepackage[betterproportions,siunitx]{circuitikz}
\usetikzlibrary{bending}
\usetikzlibrary{arrows.meta,arrows}
\usepackage{pgfplots}
\usetikzlibrary{intersections}
\usepgfplotslibrary{units}
\pgfplotsset{compat=1.16}

%\makeatletter
%\usepgflibrary{profiler}
%\pgfprofilenewforenvironment{tikzpicture}
%\pgfprofilenewforenvironment{equation}
%\pgfprofilenewforcommand{\pgfusepath}{1}
%%\pgfprofilenewforcommand{\pgf@circ@drawvoltagegeneric}{}
%\pgfprofilenewforcommand{\pgfnode}{5}
%\pgfprofilenewforcommand{\pgfmathparse}{1}
%\makeatother


% New dashed-dotted style
%\tikzstyle{loosely dashdotted} = [dash pattern=on 3pt off 4pt on \the\pgflinewidth off 4pt]
% Directory prefix for Gnuplot plots
\tikzset{prefix=gnuplot/}
% Bipoles have the same line witdh as the wires
\ctikzset{bipoles/thickness=1}
% Reinitialize the generic box for European resistor
%%%\ctikzset{bipoles/generic/height/.initial=.20}
%%%\ctikzset{bipoles/generic/width/.initial=.60}
\tikzset{bookcircuit/.style={line width=1pt,scale=1.25}}

% Get the x and y coordinates of a point
\makeatletter
\newcommand{\gettikzxy}[3]{%
  \tikz@scan@one@point\pgfutil@firstofone#1\relax
  \global\edef#2{\the\pgf@x}%
  \global\edef#3{\the\pgf@y}%
}
\makeatother

% Transform x coordinate to graph coordinate
\makeatletter
\newcommand\transformxdimension[1]{
    \pgfmathparse{((#1/\pgfplots@x@veclength)+\pgfplots@data@scale@trafo@SHIFT@x)/10^\pgfplots@data@scale@trafo@EXPONENT@x}
}
% Transform y coordinate to graph coordinate
\newcommand\transformydimension[1]{
    \pgfmathparse{((#1/\pgfplots@y@veclength)+\pgfplots@data@scale@trafo@SHIFT@y)/10^\pgfplots@data@scale@trafo@EXPONENT@y}
}
\makeatother

%%%\pgfplotsset{xticklabel={$\mathsf{\pgfmathprintnumber{\tick}}$},
%%%             yticklabel={$\mathsf{\pgfmathprintnumber{\tick}}$},
%%%             every axis/.append style={font=\sffamily}
%%%}

%%
%% Dutch inductors are like American inductors, but resistors are european
%%
\ctikzset{inductor = american}
\ctikzset{resistor = european}
%\pgf@circuit@bipole@voltage@straighttrue

\makeatletter

%%
%% Dutch independent current source
%%
%% Extra space for label
\gdef\pgf@circ@res@cursep{0ex}

%% 4 pt label separation from the arrow
\ctikzset{bipoles/isource/labelsep/.initial=4}

%%%%% Patched version of current source, adds the current arrow
\pgfcircdeclarebipole
{}
{\ctikzvalof{bipoles/isource/height}}
{isource}
{\ctikzvalof{bipoles/isource/height}}
{\ctikzvalof{bipoles/isource/width}}
{
	\pgfextra{\gdef\pgf@circ@res@cursep{\ctikzvalof{bipoles/isource/labelsep}}}
    \pgfpointorigin
    \pgfsetlinewidth{\pgfkeysvalueof{/tikz/circuitikz/bipoles/thickness}\pgfstartlinewidth}
    \pgfpathellipse{\pgfpointorigin}{\pgfpoint{0}{\pgf@circ@res@up}}{\pgfpoint{\pgf@circ@res@left}{0}}
    \pgfpathmoveto{\pgfpoint{\pgf@circ@res@step}{\pgf@circ@res@up}}
    \pgfpathlineto{\pgfpoint{\pgf@circ@res@step}{\pgf@circ@res@down}}
    \pgf@circ@draworfill
	% Next draws the arrow besides the current source
	\pgfscope
	    \pgfsetarrowsend{latex}
	    \pgfpathmoveto{\pgfpoint{0.8\pgf@circ@res@left}{1.3\pgf@circ@res@up}}
	    \pgfpathlineto{\pgfpoint{0.8\pgf@circ@res@right}{1.3\pgf@circ@res@up}}
	    \pgfusepath{draw}   %draw arrow
	\endpgfscope
}

%%
%% Dutch dependent current source, adds the current arrow
%%
\pgfcircdeclarebipole
{}
{\ctikzvalof{bipoles/cisource/height}}
{cisource}
{\ctikzvalof{bipoles/cisource/height}}
{\ctikzvalof{bipoles/cisource/width}}
{
	\pgfextra{\gdef\pgf@circ@res@cursep{\ctikzvalof{bipoles/isource/labelsep}}}
    \pgfsetlinewidth{\pgfkeysvalueof{/tikz/circuitikz/bipoles/thickness}\pgfstartlinewidth}

    \pgfpathmoveto{\pgfpoint{\pgf@circ@res@left}{\pgf@circ@res@zero}}
    \pgfpathlineto{\pgfpoint{\pgf@circ@res@zero}{\pgf@circ@res@up}}
    \pgfpathlineto{\pgfpoint{\pgf@circ@res@right}{\pgf@circ@res@zero}}
    \pgfpathlineto{\pgfpoint{\pgf@circ@res@zero}{\pgf@circ@res@down}}
    \pgfpathclose
    \pgf@circ@draworfill

    \pgfpathmoveto{\pgfpoint{\pgf@circ@res@zero}{\pgf@circ@res@up}}
    \pgfpathlineto{\pgfpoint{\pgf@circ@res@zero}{\pgf@circ@res@down}}
    \pgfusepath{draw}

	% Next draws the arrow besides the current source
	\pgfscope
	    \pgfsetarrowsend{latex}
	    \pgfpathmoveto{\pgfpoint{0.8\pgf@circ@res@left}{1.3\pgf@circ@res@up}}
	    \pgfpathlineto{\pgfpoint{0.8\pgf@circ@res@right}{1.3\pgf@circ@res@up}}
	    \pgfusepath{draw}   %draw arrow
	\endpgfscope

}

%%
%% Patched version of drawing labels: adds extra space for the current source labels
%%
\def\pgf@circ@drawreglabels#1{
    %Now calculate all shape positions
    %Use mid-anchor at x-axis and base-anchor at y-axis, respectively.
    %All points between will be addressed by angled-anchors:
    \pgfextra{
        % scale ex-distance to make it independent on scale
        % thanks @marmot see https://tex.stackexchange.com/a/476018/38080
        \pgfgettransformentries{\tmp}{\tmp}{\tmp}{\myscale}{\tmp}{\tmp}
        \pgfmathsetlength\pgf@circ@res@temp{\pgf@circ@ls/\myscale}
        \pgfmathadd{\pgf@circ@labanc}{90}
        \pgfmathround{\pgfmathresult}
        \def\pgf@circ@labanctext{\pgf@circ@labanc}
        \edef\pgf@circ@temp{\expandafter\pgf@circ@stripdecimals\pgfmathresult\pgf@nil}
        \pgfmathparse{mod(\pgf@circ@temp,180)>135?mod(\pgf@circ@temp,180)-180:mod(\pgf@circ@temp,180)}
        \edef\pgfcircmathresult{\expandafter\pgf@circ@stripdecimals\pgfmathresult\pgf@nil}
    }
    %Values around 90 are at both y-axis
    \ifnum \pgfcircmathresult > 84 \ifnum \pgfcircmathresult< 96
        \pgfextra{\edef\pgf@circ@labpos{\expandafter\pgf@circ@stripdecimals\pgf@circ@direction\pgf@nil}}
        \ifnum \pgf@circ@labpos > 180
            \ifnum \ctikzvalof{bipole/#1/position} > 0
                    \pgfextra{\def\pgf@circ@labanctext{mid west}}
            \else
                    \pgfextra{\def\pgf@circ@labanctext{mid east}}
            \fi
        \else
            \ifnum \ctikzvalof{bipole/#1/position} > 0
                    \pgfextra{\def\pgf@circ@labanctext{mid east}}
            \else
                    \pgfextra{\def\pgf@circ@labanctext{mid west}}
            \fi
        \fi
    \fi\fi
    %Values between -5 and 5 are at pos /neg x-axis
    \pgfextra{\def\uffa{}\newdimen\realshift\realshift=\dimexpr1pt\relax}
        \ifnum \pgfcircmathresult <6 \ifnum \pgfcircmathresult > -6
            \ifnum \ctikzvalof{bipole/#1/position} < 0
                \ifnum \pgf@circ@labanc > 90
                    % using base coordinate instead of south to naturally align
                    % symbols with descendants; but this invalidate the effect of
                    % the inner sep, so recover it by shifting the anchor
                    % reset cm is not working sometime, use @marmot solution
                    % see https://tex.stackexchange.com/a/476018/38080
                    (labelcoor) ++(-\pgf@circ@labanc:\pgf@circ@res@temp+\pgf@circ@res@cursep) coordinate(labelcoor)
                    \pgfextra{\def\pgf@circ@labanctext{base}}%base
                \else
                    \pgfextra{\def\pgf@circ@labanctext{north}}%north
                \fi
             \else
                \ifnum \pgf@circ@labanc < 90
                    % shift, as above
                    (labelcoor) ++(-\pgf@circ@labanc:\pgf@circ@res@temp+\pgf@circ@res@cursep) coordinate(labelcoor)
                    \pgfextra{\def\pgf@circ@labanctext{base}}%base
                \else
                    \ifnum \pgf@circ@labanc > 180
                        % this shouldn't  happen, but somehow it does (270 degree anchors)
                        % shift, as above
                        (labelcoor) ++(-\pgf@circ@labanc:\pgf@circ@res@temp+\pgf@circ@res@cursep) coordinate(labelcoor)
                         \pgfextra{\def\pgf@circ@labanctext{base}}%base
                    \else
                      \pgfextra{\def\pgf@circ@labanctext{north}}%north
                   \fi
                \fi
            \fi
        \fi\fi
    	(labelcoor) node[anchor=\pgf@circ@labanctext,
        inner sep=0.5\pgf@circ@res@temp, outer sep=\pgf@circ@res@cursep,
        ](\ctikzvalof{bipole/name}#1){\strut\pgf@circ@finallabels{#1}%
    }
    %% (posibly) current souce done, reset current source label separation
	\pgfextra{\gdef\pgf@circ@res@cursep{0ex}}
}

%%
%% Create a Dutch version of a voltage source
%%
\ctikzset{voltage/american font/.initial={\normalsize}}
\def\pgf@circ@drawvoltagegenerator{
    % the following is affected indirectly by voltage/shift, you can move the arrow with voltage/bump a.
    % it's not perfect, but I can't find the way to do it correctly...
    \pgfextra{
        \edef\shiftv{\ctikzvalof{voltage/shift}}
        \edef\bumpa{\ctikzvalof{voltage/bump a}}
        \pgfmathsetmacro{\bumpaplus}{\bumpa + 0.5*\shiftv} % coefficient added "by feel". Sorry.
    }
    \ifpgf@circuit@bipole@voltage@below
        coordinate (pgfcirc@Vfrom) at ($(\ctikzvalof{bipole/name}.center) ! \bumpaplus ! (\ctikzvalof{bipole/name}.-120)$)
        coordinate (pgfcirc@Vto) at ($(\ctikzvalof{bipole/name}.center) ! \bumpaplus ! (\ctikzvalof{bipole/name}.-60)$)
    \else
        coordinate (pgfcirc@Vfrom) at ($ (\ctikzvalof{bipole/name}.center) ! \bumpaplus ! (\ctikzvalof{bipole/name}.120)$)
        coordinate (pgfcirc@Vto) at ($ (\ctikzvalof{bipole/name}.center) ! \bumpaplus ! (\ctikzvalof{bipole/name}.60)$)
    \fi
    % fix the (unused in this case) Vcont1/2 coords for label placement along the line
    coordinate (pgfcirc@Vcont1) at (pgfcirc@Vto)
    coordinate (pgfcirc@Vcont2) at (pgfcirc@Vfrom)
    \ifpgf@circuit@europeanvoltage
        \ifpgf@circuit@bipole@voltage@backward
            %(pgfcirc@Vto)  -- node[currarrow, sloped,  allow upside down, pos=1] {} (pgfcirc@Vfrom)
            (pgfcirc@Vfrom)  node[node font=\pgf@circ@avfont] {\pgf@circ@avplus}
            (pgfcirc@Vto) node[node font=\pgf@circ@avfont] {\pgf@circ@avminus}
        \else
            %(pgfcirc@Vfrom)  -- node[currarrow, sloped,  allow upside down, pos=1] {} (pgfcirc@Vto)
            (pgfcirc@Vfrom)  node[node font=\pgf@circ@avfont] {\pgf@circ@avminus}
            (pgfcirc@Vto) node[node font=\pgf@circ@avfont] {\pgf@circ@avplus}
        \fi
        \else% american voltage
        \ifpgf@circuit@bipole@voltageoutsideofsymbol
            % if it is a battery, must put + and -

            \ifpgf@circ@fixbatteries
                \ifpgf@circuit@bipole@voltage@backward
                    (pgfcirc@Vfrom)  node[node font=\pgf@circ@avfont] {\pgf@circ@avplus}
                    (pgfcirc@Vto) node[node font=\pgf@circ@avfont] {\pgf@circ@avminus}
                \else
                    (pgfcirc@Vfrom)  node[node font=\pgf@circ@avfont] {\pgf@circ@avminus}
                    (pgfcirc@Vto) node[node font=\pgf@circ@avfont] {\pgf@circ@avplus}
                \fi
                \else
                \ifpgf@circuit@bipole@voltage@backward
                    (pgfcirc@Vfrom)  node[node font=\pgf@circ@avfont] {\pgf@circ@avminus}
                    (pgfcirc@Vto) node[node font=\pgf@circ@avfont] {\pgf@circ@avplus}
                \else
                    (pgfcirc@Vfrom)  node[node font=\pgf@circ@avfont] {\pgf@circ@avplus}
                    (pgfcirc@Vto) node[node font=\pgf@circ@avfont] {\pgf@circ@avminus}
                \fi
            \fi
        \fi
    \fi
}

%%
%% Patched output routine for generic bipoles, replaces voltags arrow by '+' and '-'
%%
\def\pgf@circ@drawvoltagegeneric{
    \pgfextra{
        \edef\pgf@temp{/tikz/circuitikz/bipoles/\pgfkeysvalueof{/tikz/circuitikz/bipole/kind}/voltage/straight label distance}
        \pgfkeysifdefined{\pgf@temp}
        {
            \edef\partheight{\ctikzvalof{bipoles/\pgfkeysvalueof{/tikz/circuitikz/bipole/kind}/voltage/straight label distance}}
            \edef\tmpdistfromline{(\partheight\pgf@circ@Rlen)}
        }
        {
            \pgfkeysifdefined{/tikz/circuitikz/bipoles/voltage/straight label distance}
            {
                \edef\partheight{\ctikzvalof{bipoles/voltage/straight label distance}}
                \edef\tmpdistfromline{(\partheight\pgf@circ@Rlen)}
            }
            {%calculate default value from part height
                \edef\pgf@temp{/tikz/circuitikz/bipoles/\pgfkeysvalueof{/tikz/circuitikz/bipole/kind}/height}
                \pgfkeysifdefined{\pgf@temp}
                {
                    \edef\partheight{0.5*\ctikzvalof{bipoles/\pgfkeysvalueof{/tikz/circuitikz/bipole/kind}/height}}
                    \edef\tmpdistfromline{(\partheight\pgf@circ@Rlen+0.2\pgf@circ@Rlen)}
                }
                {
                    \edef\tmpdistfromline{(.5\pgf@circ@Rlen)} %fallback to fixed value
                }
            }
        }
        \ifnum \ctikzvalof{mirror value}=-1
        \ifpgf@circuit@bipole@inverted
            \ifpgf@circuit@bipole@voltage@straight
                \def\distfromline{\tmpdistfromline}
            \else
                \def\distfromline{\ctikzvalof{voltage/distance from line}\pgf@circ@Rlen}
            \fi
            \else
            \ifpgf@circuit@bipole@voltage@straight
                \def\distfromline{-\tmpdistfromline}
            \else
                \def\distfromline{-\ctikzvalof{voltage/distance from line}\pgf@circ@Rlen}
            \fi
        \fi
        \else
            \ifpgf@circuit@bipole@inverted
                \ifpgf@circuit@bipole@voltage@straight
                    \def\distfromline{-\tmpdistfromline}
                \else
                    \def\distfromline{-\ctikzvalof{voltage/distance from line}\pgf@circ@Rlen}
                \fi
                \else
                \ifpgf@circuit@bipole@voltage@straight
                    \def\distfromline{\tmpdistfromline}
                \else
                    \def\distfromline{\ctikzvalof{voltage/distance from line}\pgf@circ@Rlen}
                \fi
            \fi
        \fi
        \ifpgf@circuit@bipole@voltage@below
            \def\pgf@circ@voltage@angle{90}
        \else
            \def\pgf@circ@voltage@angle{-90}
        \fi
        \edef\pgf@temp{/tikz/circuitikz/bipoles/\pgfkeysvalueof{/tikz/circuitikz/bipole/kind}/voltage/distance from node}
        \pgfkeysifdefined{\pgf@temp}
        { \edef\distacefromnode{\ctikzvalof{bipoles/\pgfkeysvalueof{/tikz/circuitikz/bipole/kind}/voltage/distance from node}} }
        { \edef\distacefromnode{\ctikzvalof{voltage/distance from node}} }
        \edef\pgf@temp{/tikz/circuitikz/bipoles/\pgfkeysvalueof{/tikz/circuitikz/bipole/kind}/voltage/bump b}
        \pgfkeysifdefined{\pgf@temp}
        { \edef\bumpb{\ctikzvalof{bipoles/\pgfkeysvalueof{/tikz/circuitikz/bipole/kind}/voltage/bump b}} }
        { \edef\bumpb{\ctikzvalof{voltage/bump b}} }
        \edef\shiftv{\ctikzvalof{voltage/shift}}
        \pgfmathsetmacro{\bumpb}{\bumpb + \shiftv} %% adjust the bump is shift
        \ifpgf@circuit@bipole@inverted
            \pgfmathsetmacro{\shiftv}{-\shiftv}
        \fi
        \ifnum \ctikzvalof{mirror value} = -1
            \pgfmathsetmacro{\shiftv}{-\shiftv}
        \fi
    }
    % %\pgf@circ@Rlen/\pgfkeysvalueof{/tikz/circuitikz/current arrow scale} is equal to the length of the currarrow
    coordinate (pgfcirc@midtmp) at ($(\tikztostart) ! \pgf@circ@Rlen/\pgfkeysvalueof{/tikz/circuitikz/current arrow scale} ! (anchorstartnode)$) %absolute move, minimum space is length of arrowhead
    coordinate (pgfcirc@midtmp) at ($(pgfcirc@midtmp) ! \distacefromnode ! (anchorstartnode)$)

    coordinate (pgfcirc@Vfrom) at ($(pgfcirc@midtmp) ! -\distfromline ! \pgf@circ@voltage@angle:(anchorstartnode)$)
    coordinate (pgfcirc@midtmp) at ($(\tikztotarget) ! \pgf@circ@Rlen/\pgfkeysvalueof{/tikz/circuitikz/current arrow scale} ! (anchorendnode)$)%absolute move, minimum space is length of arrowhead
    coordinate (pgfcirc@midtmp) at ($(pgfcirc@midtmp) ! \distacefromnode ! (anchorendnode)$)

    coordinate (pgfcirc@Vto) at ($(pgfcirc@midtmp) ! \distfromline ! \pgf@circ@voltage@angle : (anchorendnode)$)

    \ifpgf@circuit@bipole@voltage@below
        coordinate (pgfcirc@Vto) at ($(pgfcirc@Vto) ! \shiftv!90 :  (anchorendnode)$)
        coordinate (pgfcirc@Vfrom) at ($(pgfcirc@Vfrom) ! \shiftv!-90 :  (anchorstartnode)$)
        coordinate (pgfcirc@Vcont1) at ($(\ctikzvalof{bipole/name}.center) ! \bumpb ! (\ctikzvalof{bipole/name}.-110)$)
        coordinate (pgfcirc@Vcont2) at ($(\ctikzvalof{bipole/name}.center) ! \bumpb ! (\ctikzvalof{bipole/name}.-70)$)
    \else
        coordinate (pgfcirc@Vto) at ($(pgfcirc@Vto) ! -\shiftv!90 :  (anchorendnode)$)
        coordinate (pgfcirc@Vfrom) at ($(pgfcirc@Vfrom) ! -\shiftv!-90 :  (anchorstartnode)$)
        coordinate (pgfcirc@Vcont1) at ($(\ctikzvalof{bipole/name}.center) ! \bumpb ! (\ctikzvalof{bipole/name}.110)$)
        coordinate (pgfcirc@Vcont2) at ($(\ctikzvalof{bipole/name}.center) ! \bumpb ! (\ctikzvalof{bipole/name}.70)$)
    \fi

    \ifpgf@circuit@europeanvoltage
        \ifpgf@circuit@bipole@voltage@straight
            \ifpgf@circuit@bipole@voltage@backward
                %(pgfcirc@Vto) --(pgfcirc@Vfrom) node[currarrow, sloped,  allow upside down, pos=1,anchor=tip] {}
                (pgfcirc@Vfrom) node[inner sep=0ex, node font=\pgf@circ@avfont,
                    anchor=\pgf@circ@bipole@voltage@label@anchor]{\pgf@circ@avminus}
                (pgfcirc@Vto) node[inner sep=0ex, node font=\pgf@circ@avfont,
                    anchor=\pgf@circ@bipole@voltage@label@anchor]{\pgf@circ@avplus}
            \else
                %(pgfcirc@Vfrom) --(pgfcirc@Vto) node[currarrow, sloped,  allow upside down, pos=1,anchor=tip] {}
                (pgfcirc@Vfrom) node[inner sep=0ex, node font=\pgf@circ@avfont,
                    anchor=\pgf@circ@bipole@voltage@label@anchor]{\pgf@circ@avplus}
                (pgfcirc@Vto) node[inner sep=0ex, node font=\pgf@circ@avfont,
                    anchor=\pgf@circ@bipole@voltage@label@anchor]{\pgf@circ@avminus}
            \fi
            \else
            \ifpgf@circuit@bipole@voltage@backward
                %(pgfcirc@Vto) .. controls (pgfcirc@Vcont2)  and (pgfcirc@Vcont1) ..
                %node[currarrow, sloped,  allow upside down, pos=1] {}
                %(pgfcirc@Vfrom)
                (pgfcirc@Vfrom) node[inner sep=0ex, node font=\pgf@circ@avfont,
                    anchor=\pgf@circ@bipole@voltage@label@anchor]{\pgf@circ@avminus}
                (pgfcirc@Vto) node[inner sep=0ex, node font=\pgf@circ@avfont,
                    anchor=\pgf@circ@bipole@voltage@label@anchor]{\pgf@circ@avplus}
            \else
                %(pgfcirc@Vfrom) .. controls (pgfcirc@Vcont1)  and (pgfcirc@Vcont2) ..
                %node[currarrow, sloped,  allow upside down, pos=1] {}
                %(pgfcirc@Vto)
                (pgfcirc@Vfrom) node[inner sep=0ex, node font=\pgf@circ@avfont,
                    anchor=\pgf@circ@bipole@voltage@label@anchor]{\pgf@circ@avplus}
                (pgfcirc@Vto) node[inner sep=0ex, node font=\pgf@circ@avfont,
                    anchor=\pgf@circ@bipole@voltage@label@anchor]{\pgf@circ@avminus}
            \fi
        \fi
        \else
        \ifpgf@circuit@bipole@voltage@backward
            \ifpgf@circ@oldvoltagedirection
                (pgfcirc@Vfrom) node[inner sep=0, node font=\pgf@circ@avfont,
                    anchor=\pgf@circ@bipole@voltage@label@anchor]{\pgf@circ@avplus}
                (pgfcirc@Vto) node[inner sep=0, node font=\pgf@circ@avfont,
                    anchor=\pgf@circ@bipole@voltage@label@anchor]{\pgf@circ@avminus}
            \else
                (pgfcirc@Vfrom) node[inner sep=0, node font=\pgf@circ@avfont,
                    anchor=\pgf@circ@bipole@voltage@label@anchor]{\pgf@circ@avminus}
                (pgfcirc@Vto) node[inner sep=0, node font=\pgf@circ@avfont,
                    anchor=\pgf@circ@bipole@voltage@label@anchor]{\pgf@circ@avplus}
            \fi
            \else
            \ifpgf@circ@oldvoltagedirection
                (pgfcirc@Vfrom) node[inner sep=0, node font=\pgf@circ@avfont,
                    anchor=\pgf@circ@bipole@voltage@label@anchor]{\pgf@circ@avminus}
                (pgfcirc@Vto) node[inner sep=0, node font=\pgf@circ@avfont,
                    anchor=\pgf@circ@bipole@voltage@label@anchor]{\pgf@circ@avplus}
            \else
                (pgfcirc@Vfrom) node[inner sep=0, node font=\pgf@circ@avfont,
                    anchor=\pgf@circ@bipole@voltage@label@anchor]{\pgf@circ@avplus}
                (pgfcirc@Vto) node[inner sep=0, node font=\pgf@circ@avfont,
                    anchor=\pgf@circ@bipole@voltage@label@anchor]{\pgf@circ@avminus}
            \fi
        \fi
    \fi
}



%%%%% Create a Dutch version of a voltage source
%%%\makeatletter
%%%%% Output routine for voltage sources
%%%\def\pgf@circ@drawvoltagegenerator{
%%%	\ifpgf@circuit@bipole@voltage@below
%%%		coordinate (pgfcirc@Vfrom) at ($(\ctikzvalof{bipole/name}.center) ! \ctikzvalof{voltage/bump a} ! (\ctikzvalof{bipole/name}.-120)$)
%%%		coordinate (pgfcirc@Vto) at ($(\ctikzvalof{bipole/name}.center) ! \ctikzvalof{voltage/bump a} ! (\ctikzvalof{bipole/name}.-60)$)
%%%	\else
%%%		coordinate (pgfcirc@Vfrom) at ($ (\ctikzvalof{bipole/name}.center) ! \ctikzvalof{voltage/bump a} ! (\ctikzvalof{bipole/name}.120)$)
%%%		coordinate (pgfcirc@Vto) at ($ (\ctikzvalof{bipole/name}.center) ! \ctikzvalof{voltage/bump a} ! (\ctikzvalof{bipole/name}.60)$)
%%%	\fi
%%%	\ifpgf@circuit@europeanvoltage
%%%		\ifpgf@circuit@bipole@voltage@backward
%%%				(pgfcirc@Vfrom)  node {$-$}  (pgfcirc@Vto) node {$+$}
%%%		\else
%%%				(pgfcirc@Vfrom)  node {$+$}  (pgfcirc@Vto) node {$-$}
%%%		\fi
%%%	\else% american voltage
%%%		\ifpgf@circuit@bipole@voltageoutsideofsymbol
%%%		% if it is a battery, must put + and -
%%%			\ifpgf@circuit@bipole@voltage@backward
%%%				(pgfcirc@Vfrom)  node {$-$}  (pgfcirc@Vto) node {$+$}
%%%			\else
%%%				(pgfcirc@Vfrom)  node {$+$}  (pgfcirc@Vto) node {$-$}
%%%			\fi
%%%		\fi
%%%	\fi
%%%}
%%%

%%%
%% Output routine for generic bipoles
%%%\def\pgf@circ@drawvoltagegeneric{
%%%	\pgfextra{
%%%			\edef\pgf@temp{/tikz/circuitikz/bipoles/\pgfkeysvalueof{/tikz/circuitikz/bipole/kind}/voltage/straight label distance}
%%%			\pgfkeysifdefined{\pgf@temp}
%%%				{ 
%%%					\edef\partheight{\ctikzvalof{bipoles/\pgfkeysvalueof{/tikz/circuitikz/bipole/kind}/voltage/straight label distance}}
%%%					\edef\tmpdistfromline{(\partheight\pgf@circ@Rlen)}
%%%				}
%%%				{
%%%				\pgfkeysifdefined{/tikz/circuitikz/bipoles/voltage/straight label distance}
%%%					{
%%%						\edef\partheight{\ctikzvalof{bipoles/voltage/straight label distance}}
%%%						\edef\tmpdistfromline{(\partheight\pgf@circ@Rlen)}
%%%					}
%%%					{%calculate default value from part height
%%%						\edef\partheight{0.5*\ctikzvalof{bipoles/\pgfkeysvalueof{/tikz/circuitikz/bipole/kind}/height}}
%%%						\edef\tmpdistfromline{(\partheight\pgf@circ@Rlen+0.2\pgf@circ@Rlen)}
%%%					}
%%%				}
%%%		\ifnum \ctikzvalof{mirror value}=-1
%%%			\ifpgf@circuit@bipole@inverted
%%%				\ifpgf@circuit@bipole@voltage@straight
%%%					\def\distfromline{\tmpdistfromline}
%%%				\else
%%%					\def\distfromline{\ctikzvalof{voltage/distance from line}\pgf@circ@Rlen}
%%%				\fi
%%%			\else
%%%				\ifpgf@circuit@bipole@voltage@straight
%%%					\def\distfromline{-\tmpdistfromline}
%%%				\else
%%%					\def\distfromline{-\ctikzvalof{voltage/distance from line}\pgf@circ@Rlen}
%%%				\fi
%%%			\fi
%%%		\else
%%%			\ifpgf@circuit@bipole@inverted
%%%				\ifpgf@circuit@bipole@voltage@straight
%%%					\def\distfromline{-\tmpdistfromline}
%%%				\else
%%%					\def\distfromline{-\ctikzvalof{voltage/distance from line}\pgf@circ@Rlen}
%%%				\fi
%%%			\else
%%%				\ifpgf@circuit@bipole@voltage@straight
%%%					\def\distfromline{\tmpdistfromline}
%%%				\else
%%%					\def\distfromline{\ctikzvalof{voltage/distance from line}\pgf@circ@Rlen}
%%%				\fi
%%%			\fi
%%%		\fi
%%%		\ifpgf@circuit@bipole@voltage@below
%%%			\def\pgf@circ@voltage@angle{90}
%%%		\else
%%%			\def\pgf@circ@voltage@angle{-90}
%%%		\fi
%%%		\edef\pgf@temp{/tikz/circuitikz/bipoles/\pgfkeysvalueof{/tikz/circuitikz/bipole/kind}/voltage/distance from node}
%%%		\pgfkeysifdefined{\pgf@temp}
%%%			{ \edef\distacefromnode{\ctikzvalof{bipoles/\pgfkeysvalueof{/tikz/circuitikz/bipole/kind}/voltage/distance from node}} }
%%%			{ \edef\distacefromnode{\ctikzvalof{voltage/distance from node}} }
%%%		\edef\pgf@temp{/tikz/circuitikz/bipoles/\pgfkeysvalueof{/tikz/circuitikz/bipole/kind}/voltage/bump b}
%%%		\pgfkeysifdefined{\pgf@temp}
%%%			{ \edef\bumpb{\ctikzvalof{bipoles/\pgfkeysvalueof{/tikz/circuitikz/bipole/kind}/voltage/bump b}} }
%%%			{ \edef\bumpb{\ctikzvalof{voltage/bump b}} }
%%%	}
%%%	% %\pgf@circ@Rlen/16 is equal to the length of the currarrow
%%%	coordinate (pgfcirc@midtmp) at ($(\tikztostart) ! \pgf@circ@Rlen/16 ! (anchorstartnode)$) %absolute move, minimum space is length of arrowhead
%%%	coordinate (pgfcirc@midtmp) at ($(pgfcirc@midtmp) ! \distacefromnode ! (anchorstartnode)$)
%%%
%%%	coordinate (pgfcirc@Vfrom) at ($(pgfcirc@midtmp) ! -\distfromline ! \pgf@circ@voltage@angle:(anchorstartnode)$)
%%%	coordinate (pgfcirc@midtmp) at ($(\tikztotarget) ! \pgf@circ@Rlen/16 ! (anchorendnode)$)%absolute move, minimum space is length of arrowhead
%%%	coordinate (pgfcirc@midtmp) at ($(pgfcirc@midtmp) ! \distacefromnode ! (anchorendnode)$)
%%%
%%%	coordinate (pgfcirc@Vto) at ($(pgfcirc@midtmp) ! \distfromline ! \pgf@circ@voltage@angle : (anchorendnode)$)
%%%
%%%	\ifpgf@circuit@bipole@voltage@below
%%%		coordinate (pgfcirc@Vcont1) at ($(\ctikzvalof{bipole/name}.center) ! \bumpb ! (\ctikzvalof{bipole/name}.-110)$)
%%%		coordinate (pgfcirc@Vcont2) at ($(\ctikzvalof{bipole/name}.center) ! \bumpb ! (\ctikzvalof{bipole/name}.-70)$)
%%%	\else
%%%		coordinate (pgfcirc@Vcont1) at ($(\ctikzvalof{bipole/name}.center) ! \bumpb ! (\ctikzvalof{bipole/name}.110)$)
%%%		coordinate (pgfcirc@Vcont2) at ($(\ctikzvalof{bipole/name}.center) ! \bumpb ! (\ctikzvalof{bipole/name}.70)$)
%%%	\fi
%%%
%%%	\ifpgf@circuit@europeanvoltage
%%%		\ifpgf@circuit@bipole@voltage@straight
%%%			\ifpgf@circuit@bipole@voltage@backward
%%%				(pgfcirc@Vto) --(pgfcirc@Vfrom) node[currarrow, sloped,  allow upside down, pos=1,anchor=tip] {}
%%%			\else
%%%				(pgfcirc@Vfrom) --(pgfcirc@Vto) node[currarrow, sloped,  allow upside down, pos=1,anchor=tip] {}
%%%			\fi
%%%		\else
%%%			\ifpgf@circuit@bipole@voltage@backward
%%%%%%				(pgfcirc@Vto) .. controls (pgfcirc@Vcont2)  and (pgfcirc@Vcont1) ..
%%%%%%					node[currarrow, sloped,  allow upside down, pos=1] {}
%%%%%%				(pgfcirc@Vfrom)
%%%				(pgfcirc@Vfrom) node[inner sep=0, anchor=\pgf@circ@bipole@voltage@label@anchor]{\small$+$}
%%%				(pgfcirc@Vto) node[inner sep=0, anchor=\pgf@circ@bipole@voltage@label@anchor]{\small$-$}
%%%			\else
%%%%%%				(pgfcirc@Vfrom) .. controls (pgfcirc@Vcont1)  and (pgfcirc@Vcont2) ..
%%%%%%					node[currarrow, sloped,  allow upside down, pos=1] {}
%%%%%%				(pgfcirc@Vto)
%%%				(pgfcirc@Vfrom) node[inner sep=0, anchor=\pgf@circ@bipole@voltage@label@anchor]{\small$+$}
%%%				(pgfcirc@Vto) node[inner sep=0, anchor=\pgf@circ@bipole@voltage@label@anchor]{\small$-$}
%%%			\fi
%%%		\fi
%%%	\else
%%%		\ifpgf@circuit@bipole@voltage@backward
%%%			\ifpgf@circ@oldvoltagedirection
%%%				(pgfcirc@Vfrom) node[inner sep=0, anchor=\pgf@circ@bipole@voltage@label@anchor]{$+$}
%%%				(pgfcirc@Vto) node[inner sep=0, anchor=\pgf@circ@bipole@voltage@label@anchor]{$-$}
%%%			\else
%%%				(pgfcirc@Vfrom) node[inner sep=0, anchor=\pgf@circ@bipole@voltage@label@anchor]{$+$}
%%%				(pgfcirc@Vto) node[inner sep=0, anchor=\pgf@circ@bipole@voltage@label@anchor]{$-$}
%%%			\fi
%%%		\else
%%%			\ifpgf@circ@oldvoltagedirection
%%%				(pgfcirc@Vfrom) node[inner sep=0, anchor=\pgf@circ@bipole@voltage@label@anchor]{$+$}
%%%				(pgfcirc@Vto) node[inner sep=0, anchor=\pgf@circ@bipole@voltage@label@anchor]{$-$}
%%%			\else
%%%				(pgfcirc@Vfrom) node[inner sep=0, anchor=\pgf@circ@bipole@voltage@label@anchor]{$+$}
%%%				(pgfcirc@Vto) node[inner sep=0, anchor=\pgf@circ@bipole@voltage@label@anchor]{$-$}
%%%			\fi
%%%		\fi
%%%	\fi
%%%}


%%%%%%%%%%%%%%%%%%%%%%%%%%%%%%%%%%%%%%%%%%%%%%%%%%%%%%%%%%%%%%%%%%%%%

%%
%% based on https://tex.stackexchange.com/questions/476045/euler-and-minus-sign
%%
%% Imaginary unit, e, and Euler
\newcommand\imaginaryunit{j}                   % the imaginary unit, i for mathematician and theoretical physicist,
                                               % j for the rest of the world.
\newcommand\imunit{\mathrm{\imaginaryunit}}    % ... in upright math
\newcommand\ce{\mathrm{e}}                     % the constant e, upright of course

\newcommand{\epowre}[1]{\ce^{#1}}              % e to the power of real
\makeatletter
\newcommand{\fiximunit@@}{\if\imaginaryunit j\,\fi}
\newcommand{\epowim}[1]{\ce^{\epowim@#1}}      % e to the power of imaginary
\newcommand{\epowim@}{\@ifnextchar-{\epowim@@}{\epowim@@{\fiximunit@@}}}
\newcommand{\epowim@@}[1]{#1\imunit}
\newcommand{\epowim@@@}{\@ifnextchar-{\epowim@@@@}{+\epowim@@@@{}}}
\newcommand{\epowim@@@@}[1]{#1\imunit}
\newcommand{\epowcom}[2]{\ce^{#1\epowim@@@#2}} % e to the power of complex
\newcommand{\cis}[1]{\cis@#1}                  % cos + imaginary sin
\newcommand{\cis@}{\@ifnextchar-{\cis@@@}{\cis@@}}
\newcommand{\cis@@}[1]{\cos#1 + \imunit\sin#1}
\newcommand{\cis@@@}[2]{\cos#2 - \imunit\sin#2}
%\renewcommand{\Re}{\mathrm{Re}}  % Redefine \Re
%\renewcommand{\Im}{\mathrm{Im}}  % Redefine \Im

\makeatother

\endinput
