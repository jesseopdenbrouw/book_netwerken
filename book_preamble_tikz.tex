%%%%%%%%%%%%%%%%%%%%%%%%%%%%%%%%%%%%%%%%%%%%%%%%%%%%%%%%%%%%%%%%%%%%%
%%
%% TIKZ AND FRIENDS
%%
%%%%%%%%%%%%%%%%%%%%%%%%%%%%%%%%%%%%%%%%%%%%%%%%%%%%%%%%%%%%%%%%%%%%%
\usepackage{calc}
\usepackage{tikz}
\usepackage[european,betterproportions,americaninductors]{circuitikz}
%\usepackage[]{circuitikz}
\usepackage{pgfplots}
\usetikzlibrary{intersections}
\usepgfplotslibrary{units}

% New dashed-dotted style
%\tikzstyle{loosely dashdotted} = [dash pattern=on 3pt off 4pt on \the\pgflinewidth off 4pt]
% Directory prefix for Gnuplot plots
\tikzset{prefix=gnuplot/}
% Bipoles have the same line witdh as the wires
\ctikzset{bipoles/thickness=1}
% Reinitialize the generic box for European resistor
%%%\ctikzset{bipoles/generic/height/.initial=.20}
%%%\ctikzset{bipoles/generic/width/.initial=.60}
\tikzset{bookcircuit/.style={line width=1pt,scale=1.25}}

% Get the x and y coordinates of a point
\makeatletter
\newcommand{\gettikzxy}[3]{%
  \tikz@scan@one@point\pgfutil@firstofone#1\relax
  \global\edef#2{\the\pgf@x}%
  \global\edef#3{\the\pgf@y}%
}
\makeatother

% Transform x coordinate to graph coordinate
\makeatletter
\newcommand\transformxdimension[1]{
    \pgfmathparse{((#1/\pgfplots@x@veclength)+\pgfplots@data@scale@trafo@SHIFT@x)/10^\pgfplots@data@scale@trafo@EXPONENT@x}
}
% Transform y coordinate to graph coordinate
\newcommand\transformydimension[1]{
    \pgfmathparse{((#1/\pgfplots@y@veclength)+\pgfplots@data@scale@trafo@SHIFT@y)/10^\pgfplots@data@scale@trafo@EXPONENT@y}
}
\makeatother

%%%\pgfplotsset{xticklabel={$\mathsf{\pgfmathprintnumber{\tick}}$},
%%%             yticklabel={$\mathsf{\pgfmathprintnumber{\tick}}$},
%%%             every axis/.append style={font=\sffamily}
%%%}

%% Create a Dutch version of a voltage source
\makeatletter
%% Output routine for voltage sources
\def\pgf@circ@drawvoltagegenerator{
	\ifpgf@circuit@bipole@voltage@below
		coordinate (pgfcirc@Vfrom) at ($(\ctikzvalof{bipole/name}.center) ! \ctikzvalof{voltage/bump a} ! (\ctikzvalof{bipole/name}.-120)$)
		coordinate (pgfcirc@Vto) at ($(\ctikzvalof{bipole/name}.center) ! \ctikzvalof{voltage/bump a} ! (\ctikzvalof{bipole/name}.-60)$)
	\else
		coordinate (pgfcirc@Vfrom) at ($ (\ctikzvalof{bipole/name}.center) ! \ctikzvalof{voltage/bump a} ! (\ctikzvalof{bipole/name}.120)$)
		coordinate (pgfcirc@Vto) at ($ (\ctikzvalof{bipole/name}.center) ! \ctikzvalof{voltage/bump a} ! (\ctikzvalof{bipole/name}.60)$)
	\fi
	\ifpgf@circuit@europeanvoltage
		\ifpgf@circuit@bipole@voltage@backward
				(pgfcirc@Vfrom)  node {$-$}  (pgfcirc@Vto) node {$+$}
		\else
				(pgfcirc@Vfrom)  node {$+$}  (pgfcirc@Vto) node {$-$}
		\fi
	\else% american voltage
		\ifpgf@circuit@bipole@voltageoutsideofsymbol
		% if it is a battery, must put + and -
			\ifpgf@circuit@bipole@voltage@backward
				(pgfcirc@Vfrom)  node {$-$}  (pgfcirc@Vto) node {$+$}
			\else
				(pgfcirc@Vfrom)  node {$+$}  (pgfcirc@Vto) node {$-$}
			\fi
		\fi
	\fi
}

%% Create a Dutch version of the current source
\makeatletter
\pgfcircdeclarebipole{}{\ctikzvalof{bipoles/isource/height}}{isource}{\ctikzvalof{bipoles/isource/height}}{\ctikzvalof{bipoles/isource/width}}{
    \pgfpointorigin
    \pgfsetlinewidth{\pgfkeysvalueof{/tikz/circuitikz/bipoles/thickness}\pgfstartlinewidth}
    \pgfpathellipse{\pgfpointorigin}{\pgfpoint{0}{\pgf@circ@res@up}}{\pgfpoint{\pgf@circ@res@left}{0}}
    \pgfpathmoveto{\pgfpoint{\pgf@circ@res@step}{\pgf@circ@res@up}}
    \pgfpathlineto{\pgfpoint{\pgf@circ@res@step}{\pgf@circ@res@down}}
  \pgfusepath{draw}
  \pgfscope
    \pgfsetarrowsend{latex'}
    \pgfpathmoveto{\pgfpoint{0.8\pgf@circ@res@left}{1.3\pgf@circ@res@up}}
    \pgfpathlineto{\pgfpoint{0.8\pgf@circ@res@right}{1.3\pgf@circ@res@up}}
    \pgfusepath{draw}   %draw arrow
  \endpgfscope
}
\makeatother

%% Output routine for generic bipoles
\makeatletter
\def\pgf@circ@drawvoltagegeneric{
	\pgfextra{
			\edef\pgf@temp{/tikz/circuitikz/bipoles/\pgfkeysvalueof{/tikz/circuitikz/bipole/kind}/voltage/straight label distance}
			\pgfkeysifdefined{\pgf@temp}
				{ 
					\edef\partheight{\ctikzvalof{bipoles/\pgfkeysvalueof{/tikz/circuitikz/bipole/kind}/voltage/straight label distance}}
					\edef\tmpdistfromline{(\partheight\pgf@circ@Rlen)}
				}
				{
				\pgfkeysifdefined{/tikz/circuitikz/bipoles/voltage/straight label distance}
					{
						\edef\partheight{\ctikzvalof{bipoles/voltage/straight label distance}}
						\edef\tmpdistfromline{(\partheight\pgf@circ@Rlen)}
					}
					{%calculate default value from part height
						\edef\partheight{0.5*\ctikzvalof{bipoles/\pgfkeysvalueof{/tikz/circuitikz/bipole/kind}/height}}
						\edef\tmpdistfromline{(\partheight\pgf@circ@Rlen+0.2\pgf@circ@Rlen)}
					}
				}
		\ifnum \ctikzvalof{mirror value}=-1
			\ifpgf@circuit@bipole@inverted
				\ifpgf@circuit@bipole@voltage@straight
					\def\distfromline{\tmpdistfromline}
				\else
					\def\distfromline{\ctikzvalof{voltage/distance from line}\pgf@circ@Rlen}
				\fi
			\else
				\ifpgf@circuit@bipole@voltage@straight
					\def\distfromline{-\tmpdistfromline}
				\else
					\def\distfromline{-\ctikzvalof{voltage/distance from line}\pgf@circ@Rlen}
				\fi
			\fi
		\else
			\ifpgf@circuit@bipole@inverted
				\ifpgf@circuit@bipole@voltage@straight
					\def\distfromline{-\tmpdistfromline}
				\else
					\def\distfromline{-\ctikzvalof{voltage/distance from line}\pgf@circ@Rlen}
				\fi
			\else
				\ifpgf@circuit@bipole@voltage@straight
					\def\distfromline{\tmpdistfromline}
				\else
					\def\distfromline{\ctikzvalof{voltage/distance from line}\pgf@circ@Rlen}
				\fi
			\fi
		\fi
		\ifpgf@circuit@bipole@voltage@below
			\def\pgf@circ@voltage@angle{90}
		\else
			\def\pgf@circ@voltage@angle{-90}
		\fi
		\edef\pgf@temp{/tikz/circuitikz/bipoles/\pgfkeysvalueof{/tikz/circuitikz/bipole/kind}/voltage/distance from node}
		\pgfkeysifdefined{\pgf@temp}
			{ \edef\distacefromnode{\ctikzvalof{bipoles/\pgfkeysvalueof{/tikz/circuitikz/bipole/kind}/voltage/distance from node}} }
			{ \edef\distacefromnode{\ctikzvalof{voltage/distance from node}} }
		\edef\pgf@temp{/tikz/circuitikz/bipoles/\pgfkeysvalueof{/tikz/circuitikz/bipole/kind}/voltage/bump b}
		\pgfkeysifdefined{\pgf@temp}
			{ \edef\bumpb{\ctikzvalof{bipoles/\pgfkeysvalueof{/tikz/circuitikz/bipole/kind}/voltage/bump b}} }
			{ \edef\bumpb{\ctikzvalof{voltage/bump b}} }
	}
	% %\pgf@circ@Rlen/16 is equal to the length of the currarrow
	coordinate (pgfcirc@midtmp) at ($(\tikztostart) ! \pgf@circ@Rlen/16 ! (anchorstartnode)$) %absolute move, minimum space is length of arrowhead
	coordinate (pgfcirc@midtmp) at ($(pgfcirc@midtmp) ! \distacefromnode ! (anchorstartnode)$)

	coordinate (pgfcirc@Vfrom) at ($(pgfcirc@midtmp) ! -\distfromline ! \pgf@circ@voltage@angle:(anchorstartnode)$)
	coordinate (pgfcirc@midtmp) at ($(\tikztotarget) ! \pgf@circ@Rlen/16 ! (anchorendnode)$)%absolute move, minimum space is length of arrowhead
	coordinate (pgfcirc@midtmp) at ($(pgfcirc@midtmp) ! \distacefromnode ! (anchorendnode)$)

	coordinate (pgfcirc@Vto) at ($(pgfcirc@midtmp) ! \distfromline ! \pgf@circ@voltage@angle : (anchorendnode)$)

	\ifpgf@circuit@bipole@voltage@below
		coordinate (pgfcirc@Vcont1) at ($(\ctikzvalof{bipole/name}.center) ! \bumpb ! (\ctikzvalof{bipole/name}.-110)$)
		coordinate (pgfcirc@Vcont2) at ($(\ctikzvalof{bipole/name}.center) ! \bumpb ! (\ctikzvalof{bipole/name}.-70)$)
	\else
		coordinate (pgfcirc@Vcont1) at ($(\ctikzvalof{bipole/name}.center) ! \bumpb ! (\ctikzvalof{bipole/name}.110)$)
		coordinate (pgfcirc@Vcont2) at ($(\ctikzvalof{bipole/name}.center) ! \bumpb ! (\ctikzvalof{bipole/name}.70)$)
	\fi

	\ifpgf@circuit@europeanvoltage
		\ifpgf@circuit@bipole@voltage@straight
			\ifpgf@circuit@bipole@voltage@backward
				(pgfcirc@Vto) --(pgfcirc@Vfrom) node[currarrow, sloped,  allow upside down, pos=1,anchor=tip] {}
			\else
				(pgfcirc@Vfrom) --(pgfcirc@Vto) node[currarrow, sloped,  allow upside down, pos=1,anchor=tip] {}
			\fi
		\else
			\ifpgf@circuit@bipole@voltage@backward
%%%				(pgfcirc@Vto) .. controls (pgfcirc@Vcont2)  and (pgfcirc@Vcont1) ..
%%%					node[currarrow, sloped,  allow upside down, pos=1] {}
%%%				(pgfcirc@Vfrom)
				(pgfcirc@Vfrom) node[inner sep=0, anchor=\pgf@circ@bipole@voltage@label@anchor]{\small$+$}
				(pgfcirc@Vto) node[inner sep=0, anchor=\pgf@circ@bipole@voltage@label@anchor]{\small$-$}
			\else
%%%				(pgfcirc@Vfrom) .. controls (pgfcirc@Vcont1)  and (pgfcirc@Vcont2) ..
%%%					node[currarrow, sloped,  allow upside down, pos=1] {}
%%%				(pgfcirc@Vto)
				(pgfcirc@Vfrom) node[inner sep=0, anchor=\pgf@circ@bipole@voltage@label@anchor]{\small$+$}
				(pgfcirc@Vto) node[inner sep=0, anchor=\pgf@circ@bipole@voltage@label@anchor]{\small$-$}
			\fi
		\fi
	\else
		\ifpgf@circuit@bipole@voltage@backward
			\ifpgf@circ@oldvoltagedirection
				(pgfcirc@Vfrom) node[inner sep=0, anchor=\pgf@circ@bipole@voltage@label@anchor]{$+$}
				(pgfcirc@Vto) node[inner sep=0, anchor=\pgf@circ@bipole@voltage@label@anchor]{$-$}
			\else
				(pgfcirc@Vfrom) node[inner sep=0, anchor=\pgf@circ@bipole@voltage@label@anchor]{$+$}
				(pgfcirc@Vto) node[inner sep=0, anchor=\pgf@circ@bipole@voltage@label@anchor]{$-$}
			\fi
		\else
			\ifpgf@circ@oldvoltagedirection
				(pgfcirc@Vfrom) node[inner sep=0, anchor=\pgf@circ@bipole@voltage@label@anchor]{$+$}
				(pgfcirc@Vto) node[inner sep=0, anchor=\pgf@circ@bipole@voltage@label@anchor]{$-$}
			\else
				(pgfcirc@Vfrom) node[inner sep=0, anchor=\pgf@circ@bipole@voltage@label@anchor]{$+$}
				(pgfcirc@Vto) node[inner sep=0, anchor=\pgf@circ@bipole@voltage@label@anchor]{$-$}
			\fi
		\fi
	\fi
}
\makeatother

\usepackage{sansmath}

%%%%%%%%%%%%%%%%%%%%%%%%%%%%%%%%%%%%%%%%%%%%%%%%%%%%%%%%%%%%%%%%%%%%%

\endinput