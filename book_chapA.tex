%%%%%%%%%%%%%%%%%%%%%%%%%%%%%%%%%%%%%%%%%%%%%%%%%%%%%%%%%%%%%%%%%%%%%%%%%%%%%%%
%%%
%%%   OPLOSSEN
%%%
%%%%%%%%%%%%%%%%%%%%%%%%%%%%%%%%%%%%%%%%%%%%%%%%%%%%%%%%%%%%%%%%%%%%%%%%%%%%%


\chapter{Oplossen van lineaire vergelijkingen}
\label{cha:linsolve}
In de elektrotechniek komen veelvuldig \textsl{stelsels van lineaire vergelijkingen} voor bij het berekenen van spanningen en stromen. Er zijn diverse technieken om zulke stelsels op te lossen, bijvoorbeeld Gauss-eliminatie en de regel van Cramer.

De vergelijking:
\begin{equation}
5x_1 + 2x_2 + 8x_3 - 6x_4 = 12
\end{equation}
heet een lineaire vergelijking in de onbekenden $x_1$, $x_2$, $x_3$ en $x_4$. De getallen $5$, $2$, $8$, $-6$ en $12$ heten de coëfficiënten van deze vergelijking; het getal 12 wordt ook wel het rechterlid genoemd. Deze vergelijking heeft een oneindig aantal oplossingen voor $x_1$ t/m $x_4$. Als de onbekenden aan meerdere vergelijkingen moeten voldoen, dan spreken we van een stelsel van vergelijkingen. Een stelsel van $n$ lineaire vergelijkingen met $n$ onbekenden is in principe oplosbaar, maar er zijn wel voorwaarden aan verbonden.

Om twee vergelijkingen met twee onbekenden op te lossen, kunnen we de \textsl{regel van Cramer} gebruiken. Aanschouw onderstaande vergelijkingen:
%
\begin{equation}
\begin{split}
a_{11}x_1 + a_{12}x_2 &= b_1 \\
a_{21}x_1 + a_{22}x_2 &= b_2
\end{split}
\end{equation}
%
Dit kan worden geschreven in een matrixnotatie:
%
\begin{equation}
\begin{bmatrix}
a_{11} & a_{12} \\
a_{21} & a_{22}
\end{bmatrix} \cdot
\begin{bmatrix}
x_1 \\
x_2
\end{bmatrix} =
\begin{bmatrix}
b_1 \\
b_2
\end{bmatrix}
\end{equation}
%
De regel van Cramer zegt nu het volgende:
\begin{itemize}
\item Bereken de hoofddeterminant (dat is $a_{11}$ t/m $a_{22}$ tussen haken)
\item Bereken de hulpdeterminant voor de te vinden variabele. Om $x_1$ uit te rekenen moet de eerste
kolom van de hoofddeterminant ($a_{11}$ en $a_{21}$) vervangen worden door de $b$-vector (kolom met
$b_1$ en $b_2$). Om $x_2$ te vinden moeten $a_{12}$ en $a_{22}$ vervangen worden door $b_1$ en $b_2$.
\end{itemize}
%
De hoofddeterminant is:
%
\begin{equation}
\det A = \begin{vmatrix}
a_{11} & a_{12} \\
a_{21} & a_{22}
\end{vmatrix} = a_{11}a_{22} - a_{12}a_{21}
\end{equation}
%
Het stelsel is oplosbaar dan en slechts dan als de hoofddeterminant ongelijk aan 0 is.
%
De hulpdeterminanten voor $x_1$ en $x_2$ zijn
%
\begin{equation}
\det x_1 = \begin{vmatrix}
b_{1} & a_{12} \\
b_{2} & a_{22}
\end{vmatrix} = b_{1}a_{22} - a_{12}b_{2}
\end{equation}
%
en
%
\begin{equation}
\det x_2 = \begin{vmatrix}
a_{11} & b_{1} \\
a_{21} & b_{2}
\end{vmatrix} = a_{11}b_{2} - b_{1}a_{21}
\end{equation}
%
Nu kunnen $x_1$ en $x_2$ als volgt gevonden worden:
%
\begin{equation}
x_1 = \dfrac{\det x_1}{\det A} = \dfrac{\begin{vmatrix}
b_{1} & a_{12} \\
b_{2} & a_{22}
\end{vmatrix}}{\begin{vmatrix}
a_{11} & a_{12} \\
a_{21} & a_{22}
\end{vmatrix}} = \dfrac{b_{1}a_{22} - a_{12}b_{2}}{a_{11}a_{22} - a_{12}a_{21}}
\end{equation}
%
en
%
\begin{equation}
x_2 = \dfrac{\det x_2}{\det A} = \dfrac{\begin{vmatrix}
a_{11} & b_{1} \\
a_{21} & b_{2}
\end{vmatrix}}{\begin{vmatrix}
a_{11} & a_{12} \\
a_{21} & a_{22}
\end{vmatrix}} = \dfrac{a_{11}b_{2} - b_{1}a_{21}}{a_{11}a_{22} - a_{12}a_{21}}
\end{equation}

\begin{example}[Oplossen twee vergelijkingen met twee onbekenden]
Een getallenvoorbeeld:
%
\begin{equation}
\begin{split}
4x_1 - x_2 &= 24 \\
-2x_1 + 5x_2 &= 12
\end{split}
\end{equation}
%
In matrixnotatie wordt dit:
\begin{equation}
\begin{bmatrix*}[r]
4 & -1 \\
-2 & 5
\end{bmatrix*} \cdot
\begin{bmatrix}
x_1 \\ x_2
\end{bmatrix} =
\begin{bmatrix}
24 \\ 12
\end{bmatrix}
\end{equation}
%
Voor $x_1$ wordt nu gevonden:
%
\begin{equation}
x_1 = \dfrac{\begin{vmatrix}
24 & -1 \\
12 & 5
\end{vmatrix}}{\begin{vmatrix}
4 & -1 \\
-2 & 5
\end{vmatrix}} = \dfrac{24\cdot5 - (-1)\cdot12}{4\cdot5 - (-1)\cdot(-2)} = \dfrac{132}{18} = 7\dfrac{1}{3}
\end{equation}
%
en voor $x_2$ wordt gevonden:
%
\begin{equation}
x_2 = \dfrac{\begin{vmatrix}
4 & 24 \\
-2 & 12
\end{vmatrix}}{\begin{vmatrix}
4 & -1 \\
-2 & 5
\end{vmatrix}} = \dfrac{4\cdot12-24\cdot(-2)}{4\cdot5 - (-1)\cdot(-2)} = \dfrac{96}{18} = 5\dfrac{1}{3}
\end{equation}
\end{example}

De regel van Cramer werkt ook met stelsels met meer dan twee vergelijkingen, maar het berekenen van de diverse determinanten wordt dan wel lastiger. Om een stelsel van drie vergelijkingen op te lossen moeten vier determinanten worden uitgerekend. Gegeven is het stelsel:
%
\begin{equation}
\begin{split}
a_{11}x_1 + a_{12}x_2 + a_{13}x_3 &= b_1 \\
a_{12}x_1 + a_{22}x_2 + a_{23}x_3 &= b_2 \\
a_{32}x_1 + a_{32}x_2 + a_{33}x_3 &= b_3
\end{split}
\end{equation}
%
of in matrixnotatie:
%
\begin{equation}
\begin{bmatrix}
a_{11} & a_{12} & a_{13} \\
a_{21} & a_{22} & a_{23} \\
a_{31} & a_{32} & a_{33}
\end{bmatrix} \cdot
\begin{bmatrix}
x_1 \\
x_2 \\
x_3
\end{bmatrix} =
\begin{bmatrix}
b_1 \\
b_2 \\
b_3
\end{bmatrix}
\end{equation}

De hoofddeterminant $A$ is te berekenen met:
%
\begin{equation}
\begin{split}
\det A &= \begin{vmatrix}
a_{11} & a_{12} & a_{13} \\
a_{21} & a_{22} & a_{23} \\
a_{31} & a_{32} & a_{33}
\end{vmatrix} \\&=
 a_{11}a_{22}a_{33}+a_{12}a_{23}a_{31}+a_{13}a_{21}a_{32} - a_{13}a_{22}a_{31} -a_{12}a_{21}a_{33} - a_{11}a_{23}a_{32}
\end{split}
\end{equation}

Voor het berekenen van $x_1$ moet de eerste kolom vervangen worden door de $b$-vector en de hulpdeterminant worden berekend. De hulpdeterminant gedeeld door de hoofddeterminant geeft dan de waarde van $x_1$.
%
\begin{equation}
\begin{split}
\det x_1 &= \begin{vmatrix}
b_{1} & a_{12} & a_{13} \\
b_{2} & a_{22} & a_{23} \\
b_{3} & a_{32} & a_{33}
\end{vmatrix} \\&=
 b_{1}a_{22}a_{33}+a_{12}a_{23}b_{3}+a_{13}b_{2}a_{32} - a_{13}a_{22}b_{3} -a_{12}b_{2}a_{33} - b_{1}a_{23}a_{32}
\end{split}
\end{equation}
%
Op deze manier kunnen ook $x_2$ en $x_3$ worden berekend. De procedure is niet ingewikkeld maar het rekenwerk is aanzienlijk.

\begin{example}[Oplossen van drie vergelijkingen met drie onbekenden]
Gegeven is het stelsel:
%
\begin{equation}
\begin{split}\label{equ:linsystem3}
4x_1 + -5x_2 + 7x_3 &= 6 \\
-x_1 + 3x_2 + -2x_3 &= 7 \\
12x_1 + 3x_2 + 9x_3 &= 8
\end{split}
\end{equation}
%
of in matrixnotatie:
%
\begin{equation}
\begin{bmatrix}
4 & -5 & 7 \\
-1 & 3 & -2 \\
12 & 3 & 9
\end{bmatrix} \cdot
\begin{bmatrix}
x_1 \\
x_2 \\
x_3
\end{bmatrix} =
\begin{bmatrix}
6 \\
7 \\
8
\end{bmatrix}
\end{equation}
%
De hoofddeterminant levert:
\begin{equation}
\det A = 
\begin{vmatrix}
4 & -5 & 7 \\
-1 & 3 & -2 \\
12 & 3 & 9
\end{vmatrix} = -66 
\end{equation}
%
Voor de determinant van $x_1$ wordt gevonden:
\begin{equation}
\det x_1 = 
\begin{vmatrix}
6 & -5 & 7 \\
7 & 3 & -2 \\
8 & 3 & 9
\end{vmatrix} = -572 
\end{equation}
%
Nu is $x_1$ uit te rekenen:
%
\begin{equation}
x_1 = \dfrac{\det x_1}{\det A} = \dfrac{-572}{-66} = -8\dfrac{2}{3}
\end{equation}
%
Op vergelijkbare wijze wordt voor $x_2$ en $x_3$ gevonden:
%
\begin{equation}
x_2 = \dfrac{\det x_2}{\det A} = \dfrac{-418}{-66} = 6\dfrac{1}{3} \qquad\text{en}\qquad x_3 = \dfrac{\det x_3}{\det A} = \dfrac{-682}{-66} = 10\dfrac{1}{3}
\end{equation}
\end{example}

Met behulp van de programmeertaal Python kunnen gemakkelijk determinanten berekend en lineaire stelsels opgelost worden. De bibliotheek \lstinline|numpy| bevat alle noodzakelijke routines. In listing~\ref{cod:lindet} is te zien hoe de determinant van een 3x3-matrix moet worden berekend. Het betreft hier reële coëfficiënten.
%
\begin{lstlisting}[language=Python,caption=Berekenen van de determinant van een 3x3-matrix.,label=cod:lindet]
import numpy as np

matrix = [[-3, -5, 6], [1, -3, 4], [8, 5, 1]]
det = np.linalg.det(matrix)

print det                 # prints 88.0
\end{lstlisting}

Het is in Python mogelijk om complexe getallen te gebruiken, bijvoorbeeld bij netwerken met complexe impedanties. Zie listing~\ref{cod:lindet2}.
%
\begin{lstlisting}[language=Python,caption=Berekenen van de determinant van een 3x3-matrix met complexe getallen.,label=cod:lindet2]
import numpy as np

matrix = [[6+2j, 1-1j, 3+5j], [3j, -3, 4+4j], [8, 5, 1]]
det = np.linalg.det(matrix)

print det            # prints (-40-4j)
\end{lstlisting}

Een lineair stelsel van $n$ vergelijkingen en $n$ onbekenden is op te lossen met de \lstinline|solve|-routine. Ook hier kan weer gebruik gemaakt worden van complexe getallen. In listing~\ref{cod:linsolve} is het Python-programma te zien dat de oplossing van het lineaire systeem uit~\eqref{equ:linsystem3} berekent. Als de hoofddeterminant ongelijk is aan 0, kan de oplossing berekend worden.
%
\begin{lstlisting}[language=Python,caption=Berekenen van de oplossing van een lineair systeem met drie vergelijkingen.,label=cod:linsolve]
# Use the numpy numeric library
import numpy as np

a = np.array([[4,-5,7], [-1,3,-2], [12,3,9]])   # Matrix of coeffs
b = np.array([6,7,8])                           # Vector of results

try:
        x = np.linalg.solve(a, b)               # Solve lin system
        print "Solution:", x
        print "Correct solution:", np.allclose(np.dot(a, x), b)
except:
        print "No solutions for this system"
\end{lstlisting}