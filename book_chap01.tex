\chapter{Introductie}
\label{cha:introductie}

Elektriciteit is al zo oud als het ontstaan van het heelal. Men gaat ervan uit dat in het vroege heelal een scheiding heeft plaatsgevonden waardoor de deeltjes die we nu kennen zijn ontstaan. Sommige deeltjes hebben een \textsl{lading}, andere hebben dat niet. Die lading veroorzaakt elektriciteit. Nu zullen we in dit boek ons niet bezighouden met de precieze oorsprong van elektriciteit, maar we willen toch wel enkele begrippen introduceren.

Elektrische eigenschappen zijn lang bestudeerd, zelfs in de oudheid. Maar echte studies naar het fenomeen elektriciteit is begonnen in de zeventiende en achttiende eeuw. Diverse onderzoekers hebben onafhankelijk van elkaar experimenten opgezet om elektriciteit te onderzoeken. Bekend is Alessandro Volta, die als eerste een batterij heeft ontwikkeld met behulp van een chemische reactie. Naar hem is de eenheid van spanning genoemd, de volt. André-Marie Ampère deed onderzoek deed naar elektromagnetisme en legde een fundament voor elektrodynamica. Naar hem is de elektrische stroom vernoemd, de ampère. Er zijn meer bekende onderzoekers zoals Gauss, die de relatie tussen een elektrische lading en een elektrische veld vastlegt, en Faraday die de kooi van Faraday ontwierp, een ruimte die een elektromagnetisch veld blokkeert. Natuurlijk is ook Nikola Tesla bekend door zijn experimenten met het draadloos overbrengen van elektrische energie. Naar hem is de eenheid van magnetische fluxdichtheid vernoemd. Het was Maxwell die als eerste de relatie tussen de elektrische en magnetische velden beschreef en dat resulteerde in de vergelijkingen van Maxwell. Deze wetten zijn de grondslagen van de elektriciteitsleer.


\section{Spanning en stroom}
De elektrische grootheden spanning en stroom komen voort uit de theorie van de \textsl{elektromagnetische velden}. Een elektromagnetisch veld is een natuurkundig verschijnsel geproduceerd door elektrisch geladen objecten. Het is een van de vier fundamentele krachten in de natuur, naast de zwaartekracht, de zwakke wisselwerking en sterke wisselwerking. Het is lastig om een elektromagnetisch veld voor te stellen, het is immers niet te zien. We kunnen het wel meten met behulp van een \textsl{veldsterktemeter}. Elektromagnetische velden worden geproduceerd door geladen deeltjes.

We kunnen een elektromagnetisch veld grofweg verdelen in twee componenten: een \textsl{elektrisch veld} wordt veroorzaakt door stilstaande lading. Een elektrisch veld beschrijft de grootte en richting van elektrische krachten in de ruimte bij een gegeven ruimtelijke ladingsverdeling. Een \textsl{magnetisch veld} is een veld dat de ruimte doordringt en dat een magnetische kracht op bewegende elektrische ladingen weergeeft. De combinatie van deze velden wordt veroorzaakt door materie.

Materie bestaat uit atomen die zijn opgebouwd uit protonen, neutronen en elektronen. Protonen en elektronen hebben een \textsl{lading}. Lading wordt uitgedrukt in coulomb, afgekort tot \si{\coulomb}. Protonen hebben een positieve lading ter grootte van de \textsl{elementaire lading}. Deze lading is gedefinieerd als \SI{1.602176634e-19}{\coulomb} (vanaf 20 mei 2019) en wordt doorgaans aangegeven met de constante $e$ (niet te verwarren met e, het grondtal van de natuurlijke logaritme). Elektronen hebben dezelfde lading maar dan tegengesteld, dus $-e$. Neutronen hebben geen lading. De keuze voor positieve en negatieve lading is ingegeven door het historische feit dat de positieve lading (althans, wat we verstaan onder positieve lading) eerder is ontdekt dan negatieve lading. Pas later werd bekend dat elektriciteit grotendeels wordt veroorzaakt door bewegende elektronen.

Metalen zijn goede geleiders van elektriciteit. Metaalatomen zijn opgelijnd in een zogenoemd kristalrooster. De elektronen in de buitenste schil van een metaalatoom zijn wat minder sterk gebonden en kunnen zich vrijelijk bewegen. Ze springen van de ene atoom naar de andere atoom. Als een elektron migreert naar een atoom, dan wordt dit atoom tijdelijk positief geladen. Er is een \textsl{gat} ontstaan. Het atoom waar de elektron is naar toe gaat, is dan negatief geladen. Dit ladingsverschil zorgt ervoor dat elektronen zich naar het gat bewegen. Het nettoresultaat van de lading in een metalen geleider is dus 0.

%The moving charged particles in an electric current are called charge carriers. In metals, one or more electrons from each atom are loosely bound to the atom, and can move freely about within the metal. These conduction electrons are the charge carriers in metal conductors.

Elektrische spanning is het verschil in \textsl{potentiële elektrische energie} tussen twee punten per eenheid van lading. In het SI-stelsel kan dit worden uitgedrukt in joule per coulomb (\si[per-mode=symbol]{\joule\per\coulomb}), maar spanning heeft een eigen eenheid gekregen: de \textsl{volt}, afgekort tot \si{\volt}. Als symbool voor de elektrische spanning wordt $U$ gebruikt maar in Engelstalige literatuur wordt $V$ gebruikt. Als we een spanningsverschil over een metalen geleider plaatsen, dan geven we de elektronen de mogelijkheid om deze potentiële energie te benutten. Het gevolg is dat de elektronen zich makkelijker kunnen losmaken van de atomen. Omdat positieve en negatieve lading elkaar aantrekken, verplaatsen de elektronen zich naar de positieve spanningspunt. Zodoende vormt zich een \textsl{elektrische stroom}. Er ontstaat dan een tekort aan elektronen aan het negatieve spanningspunt. Om het evenwicht te bewaren, moeten de spanningsbron en de metalen geleider dus een gesloten lus vormen zodat de elektronen vrijelijk kunnen bewegen. We noemen dit een \textsl{stroomkring}. Is de lus niet gesloten, dan ontstaat er geen stroom.

Historisch gezien loopt de stroom van de positieve spanning naar de negatieve spanning, maar \textsl{de elektronenstroom} loopt van de negatieve spanning naar de positieve spanning. Elektrische stroom is de hoeveelheid verplaatste lading per tijdseenheid, en wordt uitgedrukt in ampère, afgekort tot \si{\ampere}. Als symbool wordt $I$ gebruikt. Niet de lading maar de elektrische stroom is een van de zeven grondgrootheden van het SI-stelsel. Lading is dus uit te drukken als stroom keer tijd, oftewel \si{\ampere\second} (ampère-seconde). We zouden verwachten dat de elektronen zich zeer snel verplaatsen in de geleider. Een elektrisch lichtpunt gaat immers gelijk branden als we de schakelaar omhalen. Toch is dat niet zo. De \textsl{driftsnelheid} van elektronen , de snelheid waarmee elektronen zich verplaatsen, is klein: ongeveer \SI[per-mode=symbol]{0,1}{\milli\meter\per\second}. Waarom een lichtpunt gelijk gaat branden komt omdat de (langzaam) bewegende elektronen tegen elkaar botsen en zo hun \textsl{impuls} doorgeven. Het is te vergelijken met een stilstaande rij, tegen elkaar geplaatste, biljartballen. Als er een biljartbal tegen het uiteinde van de rij botst, zal aan bij het uiteinde van de rij een biljartbal snelheid krijgen. De tussenliggende biljartballen verplaatsen zich niet, ze geven alleen de impuls door. 


%dus uitgedrukt in \si{\coulomb\per\second}. 


%Elektrische stroom is beweging van elektrische lading. De elektrische stroom die in een metalen geleider vloeit, zoals koper en aluminium, bestaat uit verplaatsende elektronen. De protonen en neutronen in de geleider verplaatsen zich niet. Als op een metalen geleider een spanning (spanningsverschil) wordt aangebracht, bewegen de elektronen zich naar de positieve spanning.  





\subsection{Definitie van stroom}
De elektrische stroom is een van de zeven basisgrootheden van het SI-stelsel. Een elektrische stroom is niets anders dan een hoeveelheid lading die per seconde door het oppervlakte van de dwarsdoorsnede van een geleider vloeit. Een stroom van \SI{1}{\ampere} is gelijk aan een lading van \SI{1}{\coulomb} per seconde door een geleider:
%
\begin{equation}
\SI{1}{\ampere} = \SI[per-mode=fraction]{1}{\coulomb\per\second}
\end{equation} 

\subsubsection*{Tot en met 19 mei 2019}
In twee rechte geleiders loopt een stroom van \SI{1}{\ampere} als de twee geleiders van oneindige lengte en verwaarloosbare diameter, geplaatst in vacuüm op een afstand van \SI{1}{\meter}, een kracht op elkaar uitoefenen van \SI[per-mode=symbol]{2e-7}{\newton\per\meter}.

\subsubsection*{Vanaf 20 mei 2019}
Vanaf deze datum is de elementaire lading vastgesteld op \SI{1.602176634e-19}{\coulomb}. De stroom is gedefinieerd op basis van de elementaire lading. Dit zorgt ook voor de herdefinitie van de coulomb. In een geleider loopt een stroom van \SI{1}{\ampere} als per seconde $\frac{1}{\num{1.602176634e-19}}$ elektronen passeren.

\subsection{Definitie van spanning}
Spanning is een afgeleide grootheid. Het is gedefinieerd op basis van \textsl{vermogen} en stroom. Over een geleider staat een spanning (of spanningsverschil) van \SI{1}{\volt} als in de geleider een vermogen wordt gedissipeerd van \SI{1}{\watt} bij een stroom van \SI{1}{\ampere}. Dit is uitgebeeld in figuur~\ref{fig:intdefinitievolt}.

\begin{figure}[!ht]
\centering
\begin{tikzpicture}[bookcircuit]
\draw (0,0) to [V=\SI{1}{\volt}] ++(0,2) to [short, i=\SI{1}{\ampere}] ++(2.5,0) to [short,n=R] ++(0,-2) to [short,-.] (0,0);
\draw[decorate, decoration={snake, segment length=5pt, amplitude=1pt},-stealth'] (R) ++(0.2,0.2) -> ++(2,0.2);
 \draw[decorate, decoration={snake, segment length=5pt, amplitude=1pt},-stealth'] (R) ++(0.2,0.0) -> ++(2,0.0) node[right] {\SI{1}{\watt}};
\draw[decorate, decoration={snake, segment length=5pt, amplitude=1pt},-stealth'] (R) ++(0.2,-0.2) -> ++(2,-0.2); \end{tikzpicture}
\caption{Definitie van de volt.}
\label{fig:intdefinitievolt}
\end{figure}

\subsection{Definitie van vermogen}
Vermogen is een afgeleide grootheid. Het is het product van de spanning over en stroom door een geleider en wordt uitgedrukt in watt, afgekort tot \si{\watt}:
%
\begin{equation}
\SI{1}{\watt} = \SI{1}{\volt}\cdot\si{\ampere} = \SI[per-mode=fraction]{1}{\joule\per\coulomb}\cdot\si[per-mode=fraction]{\coulomb\per\second} = \SI[per-mode=fraction]{1}{\joule\per\second}
\end{equation}
%
Dit is precies wat vermogen inhoudt: energieafgifte (of opname) per tijdseenheid.

\section{Meten van spanning en stroom}
Om een spanning tussen twee punten te meten, gebruiken we een \textsl{spanningsmeter}. De spanningsmeter heeft twee ingangen, de `+'-ingang en de `$-$'-ingang. Als de spanning op de `+'-ingang groter is dan de spanning op de `$-$'-ingang, dan geeft de meter een positieve spanning aan. Omgekeerd heeft de meter een negatieve spanning aan.

Om een stroom te meten in een (metalen) geleider, gebruiken we een stroommeter. Ook de stroommeter heeft twee aansluitpunten, de `+'-ingang en de `$-$'-ingang. Als de stroom van `+' naar `$-$' vloeit, dan meten we een positieve stroom. Als de stroom van `$-$' naar `+' vloeit, dan meten we een negatieve stroom.

Een meting moet de stromen en spanningen in het elektrisch netwerk niet beïnvloeden. Helaas zijn praktische meters

\textsl{Een meting beïnvloedt altijd de spanningen en stromen in het netwerk.}

We kunnen er alleen voor zorgen dat de meting de stromen en spanningen zeer licht beïnvloeden.

\begin{infobox}[Naming and shaming\ldots]
Een spanningsmeter wordt in de volksmond ook wel een \textsl{voltmeter} genoemd. Dit is echter niet juist. Er wordt een grootheid gemeten, spanning, die wordt uitgedrukt in volt. Evenzo wordt voor het meten van een stroom een stroommeter gebruikt, in de volksmond een \textsl{ampèremeter} genoemd. Ook dit is niet juist. Een stroommeter meet de grootheid stoom en die wordt uitgedrukt in ampère. En zo kunnen we doorgaan: een vermogensmeter en geen \textsl{wattmeter}, een frequentiemeter en geen \textsl{hertzmeter}, een thermometer en geen \textsl{kelvinmeter} (of graden-celsiusmeter)

In Amerika wordt echter wel de termen \textsl{voltmeter} en \textsl{ammeter} gebruikt.

Soms wordt voor spanning het woord \textsl{voltage}, voor stroom \textsl{ampèrage} en voor vermogen \textsl{wattage} gebruikt. Allemaal onjuist, hoewel voltage nog wel enigszins kan.
\end{infobox}


\section{Geleiders}
Een \textsl{geleider} is een materiaal dat elektriciteit goed geleidt. Metalen zijn over het algemeen goede geleiders. Een tabel met soortelijke weerstand is te vinden in tabel~\ref{cha:intsoortmet}.

\begin{table}[!ht]
\centering
\captionsetup{width=.7\linewidth}
\caption{Soortelijke weerstand $\rho$ in \si{\ohm\meter} bij \SI{20}{\celsius} en temperatuurcoëfficiënt $\alpha$ in \si{\per\kelvin} van enkele materialen.}
\label{cha:intsoortmet}
\begin{tabular}{lSS}
\toprule
Materiaal & {Soortelijke weerstand} & {Temp. coëfficiënt} \\
\midrule
Zilver & 1.59e-8 & 0.0041 \\
Koper & 1.75e-8 & 0.0039 \\
Goud & 2.20e-8 & 0.0036 \\
Aluminium & 2.65e-8 & 0.0043 \\ 
%&&\\
%Glas & 1e12 & {$-$} \\
\bottomrule
\end{tabular}
\end{table}

Zilver is het beste qua geleiding, maar koper is favoriet bij het implementeren van stroomdraden omdat dit metaal meer beschikbaar is.


\section{Isolatoren}
Een \textsl{isolator} is een materiaal die elektriciteit slecht geleidt. Bekende isolatoren zij kunststof, glas en porselein. Bekend is dat een stroomdraad wordt omhuld met plastic. Hierdoor kan geen contact worden gemaakt met andere geleiders.


\section{Halfgeleiders}


\section{Elektrische netwerken}
Een elektrisch netwerk is een samenstelling van elektrische componenten. We kunnen de elektrische componenten verdelen in actieve en passieve elementen. De actieve elementen zijn spanningsbronnen en stroombronnen. De passieve elementen zijn weerstand, condensator, spoel, transformator er gyrator. Om een elektrisch netwerk te bestuderen gebruiken we grafische symbolen om de elektrische componenten weer te geven.

De actieve elementen zijn de spanningsbron en stroombron. In principe leveren ze elektrische energie:

\begin{minipage}{0.2\textwidth}
\centering
\begin{tikzpicture}[bookcircuit]
\draw (0,0) to [V=$U$] ++(0,2);
\end{tikzpicture}
\end{minipage}\hfill%
\begin{minipage}{0.78\textwidth}
Een ideale spanningsbron is een netwerkelement dat een constante spanning geeft, onafhankelijk van de stroom die de bron levert. De spanning is positief tussen de punt `+' en `$-$'. Een spanning wordt aangegeven met de letter $U$.
\end{minipage}

\begin{minipage}{0.2\textwidth}
\centering
\begin{tikzpicture}[bookcircuit]
\draw (0,0) to [I, l=$I$] ++(0,2);
\end{tikzpicture}
\end{minipage}\hfill%
\begin{minipage}{0.78\textwidth}
Een ideale stroombron is een netwerkelement dat een constante stroom geeft, onafhankelijk van de spanning die over bron staat. De stroom is positief in de richting van de pijl. Een stroom wordt aangegeven met de letter $I$.
\end{minipage}

De passieve elementen zijn de weerstand, condensator, spoel, transformator en gyrator.

\begin{minipage}{0.2\textwidth}
\centering
\begin{tikzpicture}[bookcircuit]
\draw (0,0) to [R=$R$] ++(0,2);
\end{tikzpicture}
\end{minipage}\hfill%
\begin{minipage}{0.78\textwidth}
Een ideale weerstand is een netwerkelement waarvoor geldt dat de spanning over de weerstand rechtevenredig is met de stroom door de weerstand. Een weerstand wordt aangegeven met de letter $R$. Deze evenredigheid wordt aangegeven door $R$ middels de formule $U=RI$.
\end{minipage}

\begin{minipage}{0.2\textwidth}
\centering
\begin{tikzpicture}[bookcircuit]
\draw (0,0) to [C=$C$] ++(0,2);
\end{tikzpicture}
\end{minipage}\hfill%
\begin{minipage}{0.78\textwidth}
Een ideale condensator is een netwerkelement waarvoor geldt dat de lading in de condensator rechtevenredig is met de spanning over de condensator. Een condensator wordt aangegeven met de letter $C$. Deze evenredigheid wordt aangegeven door $C$ middels de formule $Q=CU$.
\end{minipage}

\begin{minipage}{0.2\textwidth}
\centering
\begin{tikzpicture}[bookcircuit]
\draw (0,0) to [L=$L$] ++(0,2);
\end{tikzpicture}
\end{minipage}\hfill%
\begin{minipage}{0.78\textwidth}
Een ideale spoel is een netwerkelement waarvoor geldt dat de magnetische flux rechtevenredig is met de stroom door de spoel. Een spoel wordt aangegeven met de letter $L$. Deze evenredigheid wordt aangegeven met $L$ middels de formule $\Phi = LI$.
\end{minipage}

\begin{minipage}{0.2\textwidth}
\centering
\begin{tikzpicture}[bookcircuit]
\draw (0,0) node[transformer] (S) {};
\end{tikzpicture}
\end{minipage}\hfill%
\begin{minipage}{0.78\textwidth}
Een ideale transformator zorgt ervoor dat de ingaande wisselspanning of -stroom rechtevenredig wordt omgezet in een uitgaande wisselspanning of -stroom. De verhouding van de spanningen aan de linker- en rechterzijde zijn evenredig met de verhoudingen met het aantal windingen aan de linker- en rechterzijde: $U_L/U_R = N_L/N_R$.
\end{minipage}

\begin{minipage}{0.2\textwidth}
\centering
\begin{tikzpicture}[bookcircuit]
\draw (0,0) node[gyrator] (S) {};
\end{tikzpicture}
\end{minipage}\hfill%
\begin{minipage}{0.78\textwidth}
Een ideale gyrator zorgt ervoor dat de ingaande wisselspanning of -stroom rechtevenredig wordt omgezet in een uitgaande wisselspanning of -stroom. De 
\end{minipage}












\section{Spanningsbronnen en stroombronnen}
Het leveren van elektriciteit wordt gerealiseerd door spanningsbronnen en stroombronnen.

\begin{figure}[!ht]
\centering
\begin{tikzpicture}[bookcircuit]
\draw (0,0) to[V=$U_1$] ++(0,2)
            to[short,-*] ++(1,0)
            to[short,-o] +(0,0.5)
            to[open] +(0,-0)
			to[short] ++(1,0)
			to[V<=$U_2$] ++(0,-2)
            to[short,-*] ++(-1,0)
            to[short,-o] ++(0,-0.5)
            to[open] ++(0,0.5)
            to[short,-.] (0,0)
;
\draw[color=red] (current bounding box.south west) -- (current bounding box.north east);
\draw[color=red] (current bounding box.north west) -- (current bounding box.south east);
\draw (6,-1) to[V=$U_1$,o-] ++(0,2)
			to[V=$U_2$,-o] ++(0,2)
;
\end{tikzpicture}
\caption{Twee spanningsbronnen mogen niet parallel geschakeld worden, maar wel in serie.}
\label{fig:intvolagesources}
\end{figure}

\begin{figure}[!ht]
\centering
\begin{tikzpicture}[bookcircuit]
\draw (0,0) to[I, l=$I_1$,o-] ++(0,2)
			to[I, l=$I_2$,-o] ++(0,2)
;
\draw[color=red] (current bounding box.south west) -- (current bounding box.north east);
\draw[color=red] (current bounding box.north west) -- (current bounding box.south east);
\draw (4,1) to[I, l=$I_1$] ++(0,2)
            to[short,-*] ++(1,0)
            to[short,-o] +(0,0.5)
            to[open] +(0,-0)
			to[short] ++(1,0)
			to[I, l=$I_2$,invert] ++(0,-2)
            to[short,-*] ++(-1,0)
            to[short,-o] ++(0,-0.5)
            to[open] ++(0,0.5)
            to[short,-.] (4,1)
;
\end{tikzpicture}
\caption{Twee stroombronnen mogen niet in serie geschakeld worden, maar wel parallel.}
\label{fig:intcurrentsources}
\end{figure}



%\section{Opgaven}
%\input{book_chap01_ques}