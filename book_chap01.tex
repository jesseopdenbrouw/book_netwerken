\chapter{Introductie}
\label{cha:introductie}


%\section{Opgaven}
%\input{book_chap01_ques}

\section{Spanning en stroom}
De elektrische grootheden spanning en stroom komen voort uit de theorie van de \textsl{elektromagnetische velden}. Een elektromagnetisch veld is een natuurkundig verschijnsel geproduceerd door elektrisch geladen objecten. Het is een van de vier fundamentele krachten in de natuur, naast de zwaartekracht, de zwakke wisselwerking en sterke wisselwerking.

Materie bestaat uit atomen en atomen zijn opgebouwd uit protonen, neutronen en elektronen. Protonen hebben een positieve lading ter grootte van de \textsl{elementaire lading}. Deze lading is gedefinieerd als \SI{1.602176634e-19}{\coulomb} en wordt doorgaangs aangegeven met de constante $e$ (niet te verwarren met e, het grondtal van de natuurlijke logaritme). Elektronen hebben dezelfde lading maar dan tegengesteld, dus $-e$. Neutronen hebben geen lading.

Elektrische stroom is beweging van elektrische lading. De elektrische stroom die in een metalen geleider vloeit, zoals koper en aluminium, bestaat uit verplaatsende elektronen. De protonen en neutronen in de geleider verplaatsen zich niet. Als op een metalen geleider een spanning (spanningsverschil) wordt aangebracht, bewegen de elektronen zich naar de positieve spanning. De \textsl{driftsnelheid} van elektronen is klein: ongeveer \SI[per-mode=symbol]{0,1}{\milli\meter\per\second}. 


Elektrische spanning is het verschil in \textsl{potentiële elektrische energie} tussen twee punten per eenheid van lading. In het SI-stelsel wordt dit uitgedrukt in \textsl{volt} (afgekort tot \si{\volt}). Als symbool voor de elektrische spanning wordt $U$ gebruikt maar in Engelstalige literatuur wordt $V$ gebruikt.


Electric current (electricity) is a flow or movement of electrical charge. The electricity that is conducted through copper wires in your home consists of moving electrons. The protons and neutrons of the copper atoms do not move. The actual progression of the individual electrons in a given direction through the wire is quite slow. The electrons have to work their way through the billions of atoms in the wire and this takes considerable time. In the case of a 12 gauge copper wire carrying 10 amperes of current (typical of home wiring), the individual electrons only move about 0.02 cm per sec or 1.2 inches per minute (in science this is called the drift velocity of the electrons.). If this is the situation in nature, why do the lights come on so quickly? At this speed it would take the electrons hours to get to the lights.

\subsection{Definitie van stroom}
De elektrische stroom is een van de zeven basisgrootheden van het SI-stelsel. Een elektrische stroom is niets anders dan een hoeveelheid lading die per seconde door het oppervlakte van de dwarsdoorsnede van een geleider vloeit:
%
\begin{equation}
\SI{1}{\ampere} = \SI[per-mode=fraction]{1}{\coulomb\per\second}
\end{equation} 

\subsubsection*{Tot en met 19 mei 2019}
In twee rechte geleiders loopt een stroom van \SI{1}{\ampere} als de twee geleiders van oneindige lengte en verwaarloosbare diameter, geplaatst in vacuüm op een afstand van \SI{1}{\meter}, een kracht op elkaar uitoefenen van \SI{2e-7}{\newton\per\meter}.

\subsubsection*{Vanaf 20 mei 2019}
Vanaf deze datum is de elementaire lading vastgesteld op \SI{1.602176634e-19}{\coulomb}. De stroom gedefinieerd op basis van de elementaire lading. Dit zorgt ook voor de herdefinitie van de coulomb. In een geleider loopt een stroom van \SI{1}{\ampere} als per seconde een lading $\frac{1}{\num{1.602176634e-19}}\si{\coulomb}$ passeert.

\section{Meten van spanning en stroom}
Om een spanning tussen twee punten te meten, gebruiken we een \textsl{spanningsmeter}. De spanningsmeter heeft twee ingangen, de `+'-ingang en de `$-$'-ingang. Als de spanning op de `+'-ingang groter is dan de spanning op de `$-$'-ingang, dan geeft de meter een positieve spanning aan. Omgekeerd heeft de meter een negatieve spanning aan.

\begin{infobox}[Naming and shaming\ldots]
Een spanningsmeter wordt in de volksmond ook wel een \textsl{voltmeter} genoemd. Dit is echter niet juist. Er wordt een grootheid gemeten, spanning, die wordt uitgedrukt in volt. Evenzo wordt voor het meten van een stroom een stroommeter gebruikt, in de volksmond een \textsl{ampèremeter} genoemd. Ook dit is niet juist. Een stroommeter meet de grootheid stoom en die wordt uitgedrukt in ampère. En zo kunnen we doorgaan: een vermogensmeter en geen \textsl{wattmeter}, een frequentiemeter en geen \textsl{hertzmeter}, een thermometer en geen \textsl{kelvinmeter} (of graden-celsiusmeter)

In Amerika wordt echter wel de termen \textsl{voltmeter} en \textsl{ammeter} gebruikt.

Soms wordt voor spanning het woord \textsl{voltage}, voor stroom \textsl{ampèrage} en voor vermogen \textsl{wattage} gebruikt. Allemaal onjuist, hoewel voltage nog wel enigszins kan.
\end{infobox}



\begin{table}[!ht]
\centering
\caption{Soortelijke weerstand $\rho$ in \si{\ohm\meter} bij \SI{20}{\celsius} en temperatuurcoëfficiënt $\alpha$ in \si{\per\kelvin} van enkele materialen.}
\begin{tabular}{lSS}
\toprule
Materiaal & {Soortelijke weerstand} & {Temp. coëfficiënt} \\
\midrule
Zilver & 1.59e-8 & 0.0041 \\
Koper & 1.75e-8 & 0.0039 \\
Goud & 2.20e-8 & 0.0036 \\
Aluminium & 2.65e-8 & 0.0043 \\ 
%&&\\
%Glas & 1e12 & {$-$} \\
\bottomrule
\end{tabular}
\end{table}